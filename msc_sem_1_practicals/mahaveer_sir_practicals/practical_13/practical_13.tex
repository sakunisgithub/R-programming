\documentclass[11pt, a4paper]{article}\usepackage[]{graphicx}\usepackage[]{xcolor}
% maxwidth is the original width if it is less than linewidth
% otherwise use linewidth (to make sure the graphics do not exceed the margin)
\makeatletter
\def\maxwidth{ %
  \ifdim\Gin@nat@width>\linewidth
    \linewidth
  \else
    \Gin@nat@width
  \fi
}
\makeatother

\definecolor{fgcolor}{rgb}{0.345, 0.345, 0.345}
\newcommand{\hlnum}[1]{\textcolor[rgb]{0.686,0.059,0.569}{#1}}%
\newcommand{\hlsng}[1]{\textcolor[rgb]{0.192,0.494,0.8}{#1}}%
\newcommand{\hlcom}[1]{\textcolor[rgb]{0.678,0.584,0.686}{\textit{#1}}}%
\newcommand{\hlopt}[1]{\textcolor[rgb]{0,0,0}{#1}}%
\newcommand{\hldef}[1]{\textcolor[rgb]{0.345,0.345,0.345}{#1}}%
\newcommand{\hlkwa}[1]{\textcolor[rgb]{0.161,0.373,0.58}{\textbf{#1}}}%
\newcommand{\hlkwb}[1]{\textcolor[rgb]{0.69,0.353,0.396}{#1}}%
\newcommand{\hlkwc}[1]{\textcolor[rgb]{0.333,0.667,0.333}{#1}}%
\newcommand{\hlkwd}[1]{\textcolor[rgb]{0.737,0.353,0.396}{\textbf{#1}}}%
\let\hlipl\hlkwb

\usepackage{framed}
\makeatletter
\newenvironment{kframe}{%
 \def\at@end@of@kframe{}%
 \ifinner\ifhmode%
  \def\at@end@of@kframe{\end{minipage}}%
  \begin{minipage}{\columnwidth}%
 \fi\fi%
 \def\FrameCommand##1{\hskip\@totalleftmargin \hskip-\fboxsep
 \colorbox{shadecolor}{##1}\hskip-\fboxsep
     % There is no \\@totalrightmargin, so:
     \hskip-\linewidth \hskip-\@totalleftmargin \hskip\columnwidth}%
 \MakeFramed {\advance\hsize-\width
   \@totalleftmargin\z@ \linewidth\hsize
   \@setminipage}}%
 {\par\unskip\endMakeFramed%
 \at@end@of@kframe}
\makeatother

\definecolor{shadecolor}{rgb}{.97, .97, .97}
\definecolor{messagecolor}{rgb}{0, 0, 0}
\definecolor{warningcolor}{rgb}{1, 0, 1}
\definecolor{errorcolor}{rgb}{1, 0, 0}
\newenvironment{knitrout}{}{} % an empty environment to be redefined in TeX

\usepackage{alltt}

\usepackage[top = 1 in, bottom = 1 in, left = 1 in, right = 1 in ]{geometry}

\usepackage{amsmath, amssymb, amsfonts}
\usepackage{enumerate}
\usepackage{array}
\usepackage{multirow}
\usepackage{dingbat}
\usepackage{fontawesome5}
\usepackage{tasks}
\usepackage{bbding}
\usepackage{twemojis}
% how to use bull's eye ----- \scalebox{2.0}{\twemoji{bullseye}}
\usepackage{fontspec}
\usepackage{customdice}
% how to put dice face ------ \dice{2}

\title{MSMS 106 : Practical 13}
\author{Ananda Biswas}
\date{\today}

\newfontface\myfont{Myfont1-Regular.ttf}[LetterSpace=0.05em]
% how to use ---- {\setlength{\spaceskip}{1em plus 0.5em minus 0.5em} \fontsize{17}{20}\myfont --write text here-- \par}
\IfFileExists{upquote.sty}{\usepackage{upquote}}{}
\begin{document}

\maketitle


\section*{\faArrowAltCircleRight[regular] \textcolor{blue}{Objective}}

\hspace{1cm} Write an R program to generate all possible subsets of the set $\{1, 2, 3\}.$




\section*{\faArrowAltCircleRight[regular] \textcolor{blue}{R Program}}

\begin{knitrout}
\definecolor{shadecolor}{rgb}{0.969, 0.969, 0.969}\color{fgcolor}\begin{kframe}
\begin{alltt}
\hldef{x1} \hlkwb{<-} \hlkwd{c}\hldef{(}\hlnum{1}\hldef{,} \hlnum{2}\hldef{,} \hlnum{3}\hldef{)}
\end{alltt}
\end{kframe}
\end{knitrout}

\begin{knitrout}
\definecolor{shadecolor}{rgb}{0.969, 0.969, 0.969}\color{fgcolor}\begin{kframe}
\begin{alltt}
\hldef{generate_subset} \hlkwb{<-} \hlkwa{function}\hldef{(}\hlkwc{set}\hldef{)\{}
  \hlkwd{backtrack_subset}\hldef{(set,} \hlnum{1}\hldef{,} \hlkwd{c}\hldef{())}
\hldef{\}}

\hldef{backtrack_subset} \hlkwb{<-} \hlkwa{function}\hldef{(}\hlkwc{set}\hldef{,} \hlkwc{index}\hldef{,} \hlkwc{current_subset}\hldef{)\{}
  \hlkwa{if}\hldef{(index} \hlopt{>} \hlkwd{length}\hldef{(set))\{}
    \hlkwd{print}\hldef{(current_subset)}
  \hldef{\}} \hlkwa{else}\hldef{\{}
    \hldef{current_subset} \hlkwb{<-} \hlkwd{unique}\hldef{(}\hlkwd{c}\hldef{(current_subset, set[index]))}
    \hlkwd{backtrack_subset}\hldef{(set, index} \hlopt{+} \hlnum{1}\hldef{, current_subset)}

    \hldef{current_subset} \hlkwb{<-} \hldef{current_subset[}\hlopt{-}\hlkwd{length}\hldef{(current_subset)]}
    \hlkwd{backtrack_subset}\hldef{(set, index} \hlopt{+} \hlnum{1}\hldef{, current_subset)}
  \hldef{\}}
\hldef{\}}
\end{alltt}
\end{kframe}
\end{knitrout}

\begin{knitrout}
\definecolor{shadecolor}{rgb}{0.969, 0.969, 0.969}\color{fgcolor}\begin{kframe}
\begin{alltt}
\hlkwd{generate_subset}\hldef{(x1)}
\end{alltt}
\begin{verbatim}
## [1] 1 2 3
## [1] 1 2
## [1] 1 3
## [1] 1
## [1] 2 3
## [1] 2
## [1] 3
## numeric(0)
\end{verbatim}
\end{kframe}
\end{knitrout}




\section*{\faArrowAltCircleRight[regular] \textcolor{blue}{Conclusion}}

\smallpencil  \hspace{0.3cm} We get a class of all 8 subsets of $\{1, 2, 3\}$, thus a $\sigma-$field on $\Omega = \{1, 2, 3\}$.

\end{document}
