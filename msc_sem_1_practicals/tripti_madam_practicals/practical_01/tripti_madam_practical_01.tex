\documentclass[11pt, a4paper]{article}\usepackage[]{graphicx}\usepackage[]{xcolor}
% maxwidth is the original width if it is less than linewidth
% otherwise use linewidth (to make sure the graphics do not exceed the margin)
\makeatletter
\def\maxwidth{ %
  \ifdim\Gin@nat@width>\linewidth
    \linewidth
  \else
    \Gin@nat@width
  \fi
}
\makeatother

\definecolor{fgcolor}{rgb}{0.345, 0.345, 0.345}
\newcommand{\hlnum}[1]{\textcolor[rgb]{0.686,0.059,0.569}{#1}}%
\newcommand{\hlsng}[1]{\textcolor[rgb]{0.192,0.494,0.8}{#1}}%
\newcommand{\hlcom}[1]{\textcolor[rgb]{0.678,0.584,0.686}{\textit{#1}}}%
\newcommand{\hlopt}[1]{\textcolor[rgb]{0,0,0}{#1}}%
\newcommand{\hldef}[1]{\textcolor[rgb]{0.345,0.345,0.345}{#1}}%
\newcommand{\hlkwa}[1]{\textcolor[rgb]{0.161,0.373,0.58}{\textbf{#1}}}%
\newcommand{\hlkwb}[1]{\textcolor[rgb]{0.69,0.353,0.396}{#1}}%
\newcommand{\hlkwc}[1]{\textcolor[rgb]{0.333,0.667,0.333}{#1}}%
\newcommand{\hlkwd}[1]{\textcolor[rgb]{0.737,0.353,0.396}{\textbf{#1}}}%
\let\hlipl\hlkwb

\usepackage{framed}
\makeatletter
\newenvironment{kframe}{%
 \def\at@end@of@kframe{}%
 \ifinner\ifhmode%
  \def\at@end@of@kframe{\end{minipage}}%
  \begin{minipage}{\columnwidth}%
 \fi\fi%
 \def\FrameCommand##1{\hskip\@totalleftmargin \hskip-\fboxsep
 \colorbox{shadecolor}{##1}\hskip-\fboxsep
     % There is no \\@totalrightmargin, so:
     \hskip-\linewidth \hskip-\@totalleftmargin \hskip\columnwidth}%
 \MakeFramed {\advance\hsize-\width
   \@totalleftmargin\z@ \linewidth\hsize
   \@setminipage}}%
 {\par\unskip\endMakeFramed%
 \at@end@of@kframe}
\makeatother

\definecolor{shadecolor}{rgb}{.97, .97, .97}
\definecolor{messagecolor}{rgb}{0, 0, 0}
\definecolor{warningcolor}{rgb}{1, 0, 1}
\definecolor{errorcolor}{rgb}{1, 0, 0}
\newenvironment{knitrout}{}{} % an empty environment to be redefined in TeX

\usepackage{alltt}

\usepackage[top=1 in, bottom = 1 in, left = 1 in, right = 1 in ]{geometry}

\usepackage{amsmath, amssymb, amsfonts}
\usepackage{enumerate}
\usepackage{dingbat}
\usepackage{fontawesome5}

\title{MSMS - 106}
\author{Ananda Biswas}
\date{}
\IfFileExists{upquote.sty}{\usepackage{upquote}}{}
\begin{document}

\maketitle

\smallpencil \hspace{0.25cm} \textbf{Consider 1, 2, 3, \ldots, 100 as population observations. Take a sample of size 40 (without replacement) by \textit{sample()} function and compute mean, median, variance, mean absolute deviation about mean, median absolute deviation about median, skewness and kurtosis. Implement the functions on your own and use functions provided by R to verify results.}

\vspace{0.5cm}

\faArrowAltCircleRight[regular] \textit{\textbf{Getting a sample}}

\begin{knitrout}\footnotesize
\definecolor{shadecolor}{rgb}{0.969, 0.969, 0.969}\color{fgcolor}\begin{kframe}
\begin{alltt}
\hldef{my_sample} \hlkwb{<-} \hlkwd{sample}\hldef{(}\hlnum{1}\hlopt{:}\hlnum{100}\hldef{,} \hlkwc{size} \hldef{=} \hlnum{40}\hldef{,} \hlkwc{replace} \hldef{=} \hlnum{FALSE}\hldef{)}
\hldef{my_sample}
\end{alltt}
\begin{verbatim}
##  [1]  87  74  23  10 100  35  51  75  60  36  22   1  83   6  90  14  67  13  32
## [20]  98  31  44  55  28  26   5  17  40  57  52  54  93  15  42  86  33  12  95
## [39]  29  68
\end{verbatim}
\end{kframe}
\end{knitrout}

\faArrowAltCircleRight[regular] \textit{\textbf{Mean}} \\

To compute mean, we add all the observations and divide the sum by number of observations i.e. $\bar{x} = \dfrac{1}{n} \sum \limits_{i = 1}^{n} x_{i}$.

\begin{knitrout}\footnotesize
\definecolor{shadecolor}{rgb}{0.969, 0.969, 0.969}\color{fgcolor}\begin{kframe}
\begin{alltt}
\hldef{my_mean_function} \hlkwb{<-} \hlkwa{function}\hldef{(}\hlkwc{x}\hldef{)\{}

  \hldef{sample_sum} \hlkwb{<-} \hlnum{0}

  \hlkwa{for} \hldef{(i} \hlkwa{in} \hlnum{1}\hlopt{:}\hlkwd{length}\hldef{(x)) \{}
    \hldef{sample_sum} \hlkwb{<-} \hldef{sample_sum} \hlopt{+} \hldef{x[i]}
  \hldef{\}}

  \hlkwd{return}\hldef{(sample_sum} \hlopt{/} \hlkwd{length}\hldef{(x))}
\hldef{\}}
\end{alltt}
\end{kframe}
\end{knitrout}

\begin{knitrout}\footnotesize
\definecolor{shadecolor}{rgb}{0.969, 0.969, 0.969}\color{fgcolor}\begin{kframe}
\begin{alltt}
\hlkwd{my_mean_function}\hldef{(my_sample);} \hlkwd{mean}\hldef{(my_sample)}
\end{alltt}
\begin{verbatim}
## [1] 46.475
## [1] 46.475
\end{verbatim}
\end{kframe}
\end{knitrout}

\faCheckCircle[regular] Results matched !! \\

\newpage

\faArrowAltCircleRight[regular] \textit{\textbf{Median}} \\

To compute median, we first sort the sample values and return the middle most value.
\begin{knitrout}\footnotesize
\definecolor{shadecolor}{rgb}{0.969, 0.969, 0.969}\color{fgcolor}\begin{kframe}
\begin{alltt}
\hldef{my_median_function} \hlkwb{<-} \hlkwa{function}\hldef{(}\hlkwc{x}\hldef{)\{}

  \hlkwa{for} \hldef{(i} \hlkwa{in} \hlnum{1}\hlopt{:}\hldef{(}\hlkwd{length}\hldef{(x)}\hlopt{-}\hlnum{1}\hldef{)) \{}
    \hlkwa{for} \hldef{(j} \hlkwa{in} \hldef{(i}\hlopt{+}\hlnum{1}\hldef{)}\hlopt{:}\hlkwd{length}\hldef{(x)) \{}

      \hlkwa{if}\hldef{(x[i]} \hlopt{>} \hldef{x[j])\{}
        \hldef{x[}\hlkwd{c}\hldef{(i, j)]} \hlkwb{<-} \hldef{x[}\hlkwd{c}\hldef{(j, i)]}
      \hldef{\}}
    \hldef{\}}
  \hldef{\}}

  \hlkwa{if}\hldef{(}\hlkwd{length}\hldef{(x)} \hlopt \hlnum{2} \hlopt{==} \hlnum{0}\hldef{)\{}
    \hlkwd{return}\hldef{((x[}\hlkwd{length}\hldef{(x)}\hlopt{/}\hlnum{2}\hldef{]} \hlopt{+} \hldef{x[}\hlkwd{length}\hldef{(x)}\hlopt{/}\hlnum{2} \hlopt{+} \hlnum{1}\hldef{])} \hlopt{/} \hlnum{2}\hldef{)}
  \hldef{\}}
  \hlkwa{else}\hldef{\{}
    \hlkwd{return}\hldef{(x[(}\hlkwd{length}\hldef{(x)} \hlopt{+} \hlnum{1}\hldef{)} \hlopt{/} \hlnum{2}\hldef{])}
  \hldef{\}}
\hldef{\}}
\end{alltt}
\end{kframe}
\end{knitrout}

\begin{knitrout}\footnotesize
\definecolor{shadecolor}{rgb}{0.969, 0.969, 0.969}\color{fgcolor}\begin{kframe}
\begin{alltt}
\hlkwd{my_median_function}\hldef{(my_sample);} \hlkwd{median}\hldef{(my_sample)}
\end{alltt}
\begin{verbatim}
## [1] 41
## [1] 41
\end{verbatim}
\end{kframe}
\end{knitrout}

\faCheckCircle[regular] Results matched !! \\

\vspace{1cm}

\faArrowAltCircleRight[regular] \textit{\textbf{Variance}} \\

To compute variance, we get the squared deviations, their sum and then we divide the sum by $(n-1)$ i.e. $\dfrac{1}{n-1} \sum \limits_{i = 1}^{n} (x_{i} - \bar{x})^2$.
\begin{knitrout}\footnotesize
\definecolor{shadecolor}{rgb}{0.969, 0.969, 0.969}\color{fgcolor}\begin{kframe}
\begin{alltt}
\hldef{my_sample_central_moments_function} \hlkwb{<-} \hlkwa{function}\hldef{(}\hlkwc{x}\hldef{,} \hlkwc{r}\hldef{)\{}
  \hldef{temp} \hlkwb{<-} \hlnum{0}

  \hlkwa{for} \hldef{(i} \hlkwa{in} \hlnum{1}\hlopt{:}\hlkwd{length}\hldef{(x)) \{}
    \hldef{temp} \hlkwb{<-} \hldef{temp} \hlopt{+} \hldef{(x[i]} \hlopt{-} \hlkwd{my_mean_function}\hldef{(x))}\hlopt{^}\hldef{r}
  \hldef{\}}

  \hlkwd{return}\hldef{(temp} \hlopt{/} \hldef{(}\hlkwd{length}\hldef{(x)} \hlopt{-} \hlnum{1}\hldef{))}
\hldef{\}}
\end{alltt}
\end{kframe}
\end{knitrout}

\begin{knitrout}\footnotesize
\definecolor{shadecolor}{rgb}{0.969, 0.969, 0.969}\color{fgcolor}\begin{kframe}
\begin{alltt}
\hlkwd{my_sample_central_moments_function}\hldef{(my_sample,} \hlnum{2}\hldef{);} \hlkwd{var}\hldef{(my_sample)}
\end{alltt}
\begin{verbatim}
## [1] 885.5891
## [1] 885.5891
\end{verbatim}
\end{kframe}
\end{knitrout}

\faCheckCircle[regular] Results matched !! \\

\newpage

\faArrowAltCircleRight[regular] \textit{\textbf{Mean Absolute Deviation about Mean}} \\

We first obtain the deviations $(x_i - \bar{x}) \,\, \forall i$ and then we calculate their absolute values. Lastly we calculate their mean.

\begin{knitrout}\footnotesize
\definecolor{shadecolor}{rgb}{0.969, 0.969, 0.969}\color{fgcolor}\begin{kframe}
\begin{alltt}
\hldef{deviations_about_mean} \hlkwb{<-} \hldef{my_sample} \hlopt{-} \hlkwd{my_mean_function}\hldef{(my_sample)}

\hldef{absolute_deviations_about_mean} \hlkwb{<-} \hlkwd{c}\hldef{()}

\hlkwa{for} \hldef{(i} \hlkwa{in} \hlnum{1}\hlopt{:}\hlkwd{length}\hldef{(deviations_about_mean)) \{}
  \hlkwa{if}\hldef{(deviations_about_mean[i]} \hlopt{<} \hlnum{0}\hldef{)\{}
    \hldef{absolute_deviations_about_mean[i]} \hlkwb{<-} \hldef{deviations_about_mean[i]} \hlopt{*} \hldef{(}\hlopt{-}\hlnum{1}\hldef{)}
  \hldef{\}}
  \hlkwa{else}\hldef{\{}
    \hldef{absolute_deviations_about_mean[i]} \hlkwb{<-} \hldef{deviations_about_mean[i]}
  \hldef{\}}
\hldef{\}}
\end{alltt}
\end{kframe}
\end{knitrout}

\begin{knitrout}\footnotesize
\definecolor{shadecolor}{rgb}{0.969, 0.969, 0.969}\color{fgcolor}\begin{kframe}
\begin{alltt}
\hldef{absolute_deviations_about_mean}
\end{alltt}
\begin{verbatim}
##  [1] 40.525 27.525 23.475 36.475 53.525 11.475  4.525 28.525 13.525 10.475
## [11] 24.475 45.475 36.525 40.475 43.525 32.475 20.525 33.475 14.475 51.525
## [21] 15.475  2.475  8.525 18.475 20.475 41.475 29.475  6.475 10.525  5.525
## [31]  7.525 46.525 31.475  4.475 39.525 13.475 34.475 48.525 17.475 21.525
\end{verbatim}
\end{kframe}
\end{knitrout}

\begin{knitrout}\footnotesize
\definecolor{shadecolor}{rgb}{0.969, 0.969, 0.969}\color{fgcolor}\begin{kframe}
\begin{alltt}
\hlkwd{my_mean_function}\hldef{(absolute_deviations_about_mean)}
\end{alltt}
\begin{verbatim}
## [1] 25.4225
\end{verbatim}
\end{kframe}
\end{knitrout}

\vspace{0.5cm}

\faArrowAltCircleRight[regular] \textit{\textbf{Median Absolute Deviation about Median}} \\

We first obtain the deviations $(x_i - \widetilde{x}) \,\, \forall i$ and then we calculate their absolute values. Lastly we calculate their median.

\begin{knitrout}\footnotesize
\definecolor{shadecolor}{rgb}{0.969, 0.969, 0.969}\color{fgcolor}\begin{kframe}
\begin{alltt}
\hldef{deviations_about_median} \hlkwb{<-} \hldef{my_sample} \hlopt{-} \hlkwd{my_median_function}\hldef{(my_sample)}

\hldef{absolute_deviations_about_median} \hlkwb{<-} \hlkwd{c}\hldef{()}

\hlkwa{for} \hldef{(i} \hlkwa{in} \hlnum{1}\hlopt{:}\hlkwd{length}\hldef{(deviations_about_median)) \{}
  \hlkwa{if}\hldef{(deviations_about_median[i]} \hlopt{<} \hlnum{0}\hldef{)\{}
    \hldef{absolute_deviations_about_median[i]} \hlkwb{<-} \hldef{deviations_about_median[i]} \hlopt{*} \hldef{(}\hlopt{-}\hlnum{1}\hldef{)}
  \hldef{\}}
  \hlkwa{else}\hldef{\{}
    \hldef{absolute_deviations_about_median[i]} \hlkwb{<-} \hldef{deviations_about_median[i]}
  \hldef{\}}
\hldef{\}}
\end{alltt}
\end{kframe}
\end{knitrout}

\begin{knitrout}\footnotesize
\definecolor{shadecolor}{rgb}{0.969, 0.969, 0.969}\color{fgcolor}\begin{kframe}
\begin{alltt}
\hldef{absolute_deviations_about_median}
\end{alltt}
\begin{verbatim}
##  [1] 46 33 18 31 59  6 10 34 19  5 19 40 42 35 49 27 26 28  9 57 10  3 14 13 15
## [26] 36 24  1 16 11 13 52 26  1 45  8 29 54 12 27
\end{verbatim}
\end{kframe}
\end{knitrout}


\begin{knitrout}\footnotesize
\definecolor{shadecolor}{rgb}{0.969, 0.969, 0.969}\color{fgcolor}\begin{kframe}
\begin{alltt}
\hlkwd{my_median_function}\hldef{(absolute_deviations_about_median)}
\end{alltt}
\begin{verbatim}
## [1] 25
\end{verbatim}
\end{kframe}
\end{knitrout}

\newpage

\faArrowAltCircleRight[regular] \textit{\textbf{Skewness}} \\

We compute $\dfrac{m_3}{m_2^\frac{3}{2}}$ where $m_2$ and $m_3$ are 2nd and 3rd order central moments respectively.
\begin{knitrout}\footnotesize
\definecolor{shadecolor}{rgb}{0.969, 0.969, 0.969}\color{fgcolor}\begin{kframe}
\begin{alltt}
\hldef{my_skewness_function} \hlkwb{<-} \hlkwa{function}\hldef{(}\hlkwc{x}\hldef{)\{}
  \hlkwd{return}\hldef{(}\hlkwd{my_sample_central_moments_function}\hldef{(x,} \hlnum{3}\hldef{)} \hlopt{/} \hlkwd{my_sample_central_moments_function}\hldef{(x,} \hlnum{2}\hldef{)}\hlopt{^}\hldef{(}\hlnum{3}\hlopt{/}\hlnum{2}\hldef{))}
\hldef{\}}
\end{alltt}
\end{kframe}
\end{knitrout}

\begin{knitrout}\footnotesize
\definecolor{shadecolor}{rgb}{0.969, 0.969, 0.969}\color{fgcolor}\begin{kframe}
\begin{alltt}
\hlkwd{my_skewness_function}\hldef{(my_sample)}
\end{alltt}
\begin{verbatim}
## [1] 0.3137019
\end{verbatim}
\end{kframe}
\end{knitrout}


\vspace{2cm}

\faArrowAltCircleRight[regular] \textit{\textbf{Kurtosis}} \\

We compute $\dfrac{m_4}{m_2^2} - 3$ where $m_4$ is the 4th order central moment.
\begin{knitrout}\footnotesize
\definecolor{shadecolor}{rgb}{0.969, 0.969, 0.969}\color{fgcolor}\begin{kframe}
\begin{alltt}
\hldef{my_kurtosis_function} \hlkwb{<-} \hlkwa{function}\hldef{(}\hlkwc{x}\hldef{)\{}
  \hlkwd{return}\hldef{(}\hlkwd{my_sample_central_moments_function}\hldef{(x,} \hlnum{4}\hldef{)} \hlopt{/} \hlkwd{my_sample_central_moments_function}\hldef{(x,} \hlnum{2}\hldef{)}\hlopt{^}\hlnum{2} \hlopt{-} \hlnum{3}\hldef{)}
\hldef{\}}
\end{alltt}
\end{kframe}
\end{knitrout}

\begin{knitrout}\footnotesize
\definecolor{shadecolor}{rgb}{0.969, 0.969, 0.969}\color{fgcolor}\begin{kframe}
\begin{alltt}
\hlkwd{my_kurtosis_function}\hldef{(my_sample)}
\end{alltt}
\begin{verbatim}
## [1] -1.165996
\end{verbatim}
\end{kframe}
\end{knitrout}



\end{document}
