\documentclass[11pt, a4paper]{article}\usepackage[]{graphicx}\usepackage[]{xcolor}
% maxwidth is the original width if it is less than linewidth
% otherwise use linewidth (to make sure the graphics do not exceed the margin)
\makeatletter
\def\maxwidth{ %
  \ifdim\Gin@nat@width>\linewidth
    \linewidth
  \else
    \Gin@nat@width
  \fi
}
\makeatother

\definecolor{fgcolor}{rgb}{0.345, 0.345, 0.345}
\newcommand{\hlnum}[1]{\textcolor[rgb]{0.686,0.059,0.569}{#1}}%
\newcommand{\hlsng}[1]{\textcolor[rgb]{0.192,0.494,0.8}{#1}}%
\newcommand{\hlcom}[1]{\textcolor[rgb]{0.678,0.584,0.686}{\textit{#1}}}%
\newcommand{\hlopt}[1]{\textcolor[rgb]{0,0,0}{#1}}%
\newcommand{\hldef}[1]{\textcolor[rgb]{0.345,0.345,0.345}{#1}}%
\newcommand{\hlkwa}[1]{\textcolor[rgb]{0.161,0.373,0.58}{\textbf{#1}}}%
\newcommand{\hlkwb}[1]{\textcolor[rgb]{0.69,0.353,0.396}{#1}}%
\newcommand{\hlkwc}[1]{\textcolor[rgb]{0.333,0.667,0.333}{#1}}%
\newcommand{\hlkwd}[1]{\textcolor[rgb]{0.737,0.353,0.396}{\textbf{#1}}}%
\let\hlipl\hlkwb

\usepackage{framed}
\makeatletter
\newenvironment{kframe}{%
 \def\at@end@of@kframe{}%
 \ifinner\ifhmode%
  \def\at@end@of@kframe{\end{minipage}}%
  \begin{minipage}{\columnwidth}%
 \fi\fi%
 \def\FrameCommand##1{\hskip\@totalleftmargin \hskip-\fboxsep
 \colorbox{shadecolor}{##1}\hskip-\fboxsep
     % There is no \\@totalrightmargin, so:
     \hskip-\linewidth \hskip-\@totalleftmargin \hskip\columnwidth}%
 \MakeFramed {\advance\hsize-\width
   \@totalleftmargin\z@ \linewidth\hsize
   \@setminipage}}%
 {\par\unskip\endMakeFramed%
 \at@end@of@kframe}
\makeatother

\definecolor{shadecolor}{rgb}{.97, .97, .97}
\definecolor{messagecolor}{rgb}{0, 0, 0}
\definecolor{warningcolor}{rgb}{1, 0, 1}
\definecolor{errorcolor}{rgb}{1, 0, 0}
\newenvironment{knitrout}{}{} % an empty environment to be redefined in TeX

\usepackage{alltt}

\usepackage[top=1 in, bottom = 1 in, left = 1 in, right = 1 in ]{geometry}

\usepackage{amsmath, amssymb, amsfonts}
\usepackage{enumerate}
\usepackage{dingbat}
\usepackage{fontawesome5}
\usepackage{twemojis}
\usepackage{tasks}

\title{MSMS 106}
\author{Ananda Biswas}
\date{December 5, 2024}
\IfFileExists{upquote.sty}{\usepackage{upquote}}{}
\begin{document}

\maketitle

\begin{center}
\textbf{Practical 07}
\end{center}

\scalebox{2.0}{\twemoji{bullseye}} \hspace{0.2cm} \textcolor{blue}{\textbf{Gauss-Legendre Integration Methods}}

\vspace{0.5cm}

\smallpencil \hspace{0.2cm} \textcolor{blue}{\textbf{Question 1 : }} \\

\hspace{1cm} Write an R program to approximate the integral of $f(x)$ over the finite interval $[a, b]$ using Gauss-Legendre one point formula. \\

\faArrowAltCircleRight[regular] \hspace{0.2cm} Gauss-Legendre one point formula is given by $$\int \limits_{-1}^{1} f(x) \,\, dx \approx 2 \cdot f(0). $$

We transform the given interval $\left[ a, b \right]$ to $\left[ -1, 1 \right]$ by the transformation $x = \dfrac{b - a}{2} \cdot t + \dfrac{b + a}{2}$.

\begin{knitrout}
\definecolor{shadecolor}{rgb}{0.969, 0.969, 0.969}\color{fgcolor}\begin{kframe}
\begin{alltt}
\hldef{transformation} \hlkwb{<-} \hlkwa{function}\hldef{(}\hlkwc{t}\hldef{,} \hlkwc{a}\hldef{,} \hlkwc{b}\hldef{)\{}
  \hlkwd{return}\hldef{( ( (b} \hlopt{-} \hldef{a)} \hlopt{/} \hlnum{2} \hldef{)} \hlopt{*} \hldef{t} \hlopt{+} \hldef{(b} \hlopt{+} \hldef{a)} \hlopt{/} \hlnum{2}\hldef{)}
\hldef{\}}
\end{alltt}
\end{kframe}
\end{knitrout}


\begin{knitrout}
\definecolor{shadecolor}{rgb}{0.969, 0.969, 0.969}\color{fgcolor}\begin{kframe}
\begin{alltt}
\hldef{one_point} \hlkwb{<-} \hlkwa{function}\hldef{(}\hlkwc{func}\hldef{,} \hlkwc{a}\hldef{,} \hlkwc{b}\hldef{)\{}

  \hldef{integration} \hlkwb{<-} \hlnum{2} \hlopt{*} \hlkwd{func}\hldef{(}\hlkwd{transformation}\hldef{(}\hlnum{0}\hldef{, a, b))}

  \hldef{result} \hlkwb{<-} \hldef{( (b} \hlopt{-} \hldef{a)} \hlopt{/} \hlnum{2} \hldef{)} \hlopt{*} \hldef{integration}

  \hlkwd{return}\hldef{(result)}
\hldef{\}}
\end{alltt}
\end{kframe}
\end{knitrout}

\smallpencil \hspace{0.2cm} \textcolor{blue}{\textbf{Question 2 : }} \\

\hspace{1cm} Write an R program to approximate the integral of $f(x)$ over the finite interval $[a, b]$ using Gauss-Legendre two point formula. \\

\faArrowAltCircleRight[regular] \hspace{0.2cm} Gauss-Legendre two point formula is given by $$\int \limits_{-1}^{1} f(x) \,\, dx \approx f\left(-\dfrac{1}{\sqrt{3}}\right) + f\left(\dfrac{1}{\sqrt{3}}\right). $$

We transform the given interval $\left[ a, b \right]$ to $\left[ -1, 1 \right]$ by the transformation $x = \dfrac{b - a}{2} \cdot t + \dfrac{b + a}{2}$.

\begin{knitrout}
\definecolor{shadecolor}{rgb}{0.969, 0.969, 0.969}\color{fgcolor}\begin{kframe}
\begin{alltt}
\hldef{two_point} \hlkwb{<-} \hlkwa{function}\hldef{(}\hlkwc{func}\hldef{,} \hlkwc{a}\hldef{,} \hlkwc{b}\hldef{)\{}

  \hldef{integration} \hlkwb{<-} \hlkwd{func}\hldef{(}\hlkwd{transformation}\hldef{(}\hlopt{-}\hldef{(}\hlnum{1} \hlopt{/} \hlkwd{sqrt}\hldef{(}\hlnum{3}\hldef{)), a, b))} \hlopt{+}
                 \hlkwd{func}\hldef{(}\hlkwd{transformation}\hldef{(}\hlnum{1} \hlopt{/} \hlkwd{sqrt}\hldef{(}\hlnum{3}\hldef{), a, b))}

  \hldef{result} \hlkwb{<-} \hldef{( (b} \hlopt{-} \hldef{a)} \hlopt{/} \hlnum{2} \hldef{)} \hlopt{*} \hldef{integration}

  \hlkwd{return}\hldef{(result)}
\hldef{\}}
\end{alltt}
\end{kframe}
\end{knitrout}

\smallpencil \hspace{0.2cm} \textcolor{blue}{\textbf{Question 3 : }} \\

\hspace{1cm} Write an R program to approximate the integral of $f(x)$ over the finite interval $[a, b]$ using Gauss-Legendre three point formula. \\

\faArrowAltCircleRight[regular] \hspace{0.2cm} Gauss-Legendre three point formula is given by $$\int \limits_{-1}^{1} f(x) \,\, dx \approx \dfrac{1}{9} \left[ 5 f\left( - \sqrt{\dfrac{3}{5}}\right) + 8 f(0) + 5 f\left(  \sqrt{\dfrac{3}{5}}\right) \right]. $$

We transform the given interval $\left[ a, b \right]$ to $\left[ -1, 1 \right]$ by the transformation $x = \dfrac{b - a}{2} \cdot t + \dfrac{b + a}{2}$.

\begin{knitrout}
\definecolor{shadecolor}{rgb}{0.969, 0.969, 0.969}\color{fgcolor}\begin{kframe}
\begin{alltt}
\hldef{three_point} \hlkwb{<-} \hlkwa{function}\hldef{(}\hlkwc{func}\hldef{,} \hlkwc{a}\hldef{,} \hlkwc{b}\hldef{)\{}

  \hldef{temp} \hlkwb{<-} \hlnum{5} \hlopt{*} \hlkwd{func}\hldef{(}\hlkwd{transformation}\hldef{(}\hlopt{-}\hlkwd{sqrt}\hldef{(}\hlnum{3}\hlopt{/}\hlnum{5}\hldef{), a, b))} \hlopt{+}
                 \hlnum{8} \hlopt{*} \hlkwd{func}\hldef{(}\hlkwd{transformation}\hldef{(}\hlnum{0}\hldef{, a, b))} \hlopt{+}
                 \hlnum{5} \hlopt{*} \hlkwd{func}\hldef{(}\hlkwd{transformation}\hldef{(}\hlkwd{sqrt}\hldef{(}\hlnum{3}\hlopt{/}\hlnum{5}\hldef{), a, b))}

  \hldef{integration} \hlkwb{<-} \hldef{temp} \hlopt{/} \hlnum{9}

  \hldef{result} \hlkwb{<-} \hldef{( (b} \hlopt{-} \hldef{a)} \hlopt{/} \hlnum{2} \hldef{)} \hlopt{*} \hldef{integration}

  \hlkwd{return}\hldef{(result)}
\hldef{\}}
\end{alltt}
\end{kframe}
\end{knitrout}

\leftpointright \hspace{0.5cm} Evaluate the integral $\displaystyle{\int \limits_{1}^{2} \dfrac{2x}{1 + x^4} \,\, dx}$ using the Gauss-Legendre 1-point, 2-point and 3-point quadrature rules.

\begin{knitrout}
\definecolor{shadecolor}{rgb}{0.969, 0.969, 0.969}\color{fgcolor}\begin{kframe}
\begin{alltt}
\hldef{f1} \hlkwb{<-} \hlkwa{function}\hldef{(}\hlkwc{x}\hldef{)} \hlnum{2}\hlopt{*}\hldef{x} \hlopt{/} \hldef{(}\hlnum{1} \hlopt{+} \hldef{x}\hlopt{^}\hlnum{4}\hldef{)}

\hlkwd{one_point}\hldef{(f1,} \hlnum{1}\hldef{,} \hlnum{2}\hldef{)}
\end{alltt}
\begin{verbatim}
## [1] 0.4948454
\end{verbatim}
\end{kframe}
\end{knitrout}

\begin{knitrout}
\definecolor{shadecolor}{rgb}{0.969, 0.969, 0.969}\color{fgcolor}\begin{kframe}
\begin{alltt}
\hlkwd{two_point}\hldef{(f1,} \hlnum{1}\hldef{,} \hlnum{2}\hldef{)}
\end{alltt}
\begin{verbatim}
## [1] 0.5433755
\end{verbatim}
\end{kframe}
\end{knitrout}

\begin{knitrout}
\definecolor{shadecolor}{rgb}{0.969, 0.969, 0.969}\color{fgcolor}\begin{kframe}
\begin{alltt}
\hlkwd{three_point}\hldef{(f1,} \hlnum{1}\hldef{,} \hlnum{2}\hldef{)}
\end{alltt}
\begin{verbatim}
## [1] 0.5405911
\end{verbatim}
\end{kframe}
\end{knitrout}

\leftpointright \hspace{0.5cm} Evaluate the integrals
\settasks{
	label=(\roman*),
	label-width=4ex
}
	\begin{tasks}(2)
		\task $I = \displaystyle{\int \limits_{0}^{2} \dfrac{1}{3 + 4x} \,\, dx}$
		\task $I = \displaystyle{\int \limits_{0}^{2} \dfrac{1}{x^2 + 2x + 10} \,\, dx}$
	\end{tasks}

\begin{enumerate}[(a)]

\item by Gauss-Legendre two-point and three-point formulas;

\item Write $I$ as $I_1 + I_2$ where $I_1 = \displaystyle{\int \limits_{0}^{1} f(x) \,\, dx}$ and $I_2 = \displaystyle{\int \limits_{1}^{2} f(x) \,\, dx}$. Then evaluate each of the integrals by Gauss-Legendre two-point and three-point formulas.
\end{enumerate}

\begin{knitrout}
\definecolor{shadecolor}{rgb}{0.969, 0.969, 0.969}\color{fgcolor}\begin{kframe}
\begin{alltt}
\hldef{f2} \hlkwb{<-} \hlkwa{function}\hldef{(}\hlkwc{x}\hldef{)} \hlnum{1} \hlopt{/} \hldef{(}\hlnum{3} \hlopt{+} \hlnum{4}\hlopt{*}\hldef{x)}

\hlkwd{two_point}\hldef{(f2,} \hlnum{0}\hldef{,} \hlnum{2}\hldef{)}
\end{alltt}
\begin{verbatim}
## [1] 0.3206107
\end{verbatim}
\end{kframe}
\end{knitrout}

\begin{knitrout}
\definecolor{shadecolor}{rgb}{0.969, 0.969, 0.969}\color{fgcolor}\begin{kframe}
\begin{alltt}
\hlkwd{two_point}\hldef{(f2,} \hlnum{0}\hldef{,} \hlnum{1}\hldef{)} \hlopt{+} \hlkwd{two_point}\hldef{(f2,} \hlnum{1}\hldef{,} \hlnum{2}\hldef{)}
\end{alltt}
\begin{verbatim}
## [1] 0.3242383
\end{verbatim}
\end{kframe}
\end{knitrout}

\begin{knitrout}
\definecolor{shadecolor}{rgb}{0.969, 0.969, 0.969}\color{fgcolor}\begin{kframe}
\begin{alltt}
\hlkwd{three_point}\hldef{(f2,} \hlnum{0}\hldef{,} \hlnum{2}\hldef{)}
\end{alltt}
\begin{verbatim}
## [1] 0.3243897
\end{verbatim}
\end{kframe}
\end{knitrout}

\begin{knitrout}
\definecolor{shadecolor}{rgb}{0.969, 0.969, 0.969}\color{fgcolor}\begin{kframe}
\begin{alltt}
\hlkwd{three_point}\hldef{(f2,} \hlnum{0}\hldef{,} \hlnum{1}\hldef{)} \hlopt{+} \hlkwd{three_point}\hldef{(f2,} \hlnum{1}\hldef{,} \hlnum{2}\hldef{)}
\end{alltt}
\begin{verbatim}
## [1] 0.3247954
\end{verbatim}
\end{kframe}
\end{knitrout}

\begin{knitrout}
\definecolor{shadecolor}{rgb}{0.969, 0.969, 0.969}\color{fgcolor}\begin{kframe}
\begin{alltt}
\hldef{f3} \hlkwb{<-} \hlkwa{function}\hldef{(}\hlkwc{x}\hldef{)} \hlnum{1} \hlopt{/} \hldef{(x}\hlopt{^}\hlnum{2} \hlopt{+} \hlnum{2}\hlopt{*}\hldef{x} \hlopt{+} \hlnum{10}\hldef{)}

\hlkwd{two_point}\hldef{(f3,} \hlnum{0}\hldef{,} \hlnum{2}\hldef{)}
\end{alltt}
\begin{verbatim}
## [1] 0.1546392
\end{verbatim}
\end{kframe}
\end{knitrout}

\begin{knitrout}
\definecolor{shadecolor}{rgb}{0.969, 0.969, 0.969}\color{fgcolor}\begin{kframe}
\begin{alltt}
\hlkwd{two_point}\hldef{(f3,} \hlnum{0}\hldef{,} \hlnum{1}\hldef{)} \hlopt{+} \hlkwd{two_point}\hldef{(f3,} \hlnum{1}\hldef{,} \hlnum{2}\hldef{)}
\end{alltt}
\begin{verbatim}
## [1] 0.154554
\end{verbatim}
\end{kframe}
\end{knitrout}

\begin{knitrout}
\definecolor{shadecolor}{rgb}{0.969, 0.969, 0.969}\color{fgcolor}\begin{kframe}
\begin{alltt}
\hlkwd{three_point}\hldef{(f3,} \hlnum{0}\hldef{,} \hlnum{2}\hldef{)}
\end{alltt}
\begin{verbatim}
## [1] 0.154548
\end{verbatim}
\end{kframe}
\end{knitrout}

\begin{knitrout}
\definecolor{shadecolor}{rgb}{0.969, 0.969, 0.969}\color{fgcolor}\begin{kframe}
\begin{alltt}
\hlkwd{three_point}\hldef{(f3,} \hlnum{0}\hldef{,} \hlnum{1}\hldef{)} \hlopt{+} \hlkwd{three_point}\hldef{(f3,} \hlnum{1}\hldef{,} \hlnum{2}\hldef{)}
\end{alltt}
\begin{verbatim}
## [1] 0.1545492
\end{verbatim}
\end{kframe}
\end{knitrout}

\leftpointright \hspace{0.5cm} Evaluate the integral $$I = \int \limits_{2}^{3} \dfrac{\cos 2x}{1 + \sin x} \,\, dx$$ by Gauss-Legendre two-point and three-point integration rules.

\begin{knitrout}
\definecolor{shadecolor}{rgb}{0.969, 0.969, 0.969}\color{fgcolor}\begin{kframe}
\begin{alltt}
\hldef{f4} \hlkwb{<-} \hlkwa{function}\hldef{(}\hlkwc{x}\hldef{)} \hlkwd{cos}\hldef{(}\hlnum{2}\hlopt{*}\hldef{x)} \hlopt{/} \hldef{(}\hlnum{1} \hlopt{+} \hlkwd{sin}\hldef{(x))}

\hlkwd{two_point}\hldef{(f4,} \hlnum{2}\hldef{,} \hlnum{3}\hldef{)}
\end{alltt}
\begin{verbatim}
## [1] 0.2035084
\end{verbatim}
\end{kframe}
\end{knitrout}

\begin{knitrout}
\definecolor{shadecolor}{rgb}{0.969, 0.969, 0.969}\color{fgcolor}\begin{kframe}
\begin{alltt}
\hlkwd{three_point}\hldef{(f4,} \hlnum{2}\hldef{,} \hlnum{3}\hldef{)}
\end{alltt}
\begin{verbatim}
## [1] 0.2027139
\end{verbatim}
\end{kframe}
\end{knitrout}

\leftpointright \hspace{0.5cm} Obtain an approximate value of $$I = \int \limits_{-1}^{1} \sqrt{1 - x^2} \,\, \cos x \,\, dx$$ by Gauss-Legendre three-point formula.

\begin{knitrout}
\definecolor{shadecolor}{rgb}{0.969, 0.969, 0.969}\color{fgcolor}\begin{kframe}
\begin{alltt}
\hldef{f5} \hlkwb{<-} \hlkwa{function}\hldef{(}\hlkwc{x}\hldef{)} \hlkwd{sqrt}\hldef{(}\hlnum{1} \hlopt{-} \hldef{x}\hlopt{^}\hlnum{2}\hldef{)} \hlopt{*} \hlkwd{cos}\hldef{(x)}

\hlkwd{three_point}\hldef{(f5,} \hlopt{-}\hlnum{1}\hldef{,} \hlnum{1}\hldef{)}
\end{alltt}
\begin{verbatim}
## [1] 1.391131
\end{verbatim}
\end{kframe}
\end{knitrout}


\leftpointright \hspace{0.5cm} Evaluate the integral $$I = \int \limits_{0}^{1} \dfrac{1}{1 + x} \,\, dx$$ by using Gauss-Legendre three-point formula.

\begin{knitrout}
\definecolor{shadecolor}{rgb}{0.969, 0.969, 0.969}\color{fgcolor}\begin{kframe}
\begin{alltt}
\hldef{f6} \hlkwb{<-} \hlkwa{function}\hldef{(}\hlkwc{x}\hldef{)} \hlnum{1} \hlopt{/} \hldef{(}\hlnum{1} \hlopt{+} \hldef{x)}

\hlkwd{two_point}\hldef{(f6,} \hlnum{0}\hldef{,} \hlnum{1}\hldef{)}
\end{alltt}
\begin{verbatim}
## [1] 0.6923077
\end{verbatim}
\end{kframe}
\end{knitrout}

\begin{knitrout}
\definecolor{shadecolor}{rgb}{0.969, 0.969, 0.969}\color{fgcolor}\begin{kframe}
\begin{alltt}
\hlkwd{three_point}\hldef{(f6,} \hlnum{0}\hldef{,} \hlnum{1}\hldef{)}
\end{alltt}
\begin{verbatim}
## [1] 0.6931217
\end{verbatim}
\end{kframe}
\end{knitrout}

\leftpointright \hspace{0.5cm} Obtain the approximate value of $$I = \int \limits_{-1}^{1} e^{-x^2} \cos x \,\, dx$$ by Gauss-Legendre three-point formula.

\begin{knitrout}
\definecolor{shadecolor}{rgb}{0.969, 0.969, 0.969}\color{fgcolor}\begin{kframe}
\begin{alltt}
\hldef{f7} \hlkwb{<-} \hlkwa{function}\hldef{(}\hlkwc{x}\hldef{)} \hlkwd{exp}\hldef{(}\hlopt{-}\hldef{x}\hlopt{^}\hlnum{2}\hldef{)} \hlopt{*} \hlkwd{cos}\hldef{(x)}

\hlkwd{three_point}\hldef{(f7,} \hlopt{-}\hlnum{1}\hldef{,} \hlnum{1}\hldef{)}
\end{alltt}
\begin{verbatim}
## [1] 1.324708
\end{verbatim}
\end{kframe}
\end{knitrout}

\leftpointright \hspace{0.5cm} Evaluate the integral $$I = \int \limits_{1}^{2} \dfrac{1}{1 + x^3} \,\, dx$$ by Gauss-Legendre three-point formula.

\begin{knitrout}
\definecolor{shadecolor}{rgb}{0.969, 0.969, 0.969}\color{fgcolor}\begin{kframe}
\begin{alltt}
\hldef{f8} \hlkwb{<-} \hlkwa{function}\hldef{(}\hlkwc{x}\hldef{)} \hlnum{1} \hlopt{/} \hldef{(}\hlnum{1} \hlopt{+} \hldef{x}\hlopt{^}\hlnum{3}\hldef{)}

\hlkwd{three_point}\hldef{(f8,} \hlnum{1}\hldef{,} \hlnum{2}\hldef{)}
\end{alltt}
\begin{verbatim}
## [1] 0.254387
\end{verbatim}
\end{kframe}
\end{knitrout}

\leftpointright \hspace{0.5cm} Evaluate the integral $$I = \int \limits_{0}^{2} \dfrac{x^2 + 2x + 1}{1 + (x+1)^4} \,\, dx$$ by Gauss-Legendre three-point formula.

\begin{knitrout}
\definecolor{shadecolor}{rgb}{0.969, 0.969, 0.969}\color{fgcolor}\begin{kframe}
\begin{alltt}
\hldef{f9} \hlkwb{<-} \hlkwa{function}\hldef{(}\hlkwc{x}\hldef{) (x}\hlopt{^}\hlnum{2} \hlopt{+} \hlnum{2}\hlopt{*}\hldef{x} \hlopt{+} \hlnum{1}\hldef{)} \hlopt{/} \hldef{(}\hlnum{1} \hlopt{+} \hldef{(x} \hlopt{+} \hlnum{1}\hldef{)}\hlopt{^}\hlnum{4}\hldef{)}

\hlkwd{three_point}\hldef{(f9,} \hlnum{0}\hldef{,} \hlnum{2}\hldef{)}
\end{alltt}
\begin{verbatim}
## [1] 0.5364222
\end{verbatim}
\end{kframe}
\end{knitrout}

\leftpointright \hspace{0.5cm} Apply Gauss-Legendre two-point formula to evaluate $\displaystyle{\int \limits_{-1}^{1} \dfrac{1}{1 + x^2} \,\, dx}$.

\begin{knitrout}
\definecolor{shadecolor}{rgb}{0.969, 0.969, 0.969}\color{fgcolor}\begin{kframe}
\begin{alltt}
\hldef{f10} \hlkwb{<-} \hlkwa{function}\hldef{(}\hlkwc{x}\hldef{)} \hlnum{1} \hlopt{/} \hldef{(}\hlnum{1} \hlopt{+} \hldef{x}\hlopt{^}\hlnum{2}\hldef{)}

\hlkwd{two_point}\hldef{(f10,} \hlopt{-}\hlnum{1}\hldef{,} \hlnum{1}\hldef{)}
\end{alltt}
\begin{verbatim}
## [1] 1.5
\end{verbatim}
\end{kframe}
\end{knitrout}

\newpage

\leftpointright \hspace{0.5cm} Use Gauss-Legendre three-point formula to evaluate
\settasks{
	label=(\roman*),
	label-width=4ex
}
	\begin{tasks}(2)
		\task $\displaystyle{\int \limits_{1}^{2} \dfrac{1}{x} \,\, dx}$
		\task $\displaystyle{\int \limits_{0}^{1} \dfrac{1}{1 + x^2} \,\, dx}$
		\task $\displaystyle{\int \limits_{0}^{1} \dfrac{1}{\sqrt{1 + x^4}}\,\,dx}$
		\task $\displaystyle{\int \limits_{0.2}^{1.5} e^{-x^2} \,\, dx}$
	\end{tasks}

\begin{knitrout}
\definecolor{shadecolor}{rgb}{0.969, 0.969, 0.969}\color{fgcolor}\begin{kframe}
\begin{alltt}
\hldef{f11} \hlkwb{<-} \hlkwa{function}\hldef{(}\hlkwc{x}\hldef{)} \hlnum{1} \hlopt{/} \hldef{x}

\hlkwd{three_point}\hldef{(f11,} \hlnum{1}\hldef{,} \hlnum{2}\hldef{)}
\end{alltt}
\begin{verbatim}
## [1] 0.6931217
\end{verbatim}
\end{kframe}
\end{knitrout}

\begin{knitrout}
\definecolor{shadecolor}{rgb}{0.969, 0.969, 0.969}\color{fgcolor}\begin{kframe}
\begin{alltt}
\hlkwd{three_point}\hldef{(f10,} \hlnum{0}\hldef{,} \hlnum{1}\hldef{)}
\end{alltt}
\begin{verbatim}
## [1] 0.785267
\end{verbatim}
\end{kframe}
\end{knitrout}

\begin{knitrout}
\definecolor{shadecolor}{rgb}{0.969, 0.969, 0.969}\color{fgcolor}\begin{kframe}
\begin{alltt}
\hldef{f12} \hlkwb{<-} \hlkwa{function}\hldef{(}\hlkwc{x}\hldef{)} \hlnum{1} \hlopt{/} \hlkwd{sqrt}\hldef{(}\hlnum{1} \hlopt{+} \hldef{x}\hlopt{^}\hlnum{4}\hldef{)}

\hlkwd{three_point}\hldef{(f12,} \hlnum{0}\hldef{,} \hlnum{1}\hldef{)}
\end{alltt}
\begin{verbatim}
## [1] 0.9271835
\end{verbatim}
\end{kframe}
\end{knitrout}

\begin{knitrout}
\definecolor{shadecolor}{rgb}{0.969, 0.969, 0.969}\color{fgcolor}\begin{kframe}
\begin{alltt}
\hldef{f13} \hlkwb{<-} \hlkwa{function}\hldef{(}\hlkwc{x}\hldef{)} \hlkwd{exp}\hldef{(}\hlopt{-}\hldef{x}\hlopt{^}\hlnum{2}\hldef{)}

\hlkwd{three_point}\hldef{(f13,} \hlnum{0.2}\hldef{,} \hlnum{1.5}\hldef{)}
\end{alltt}
\begin{verbatim}
## [1] 0.6586021
\end{verbatim}
\end{kframe}
\end{knitrout}

\end{document}
