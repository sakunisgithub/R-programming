\documentclass[11pt, a4paper]{article}\usepackage[]{graphicx}\usepackage[]{xcolor}
% maxwidth is the original width if it is less than linewidth
% otherwise use linewidth (to make sure the graphics do not exceed the margin)
\makeatletter
\def\maxwidth{ %
  \ifdim\Gin@nat@width>\linewidth
    \linewidth
  \else
    \Gin@nat@width
  \fi
}
\makeatother

\definecolor{fgcolor}{rgb}{0.345, 0.345, 0.345}
\newcommand{\hlnum}[1]{\textcolor[rgb]{0.686,0.059,0.569}{#1}}%
\newcommand{\hlsng}[1]{\textcolor[rgb]{0.192,0.494,0.8}{#1}}%
\newcommand{\hlcom}[1]{\textcolor[rgb]{0.678,0.584,0.686}{\textit{#1}}}%
\newcommand{\hlopt}[1]{\textcolor[rgb]{0,0,0}{#1}}%
\newcommand{\hldef}[1]{\textcolor[rgb]{0.345,0.345,0.345}{#1}}%
\newcommand{\hlkwa}[1]{\textcolor[rgb]{0.161,0.373,0.58}{\textbf{#1}}}%
\newcommand{\hlkwb}[1]{\textcolor[rgb]{0.69,0.353,0.396}{#1}}%
\newcommand{\hlkwc}[1]{\textcolor[rgb]{0.333,0.667,0.333}{#1}}%
\newcommand{\hlkwd}[1]{\textcolor[rgb]{0.737,0.353,0.396}{\textbf{#1}}}%
\let\hlipl\hlkwb

\usepackage{framed}
\makeatletter
\newenvironment{kframe}{%
 \def\at@end@of@kframe{}%
 \ifinner\ifhmode%
  \def\at@end@of@kframe{\end{minipage}}%
  \begin{minipage}{\columnwidth}%
 \fi\fi%
 \def\FrameCommand##1{\hskip\@totalleftmargin \hskip-\fboxsep
 \colorbox{shadecolor}{##1}\hskip-\fboxsep
     % There is no \\@totalrightmargin, so:
     \hskip-\linewidth \hskip-\@totalleftmargin \hskip\columnwidth}%
 \MakeFramed {\advance\hsize-\width
   \@totalleftmargin\z@ \linewidth\hsize
   \@setminipage}}%
 {\par\unskip\endMakeFramed%
 \at@end@of@kframe}
\makeatother

\definecolor{shadecolor}{rgb}{.97, .97, .97}
\definecolor{messagecolor}{rgb}{0, 0, 0}
\definecolor{warningcolor}{rgb}{1, 0, 1}
\definecolor{errorcolor}{rgb}{1, 0, 0}
\newenvironment{knitrout}{}{} % an empty environment to be redefined in TeX

\usepackage{alltt}

\usepackage[top=1 in, bottom = 1 in, left = 1 in, right = 1 in ]{geometry}

\usepackage{amsmath, amssymb, amsfonts}
\usepackage{enumerate}
\usepackage{dingbat}
\usepackage{fontawesome5}

\title{MSMS 106}
\author{Ananda Biswas}
\date{}
\IfFileExists{upquote.sty}{\usepackage{upquote}}{}
\begin{document}

\maketitle

\begin{center}
\textbf{Practical 02}
\end{center}

\smallpencil \hspace{0.5cm} \textcolor{blue}{\textbf{Implement bisection method for solution of numerical equations.}}

\vspace{0.5cm}

\faArrowAltCircleRight[regular] \hspace{0.5cm} In bisection method, first a sufficiently small interval $(a, b)$ containing at least one root of the equation $f(x) = 0$ is taken. We must have $f(a) \cdot f(b) < 0$. \\

\hspace{0.5cm} Let $x_1$ be the mid-point of the interval, i.e. $x_1 = \dfrac{a + b}{2}$. Then a real root of the equation must lie either in the interval $(a, x_1]$ or in the interval $[x_1, b)$. \\

\hspace{0.5cm} If $f(x_1) = 0$, then $x_1$ is a root of the equation. Otherwise if $f(a) \cdot f(x_1) < 0$, our interval of interest becomes $(a, x_1)$, or if $f(x_1) \cdot f(b) < 0$, our interval of interest becomes $(x_1, b)$. \\

\hspace{0.5cm} After repeating this procedure $n$ times, the mid-point of the last interval is taken as an approximate solution of the given equation $f(x) = 0$. \\

\hspace{0.5cm} The larger the value of $n$, the better will be the accuracy of the root.


\begin{knitrout}\footnotesize
\definecolor{shadecolor}{rgb}{0.969, 0.969, 0.969}\color{fgcolor}\begin{kframe}
\begin{alltt}
\hldef{bisection_method} \hlkwb{<-} \hlkwa{function}\hldef{(}\hlkwc{func}\hldef{,} \hlkwc{a}\hldef{,} \hlkwc{b}\hldef{,} \hlkwc{iterations}\hldef{)\{}

  \hlkwa{if}\hldef{(}\hlkwd{func}\hldef{(a)} \hlopt{*} \hlkwd{func}\hldef{(b)} \hlopt{>=} \hlnum{0}\hldef{)} \hlkwd{stop}\hldef{(}\hlsng{"Incorrect a or b or both."}\hldef{)}

  \hldef{i} \hlkwb{<-} \hlnum{1}

  \hlkwa{while}\hldef{(i} \hlopt{<=} \hldef{iterations)\{}

    \hldef{midpoint} \hlkwb{<-} \hldef{(a} \hlopt{+} \hldef{b)} \hlopt{/} \hlnum{2}

    \hlkwa{if}\hldef{(}\hlkwd{func}\hldef{(midpoint)} \hlopt{==} \hlnum{0}\hldef{)\{}
      \hlkwa{break}
    \hldef{\}} \hlkwa{else if}\hldef{(}\hlkwd{func}\hldef{(a)} \hlopt{*} \hlkwd{func}\hldef{(midpoint)} \hlopt{<} \hlnum{0}\hldef{)\{}
      \hldef{a} \hlkwb{<-} \hldef{a; b} \hlkwb{<-} \hldef{midpoint}
    \hldef{\}} \hlkwa{else if}\hldef{(}\hlkwd{func}\hldef{(midpoint)} \hlopt{*} \hlkwd{func}\hldef{(b)} \hlopt{<} \hlnum{0}\hldef{)\{}
      \hldef{a} \hlkwb{<-} \hldef{midpoint; b} \hlkwb{<-} \hldef{b}
    \hldef{\}}

    \hldef{i} \hlkwb{<-} \hldef{i} \hlopt{+} \hlnum{1}
  \hldef{\}}
  \hlkwd{return}\hldef{((a} \hlopt{+} \hldef{b)} \hlopt{/} \hlnum{2}\hldef{)}
\hldef{\}}
\end{alltt}
\end{kframe}
\end{knitrout}

\newpage

$\bullet$ \textbf{Example 1} : $f(x) = x^3 - 4x - 9$; $a = 2$; $b = 3$

\begin{knitrout}\footnotesize
\definecolor{shadecolor}{rgb}{0.969, 0.969, 0.969}\color{fgcolor}\begin{kframe}
\begin{alltt}
\hldef{f1} \hlkwb{<-} \hlkwa{function}\hldef{(}\hlkwc{x}\hldef{) x}\hlopt{^}\hlnum{3} \hlopt{-} \hlnum{4}\hlopt{*}\hldef{x} \hlopt{-} \hlnum{9}
\hldef{sol1} \hlkwb{<-} \hlkwd{bisection_method}\hldef{(}\hlkwc{func} \hldef{= f1,} \hlkwc{a} \hldef{=} \hlnum{2}\hldef{,} \hlkwc{b} \hldef{=} \hlnum{3}\hldef{,} \hlkwc{iterations} \hldef{=} \hlnum{5}\hldef{)}
\hldef{sol1}
\end{alltt}
\begin{verbatim}
## [1] 2.703125
\end{verbatim}
\end{kframe}
\end{knitrout}

\begin{knitrout}\footnotesize
\definecolor{shadecolor}{rgb}{0.969, 0.969, 0.969}\color{fgcolor}\begin{kframe}
\begin{alltt}
\hlkwd{f1}\hldef{(sol1)}
\end{alltt}
\begin{verbatim}
## [1] -0.06107712
\end{verbatim}
\end{kframe}
\end{knitrout}

$\bullet$ \textbf{Example 2} : $f(x) = x^4 + 2x^2 - x - 1$; $a = 0$; $b = 1$

\begin{knitrout}\footnotesize
\definecolor{shadecolor}{rgb}{0.969, 0.969, 0.969}\color{fgcolor}\begin{kframe}
\begin{alltt}
\hldef{f2} \hlkwb{<-} \hlkwa{function}\hldef{(}\hlkwc{x}\hldef{) x}\hlopt{^}\hlnum{4} \hlopt{+} \hlnum{2}\hlopt{*}\hldef{(x}\hlopt{^}\hlnum{2}\hldef{)} \hlopt{-} \hldef{x} \hlopt{-} \hlnum{1}
\hldef{sol2} \hlkwb{<-} \hlkwd{bisection_method}\hldef{(}\hlkwc{func} \hldef{= f2,} \hlkwc{a} \hldef{=} \hlnum{0}\hldef{,} \hlkwc{b} \hldef{=} \hlnum{1}\hldef{,} \hlkwc{iterations} \hldef{=} \hlnum{7}\hldef{)}
\hldef{sol2}
\end{alltt}
\begin{verbatim}
## [1] 0.8242188
\end{verbatim}
\end{kframe}
\end{knitrout}

\begin{knitrout}\footnotesize
\definecolor{shadecolor}{rgb}{0.969, 0.969, 0.969}\color{fgcolor}\begin{kframe}
\begin{alltt}
\hlkwd{f2}\hldef{(sol2)}
\end{alltt}
\begin{verbatim}
## [1] -0.004047509
\end{verbatim}
\end{kframe}
\end{knitrout}

$\bullet$ \textbf{Example 3} : $f(x) = x^3 - x - 1$; $a = 1.25$; $b = 1.5$

\begin{knitrout}\footnotesize
\definecolor{shadecolor}{rgb}{0.969, 0.969, 0.969}\color{fgcolor}\begin{kframe}
\begin{alltt}
\hldef{f3} \hlkwb{<-} \hlkwa{function}\hldef{(}\hlkwc{x}\hldef{) x}\hlopt{^}\hlnum{3} \hlopt{-} \hldef{x} \hlopt{-} \hlnum{1}
\hldef{sol3} \hlkwb{<-} \hlkwd{bisection_method}\hldef{(}\hlkwc{func} \hldef{= f3,} \hlkwc{a} \hldef{=} \hlnum{1.25}\hldef{,} \hlkwc{b} \hldef{=} \hlnum{1.5}\hldef{,} \hlkwc{iterations} \hldef{=} \hlnum{8}\hldef{)}
\hldef{sol3}
\end{alltt}
\begin{verbatim}
## [1] 1.324707
\end{verbatim}
\end{kframe}
\end{knitrout}

\begin{knitrout}\footnotesize
\definecolor{shadecolor}{rgb}{0.969, 0.969, 0.969}\color{fgcolor}\begin{kframe}
\begin{alltt}
\hlkwd{f3}\hldef{(sol3)}
\end{alltt}
\begin{verbatim}
## [1] -4.659488e-05
\end{verbatim}
\end{kframe}
\end{knitrout}

\end{document}
