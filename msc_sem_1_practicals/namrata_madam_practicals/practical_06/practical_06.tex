\documentclass[11pt, a4paper]{article}\usepackage[]{graphicx}\usepackage[]{xcolor}
% maxwidth is the original width if it is less than linewidth
% otherwise use linewidth (to make sure the graphics do not exceed the margin)
\makeatletter
\def\maxwidth{ %
  \ifdim\Gin@nat@width>\linewidth
    \linewidth
  \else
    \Gin@nat@width
  \fi
}
\makeatother

\definecolor{fgcolor}{rgb}{0.345, 0.345, 0.345}
\newcommand{\hlnum}[1]{\textcolor[rgb]{0.686,0.059,0.569}{#1}}%
\newcommand{\hlsng}[1]{\textcolor[rgb]{0.192,0.494,0.8}{#1}}%
\newcommand{\hlcom}[1]{\textcolor[rgb]{0.678,0.584,0.686}{\textit{#1}}}%
\newcommand{\hlopt}[1]{\textcolor[rgb]{0,0,0}{#1}}%
\newcommand{\hldef}[1]{\textcolor[rgb]{0.345,0.345,0.345}{#1}}%
\newcommand{\hlkwa}[1]{\textcolor[rgb]{0.161,0.373,0.58}{\textbf{#1}}}%
\newcommand{\hlkwb}[1]{\textcolor[rgb]{0.69,0.353,0.396}{#1}}%
\newcommand{\hlkwc}[1]{\textcolor[rgb]{0.333,0.667,0.333}{#1}}%
\newcommand{\hlkwd}[1]{\textcolor[rgb]{0.737,0.353,0.396}{\textbf{#1}}}%
\let\hlipl\hlkwb

\usepackage{framed}
\makeatletter
\newenvironment{kframe}{%
 \def\at@end@of@kframe{}%
 \ifinner\ifhmode%
  \def\at@end@of@kframe{\end{minipage}}%
  \begin{minipage}{\columnwidth}%
 \fi\fi%
 \def\FrameCommand##1{\hskip\@totalleftmargin \hskip-\fboxsep
 \colorbox{shadecolor}{##1}\hskip-\fboxsep
     % There is no \\@totalrightmargin, so:
     \hskip-\linewidth \hskip-\@totalleftmargin \hskip\columnwidth}%
 \MakeFramed {\advance\hsize-\width
   \@totalleftmargin\z@ \linewidth\hsize
   \@setminipage}}%
 {\par\unskip\endMakeFramed%
 \at@end@of@kframe}
\makeatother

\definecolor{shadecolor}{rgb}{.97, .97, .97}
\definecolor{messagecolor}{rgb}{0, 0, 0}
\definecolor{warningcolor}{rgb}{1, 0, 1}
\definecolor{errorcolor}{rgb}{1, 0, 0}
\newenvironment{knitrout}{}{} % an empty environment to be redefined in TeX

\usepackage{alltt}

\usepackage[top=1 in, bottom = 1 in, left = 1 in, right = 1 in ]{geometry}

\usepackage{amsmath, amssymb, amsfonts}
\usepackage{enumerate}
\usepackage{dingbat}
\usepackage{fontawesome5}
\usepackage{twemojis}

\title{MSMS 106}
\author{Ananda Biswas}
\date{November 28, 2024}
\IfFileExists{upquote.sty}{\usepackage{upquote}}{}
\begin{document}

\maketitle

\begin{center}
\textbf{Practical 06}
\end{center}

\scalebox{2.0}{\twemoji{bullseye}} \hspace{0.2cm} \textcolor{blue}{\textbf{Composite Integration}}

\vspace{0.5cm}

\smallpencil \hspace{0.2cm} \textcolor{blue}{\textbf{Question 1 : }} \\

\hspace{1cm} Write an R program to approximate the integral of $f(x)$ over the interval $[a, b]$ using Simpson's $\dfrac{1}{3}$ rule with $n$ sub-intervals of equal length. \\

\faArrowAltCircleRight[regular] \hspace{0.2cm} Simpson's $\dfrac{1}{3}$ rule is given by $$\int \limits_{a}^{b} f(x) dx \approx h \left[ \dfrac{1}{3} f(a) + \dfrac{4}{3} f\left(\dfrac{a+b}{2}\right) + \dfrac{1}{3} f(b) \right]; \,\,\, \text{where } h = \dfrac{b-a}{2}. $$

For composite integration, we divide the interval $[a, b]$ into $2N$ sub-intervals each of length $h = \dfrac{b-a}{2N}$; then we get $2N + 1$ abscissas $x_0, x_1, \ldots, x_{2N-1}, x_{2N}$ with $x_0 = a$, $x_{2N} = b$ and $x_i = x_0 + ih \,\, \forall i = 1(1)\overline{2N-1}$. \\

We write $$\int \limits_{a}^{b} f(x) dx = \int \limits_{x_0}^{x_2} f(x) dx + \int \limits_{x_2}^{x_4} f(x) dx + \cdots + \int \limits_{x_{2N-2}}^{x_{2N}} f(x) dx$$ and individually apply Simpson's $\dfrac{1}{3}$ rule to each of the integrals in RHS.

\begin{knitrout}
\definecolor{shadecolor}{rgb}{0.969, 0.969, 0.969}\color{fgcolor}\begin{kframe}
\begin{alltt}
\hldef{simpson_one_third} \hlkwb{<-} \hlkwa{function}\hldef{(}\hlkwc{func}\hldef{,} \hlkwc{a}\hldef{,} \hlkwc{b}\hldef{)\{}
  \hldef{h} \hlkwb{<-} \hldef{(b} \hlopt{-} \hldef{a)} \hlopt{/} \hlnum{2}

  \hldef{x} \hlkwb{<-} \hlkwd{seq}\hldef{(}\hlkwc{from} \hldef{= a,} \hlkwc{to} \hldef{= b,} \hlkwc{by} \hldef{= h)}

  \hldef{s} \hlkwb{<-} \hldef{(}\hlnum{1}\hlopt{/}\hlnum{3}\hldef{)} \hlopt{*} \hlkwd{func}\hldef{(x[}\hlnum{1}\hldef{])} \hlopt{+} \hldef{(}\hlnum{4}\hlopt{/}\hlnum{3}\hldef{)} \hlopt{*} \hlkwd{func}\hldef{(x[}\hlnum{2}\hldef{])} \hlopt{+} \hldef{(}\hlnum{1}\hlopt{/}\hlnum{3}\hldef{)} \hlopt{*} \hlkwd{func}\hldef{(x[}\hlnum{3}\hldef{])}

  \hlkwd{return}\hldef{(h} \hlopt{*} \hldef{s)}
\hldef{\}}
\end{alltt}
\end{kframe}
\end{knitrout}

\begin{knitrout}
\definecolor{shadecolor}{rgb}{0.969, 0.969, 0.969}\color{fgcolor}\begin{kframe}
\begin{alltt}
\hldef{composite_simpson_one_third} \hlkwb{<-} \hlkwa{function}\hldef{(}\hlkwc{func}\hldef{,} \hlkwc{a}\hldef{,} \hlkwc{b}\hldef{,} \hlkwc{n}\hldef{)\{}

  \hlkwa{if}\hldef{(n} \hlopt \hlnum{2} \hlopt{!=} \hlnum{0}\hldef{)} \hlkwd{stop}\hldef{(}\hlsng{"number of intervals must be even"}\hldef{)}

  \hldef{h} \hlkwb{<-} \hldef{(b} \hlopt{-} \hldef{a)} \hlopt{/} \hldef{n}

  \hldef{x} \hlkwb{<-} \hlkwd{seq}\hldef{(}\hlkwc{from} \hldef{= a,} \hlkwc{to} \hldef{= b,} \hlkwc{by} \hldef{= h)}

  \hldef{result} \hlkwb{<-} \hlnum{0}

  \hlcom{# x_indices is required to access the lower limits}
  \hldef{x_indices} \hlkwb{<-} \hlkwd{seq}\hldef{(}\hlkwc{from} \hldef{=} \hlnum{1}\hldef{,} \hlkwc{to} \hldef{= n} \hlopt{-} \hlnum{1}\hldef{,} \hlkwc{by} \hldef{=} \hlnum{2}\hldef{)}

  \hlkwa{for} \hldef{(i} \hlkwa{in} \hldef{x_indices) \{}
    \hldef{result} \hlkwb{<-} \hldef{result} \hlopt{+} \hlkwd{simpson_one_third}\hldef{(func, x[i], x[i}\hlopt{+}\hlnum{2}\hldef{])}
  \hldef{\}}

  \hlkwd{return}\hldef{(result)}
\hldef{\}}
\end{alltt}
\end{kframe}
\end{knitrout}

$\bullet$ $f(x) = x^2$, $[a, b] = [0, 2]$ and number of intervals = 4

\begin{knitrout}
\definecolor{shadecolor}{rgb}{0.969, 0.969, 0.969}\color{fgcolor}\begin{kframe}
\begin{alltt}
\hldef{f1} \hlkwb{<-} \hlkwa{function}\hldef{(}\hlkwc{x}\hldef{) x}\hlopt{^}\hlnum{2}

\hlkwd{composite_simpson_one_third}\hldef{(f1,} \hlnum{0}\hldef{,} \hlnum{2}\hldef{,} \hlnum{4}\hldef{)}
\end{alltt}
\begin{verbatim}
## [1] 2.666667
\end{verbatim}
\end{kframe}
\end{knitrout}

$\bullet$ $f(x) = \dfrac{1}{1+x^2}$, $[a, b] = [0, 1]$ and number of intervals = 4, 6

\begin{knitrout}
\definecolor{shadecolor}{rgb}{0.969, 0.969, 0.969}\color{fgcolor}\begin{kframe}
\begin{alltt}
\hldef{f2} \hlkwb{<-} \hlkwa{function}\hldef{(}\hlkwc{x}\hldef{)} \hlnum{1} \hlopt{/} \hldef{(}\hlnum{1} \hlopt{+} \hldef{x}\hlopt{^}\hlnum{2}\hldef{)}

\hlkwd{composite_simpson_one_third}\hldef{(f2,} \hlnum{0}\hldef{,} \hlnum{1}\hldef{,} \hlnum{4}\hldef{)}
\end{alltt}
\begin{verbatim}
## [1] 0.7853922
\end{verbatim}
\end{kframe}
\end{knitrout}

\begin{knitrout}
\definecolor{shadecolor}{rgb}{0.969, 0.969, 0.969}\color{fgcolor}\begin{kframe}
\begin{alltt}
\hlkwd{composite_simpson_one_third}\hldef{(f2,} \hlnum{0}\hldef{,} \hlnum{1}\hldef{,} \hlnum{6}\hldef{)}
\end{alltt}
\begin{verbatim}
## [1] 0.7853979
\end{verbatim}
\end{kframe}
\end{knitrout}

$\bullet$ $f(x) = \dfrac{1}{1+x}$, $[a, b] = [0, 1]$, number of intervals = 2, 4, 8

\begin{knitrout}
\definecolor{shadecolor}{rgb}{0.969, 0.969, 0.969}\color{fgcolor}\begin{kframe}
\begin{alltt}
\hldef{f3} \hlkwb{<-} \hlkwa{function}\hldef{(}\hlkwc{x}\hldef{)} \hlnum{1} \hlopt{/} \hldef{(}\hlnum{1} \hlopt{+} \hldef{x)}

\hlkwd{composite_simpson_one_third}\hldef{(f3,} \hlnum{0}\hldef{,} \hlnum{1}\hldef{,} \hlnum{2}\hldef{)}
\end{alltt}
\begin{verbatim}
## [1] 0.6944444
\end{verbatim}
\end{kframe}
\end{knitrout}

\begin{knitrout}
\definecolor{shadecolor}{rgb}{0.969, 0.969, 0.969}\color{fgcolor}\begin{kframe}
\begin{alltt}
\hlkwd{composite_simpson_one_third}\hldef{(f3,} \hlnum{0}\hldef{,} \hlnum{1}\hldef{,} \hlnum{4}\hldef{)}
\end{alltt}
\begin{verbatim}
## [1] 0.693254
\end{verbatim}
\end{kframe}
\end{knitrout}

\begin{knitrout}
\definecolor{shadecolor}{rgb}{0.969, 0.969, 0.969}\color{fgcolor}\begin{kframe}
\begin{alltt}
\hlkwd{composite_simpson_one_third}\hldef{(f3,} \hlnum{0}\hldef{,} \hlnum{1}\hldef{,} \hlnum{8}\hldef{)}
\end{alltt}
\begin{verbatim}
## [1] 0.6931545
\end{verbatim}
\end{kframe}
\end{knitrout}

\smallpencil \hspace{0.2cm} \textcolor{blue}{\textbf{Question 2 : }} \\

\hspace{1cm} Write an R program to approximate the integral of $f(x)$ over the interval $[a, b]$ using Simpson's $\dfrac{3}{8}$ rule with $n$ sub-intervals of equal length.


\faArrowAltCircleRight[regular] \hspace{0.2cm} Simpson's $\dfrac{3}{8}$ rule is given by $$\int \limits_{a}^{b} f(x) dx \approx \dfrac{3h}{8} \left[ f(a) + 3 f(a+h) + 3 f(a+2h) + f(b) \right]; \,\,\, \text{where } h = \dfrac{b-a}{3}.$$

For composite integration, we divide the interval $[a, b]$ into $3N$ sub-intervals each of length $h = \dfrac{b-a}{3N}$; then we get $3N + 1$ abscissas $x_0, x_1, \ldots, x_{3N-3}, x_{3N-2}, x_{3N-1}, x_{3N}$ with $x_0 = a$, $x_{3N} = b$ and $x_i = x_0 + ih \,\, \forall i = 1(1)\overline{3N-1}$. \\

We write $$\int \limits_{a}^{b} f(x) dx = \int \limits_{x_0}^{x_3} f(x) dx + \int \limits_{x_3}^{x_6} f(x) dx + \cdots + \int \limits_{x_{3N-3}}^{x_{3N}} f(x) dx$$ and individually apply Simpson's $\dfrac{3}{8}$ rule to each of the integrals in RHS.

\begin{knitrout}
\definecolor{shadecolor}{rgb}{0.969, 0.969, 0.969}\color{fgcolor}\begin{kframe}
\begin{alltt}
\hldef{simpson_three_eight} \hlkwb{<-} \hlkwa{function}\hldef{(}\hlkwc{func}\hldef{,} \hlkwc{a}\hldef{,} \hlkwc{b}\hldef{)\{}
  \hldef{h} \hlkwb{<-} \hldef{(b} \hlopt{-} \hldef{a)} \hlopt{/} \hlnum{3}

  \hldef{x} \hlkwb{<-} \hlkwd{seq}\hldef{(}\hlkwc{from} \hldef{= a,} \hlkwc{to} \hldef{= b,} \hlkwc{by} \hldef{= h)}

  \hldef{s} \hlkwb{<-} \hldef{(}\hlnum{3}\hlopt{/}\hlnum{8}\hldef{)} \hlopt{*} \hlkwd{func}\hldef{(x[}\hlnum{1}\hldef{])} \hlopt{+} \hldef{(}\hlnum{9}\hlopt{/}\hlnum{8}\hldef{)} \hlopt{*} \hlkwd{func}\hldef{(x[}\hlnum{2}\hldef{])} \hlopt{+} \hldef{(}\hlnum{9}\hlopt{/}\hlnum{8}\hldef{)} \hlopt{*} \hlkwd{func}\hldef{(x[}\hlnum{3}\hldef{])} \hlopt{+} \hldef{(}\hlnum{3}\hlopt{/}\hlnum{8}\hldef{)} \hlopt{*} \hlkwd{func}\hldef{(x[}\hlnum{4}\hldef{])}

  \hlkwd{return}\hldef{(h} \hlopt{*} \hldef{s)}
\hldef{\}}
\end{alltt}
\end{kframe}
\end{knitrout}

\begin{knitrout}
\definecolor{shadecolor}{rgb}{0.969, 0.969, 0.969}\color{fgcolor}\begin{kframe}
\begin{alltt}
\hldef{composite_simpson_three_eight} \hlkwb{<-} \hlkwa{function}\hldef{(}\hlkwc{func}\hldef{,} \hlkwc{a}\hldef{,} \hlkwc{b}\hldef{,} \hlkwc{n}\hldef{)\{}

  \hlkwa{if}\hldef{(n} \hlopt \hlnum{3} \hlopt{!=} \hlnum{0}\hldef{)} \hlkwd{stop}\hldef{(}\hlsng{"number of intervals must be a multiple of 3"}\hldef{)}

  \hldef{h} \hlkwb{<-} \hldef{(b} \hlopt{-} \hldef{a)} \hlopt{/} \hldef{n}

  \hldef{x} \hlkwb{<-} \hlkwd{seq}\hldef{(}\hlkwc{from} \hldef{= a,} \hlkwc{to} \hldef{= b,} \hlkwc{by} \hldef{= h)}

  \hldef{result} \hlkwb{<-} \hlnum{0}

  \hlcom{# x_indices is required to access the lower limits}
  \hldef{x_indices} \hlkwb{<-} \hlkwd{seq}\hldef{(}\hlkwc{from} \hldef{=} \hlnum{1}\hldef{,} \hlkwc{to} \hldef{= n} \hlopt{-} \hlnum{1}\hldef{,} \hlkwc{by} \hldef{=} \hlnum{3}\hldef{)}

  \hlkwa{for} \hldef{(i} \hlkwa{in} \hldef{x_indices) \{}
    \hldef{result} \hlkwb{<-} \hldef{result} \hlopt{+} \hlkwd{simpson_three_eight}\hldef{(func, x[i], x[i}\hlopt{+}\hlnum{3}\hldef{])}
  \hldef{\}}

  \hlkwd{return}\hldef{(result)}
\hldef{\}}
\end{alltt}
\end{kframe}
\end{knitrout}

$\bullet$ $f(x) = x^2$, $[a, b] = [0, 1]$ and number of intervals = 6

\begin{knitrout}
\definecolor{shadecolor}{rgb}{0.969, 0.969, 0.969}\color{fgcolor}\begin{kframe}
\begin{alltt}
\hlkwd{composite_simpson_three_eight}\hldef{(f1,} \hlnum{0}\hldef{,} \hlnum{1}\hldef{,} \hlnum{6}\hldef{)}
\end{alltt}
\begin{verbatim}
## [1] 0.3333333
\end{verbatim}
\end{kframe}
\end{knitrout}

$\bullet$ $f(x) = e^x$, $[a, b] = [0, 1]$ and number of intervals = 6

\begin{knitrout}
\definecolor{shadecolor}{rgb}{0.969, 0.969, 0.969}\color{fgcolor}\begin{kframe}
\begin{alltt}
\hldef{f4} \hlkwb{<-} \hlkwa{function}\hldef{(}\hlkwc{x}\hldef{)} \hlkwd{exp}\hldef{(x)}

\hlkwd{composite_simpson_three_eight}\hldef{(f4,} \hlnum{0}\hldef{,} \hlnum{1}\hldef{,} \hlnum{6}\hldef{)}
\end{alltt}
\begin{verbatim}
## [1] 1.718298
\end{verbatim}
\end{kframe}
\end{knitrout}

\smallpencil \hspace{0.2cm} \textcolor{blue}{\textbf{Question 3 : }} \\

\hspace{1cm} Write an R program to approximate the integral of $f(x)$ over the interval $[a, b]$ using Trapezoidal rule with $n$ sub-intervals of equal length. \\

\faArrowAltCircleRight[regular] \hspace{0.2cm} Trapezoidal Rule is given by $$\int \limits_{a}^{b} f(x) dx = \dfrac{b-a}{2} \left[ f(a) + f(b) \right].$$

For composite integration, we divide the interval $[a, b]$ into $N$ sub-intervals each of length $h = \dfrac{b-a}{N}$; then we get $N + 1$ abscissas $x_0, x_1, \ldots, x_{N-1}, x_{N}$ with $x_0 = a$, $x_{N} = b$ and $x_i = x_0 + ih \,\, \forall i = 1(1)\overline{N-1}$. \\

We write $$\int \limits_{a}^{b} f(x) dx = \int \limits_{x_0}^{x_1} f(x) dx + \int \limits_{x_1}^{x_2} f(x) dx + \cdots + \int \limits_{x_{N-1}}^{x_{N}} f(x) dx$$ and individually apply Trapezoidal rule to each of the integrals in RHS.

\begin{knitrout}
\definecolor{shadecolor}{rgb}{0.969, 0.969, 0.969}\color{fgcolor}\begin{kframe}
\begin{alltt}
\hldef{trapezoidal_rule} \hlkwb{<-} \hlkwa{function}\hldef{(}\hlkwc{func}\hldef{,} \hlkwc{a}\hldef{,} \hlkwc{b}\hldef{)\{}
  \hldef{h} \hlkwb{<-} \hldef{(b} \hlopt{-} \hldef{a)} \hlopt{/} \hlnum{2}

  \hldef{result} \hlkwb{<-} \hldef{h} \hlopt{*} \hldef{(} \hlkwd{func}\hldef{(a)} \hlopt{+} \hlkwd{func}\hldef{(b))}

  \hlkwd{return}\hldef{(result)}
\hldef{\}}
\end{alltt}
\end{kframe}
\end{knitrout}

\begin{knitrout}
\definecolor{shadecolor}{rgb}{0.969, 0.969, 0.969}\color{fgcolor}\begin{kframe}
\begin{alltt}
\hldef{composite_trapezoidal_rule} \hlkwb{<-} \hlkwa{function}\hldef{(}\hlkwc{func}\hldef{,} \hlkwc{a}\hldef{,} \hlkwc{b}\hldef{,} \hlkwc{n}\hldef{)\{}

  \hldef{h} \hlkwb{<-} \hldef{(b} \hlopt{-} \hldef{a)} \hlopt{/} \hldef{n}

  \hldef{x} \hlkwb{<-} \hlkwd{seq}\hldef{(}\hlkwc{from} \hldef{= a,} \hlkwc{to} \hldef{= b,} \hlkwc{by} \hldef{= h)}

  \hldef{result} \hlkwb{<-} \hlnum{0}

  \hlkwa{for} \hldef{(i} \hlkwa{in} \hlnum{1}\hlopt{:}\hldef{n) \{}
    \hldef{result} \hlkwb{<-} \hldef{result} \hlopt{+} \hlkwd{trapezoidal_rule}\hldef{(func, x[i], x[i}\hlopt{+}\hlnum{1}\hldef{])}
  \hldef{\}}

  \hlkwd{return}\hldef{(result)}
\hldef{\}}
\end{alltt}
\end{kframe}
\end{knitrout}

$\bullet$ $f(x) = x^2$, $[a, b] = [0, 1]$ and number of intervals = 10

\begin{knitrout}
\definecolor{shadecolor}{rgb}{0.969, 0.969, 0.969}\color{fgcolor}\begin{kframe}
\begin{alltt}
\hlkwd{composite_trapezoidal_rule}\hldef{(f1,} \hlnum{0}\hldef{,} \hlnum{1}\hldef{,} \hlnum{10}\hldef{)}
\end{alltt}
\begin{verbatim}
## [1] 0.335
\end{verbatim}
\end{kframe}
\end{knitrout}

$\bullet$ $f(x) = \dfrac{1}{1 + x}$, $[a, b] = [0, 1]$ and number of intervals = 2, 4, 8

\begin{knitrout}
\definecolor{shadecolor}{rgb}{0.969, 0.969, 0.969}\color{fgcolor}\begin{kframe}
\begin{alltt}
\hlkwd{composite_trapezoidal_rule}\hldef{(f3,} \hlnum{0}\hldef{,} \hlnum{1}\hldef{,} \hlnum{2}\hldef{)}
\end{alltt}
\begin{verbatim}
## [1] 0.7083333
\end{verbatim}
\end{kframe}
\end{knitrout}

\begin{knitrout}
\definecolor{shadecolor}{rgb}{0.969, 0.969, 0.969}\color{fgcolor}\begin{kframe}
\begin{alltt}
\hlkwd{composite_trapezoidal_rule}\hldef{(f3,} \hlnum{0}\hldef{,} \hlnum{1}\hldef{,} \hlnum{4}\hldef{)}
\end{alltt}
\begin{verbatim}
## [1] 0.6970238
\end{verbatim}
\end{kframe}
\end{knitrout}

\begin{knitrout}
\definecolor{shadecolor}{rgb}{0.969, 0.969, 0.969}\color{fgcolor}\begin{kframe}
\begin{alltt}
\hlkwd{composite_trapezoidal_rule}\hldef{(f3,} \hlnum{0}\hldef{,} \hlnum{1}\hldef{,} \hlnum{8}\hldef{)}
\end{alltt}
\begin{verbatim}
## [1] 0.6941219
\end{verbatim}
\end{kframe}
\end{knitrout}

\end{document}
