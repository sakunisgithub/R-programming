\documentclass[11pt, a4paper]{article}\usepackage[]{graphicx}\usepackage[]{xcolor}
% maxwidth is the original width if it is less than linewidth
% otherwise use linewidth (to make sure the graphics do not exceed the margin)
\makeatletter
\def\maxwidth{ %
  \ifdim\Gin@nat@width>\linewidth
    \linewidth
  \else
    \Gin@nat@width
  \fi
}
\makeatother

\definecolor{fgcolor}{rgb}{0.345, 0.345, 0.345}
\newcommand{\hlnum}[1]{\textcolor[rgb]{0.686,0.059,0.569}{#1}}%
\newcommand{\hlsng}[1]{\textcolor[rgb]{0.192,0.494,0.8}{#1}}%
\newcommand{\hlcom}[1]{\textcolor[rgb]{0.678,0.584,0.686}{\textit{#1}}}%
\newcommand{\hlopt}[1]{\textcolor[rgb]{0,0,0}{#1}}%
\newcommand{\hldef}[1]{\textcolor[rgb]{0.345,0.345,0.345}{#1}}%
\newcommand{\hlkwa}[1]{\textcolor[rgb]{0.161,0.373,0.58}{\textbf{#1}}}%
\newcommand{\hlkwb}[1]{\textcolor[rgb]{0.69,0.353,0.396}{#1}}%
\newcommand{\hlkwc}[1]{\textcolor[rgb]{0.333,0.667,0.333}{#1}}%
\newcommand{\hlkwd}[1]{\textcolor[rgb]{0.737,0.353,0.396}{\textbf{#1}}}%
\let\hlipl\hlkwb

\usepackage{framed}
\makeatletter
\newenvironment{kframe}{%
 \def\at@end@of@kframe{}%
 \ifinner\ifhmode%
  \def\at@end@of@kframe{\end{minipage}}%
  \begin{minipage}{\columnwidth}%
 \fi\fi%
 \def\FrameCommand##1{\hskip\@totalleftmargin \hskip-\fboxsep
 \colorbox{shadecolor}{##1}\hskip-\fboxsep
     % There is no \\@totalrightmargin, so:
     \hskip-\linewidth \hskip-\@totalleftmargin \hskip\columnwidth}%
 \MakeFramed {\advance\hsize-\width
   \@totalleftmargin\z@ \linewidth\hsize
   \@setminipage}}%
 {\par\unskip\endMakeFramed%
 \at@end@of@kframe}
\makeatother

\definecolor{shadecolor}{rgb}{.97, .97, .97}
\definecolor{messagecolor}{rgb}{0, 0, 0}
\definecolor{warningcolor}{rgb}{1, 0, 1}
\definecolor{errorcolor}{rgb}{1, 0, 0}
\newenvironment{knitrout}{}{} % an empty environment to be redefined in TeX

\usepackage{alltt}

\usepackage[top=1 in, bottom = 1 in, left = 1 in, right = 1 in ]{geometry}

\usepackage{amsmath, amssymb, amsfonts}
\usepackage{enumerate}
\usepackage{dingbat}
\usepackage{fontawesome5}

\title{MSMS 106}
\author{Ananda Biswas}
\date{}
\IfFileExists{upquote.sty}{\usepackage{upquote}}{}
\begin{document}

\maketitle

\begin{center}
\textbf{Practical 03}
\end{center}

\smallpencil \hspace{0.5cm} \textcolor{blue}{\textbf{Implement Newton-Raphson method for solution of single-variable numerical equations.}}

\vspace{0.5cm}

\faArrowAltCircleRight[regular] \hspace{0.5cm} Suppose $f(x) = 0$ be our equation and $f'(x)$ exists for all $x$. \\

We start with an initial approximation $x_0$ and successively calculate $$x_{n+1} = x_n - \dfrac{f(x_n)}{f'(x_n)} \,\, \text{for} \,\, n = 0, 1, 2, \ldots.$$ Upon reaching desired accuracy or after doing a certain number of iterations, we report final $x_n$ as our approximate solution to $f(x) = 0$.

\begin{knitrout}
\definecolor{shadecolor}{rgb}{0.969, 0.969, 0.969}\color{fgcolor}\begin{kframe}
\begin{alltt}
\hldef{newton_raphson_1} \hlkwb{<-} \hlkwa{function}\hldef{(}\hlkwc{func}\hldef{,} \hlkwc{x_0}\hldef{,} \hlkwc{iterations}\hldef{)\{}
  \hlkwd{library}\hldef{(Deriv)}

  \hldef{func_dash} \hlkwb{<-} \hlkwd{Deriv}\hldef{(func)}

  \hldef{i} \hlkwb{<-} \hlnum{1}
  \hldef{x} \hlkwb{<-} \hlkwd{c}\hldef{(x_0)}

  \hlkwa{while}\hldef{(i} \hlopt{<=} \hldef{iterations)\{}
    \hldef{x[i}\hlopt{+}\hlnum{1}\hldef{]} \hlkwb{<-} \hldef{x[i]} \hlopt{-} \hlkwd{func}\hldef{(x[i])} \hlopt{/} \hlkwd{func_dash}\hldef{(x[i])}

    \hlkwa{if}\hldef{(}\hlkwd{abs}\hldef{(}\hlkwd{func}\hldef{(x[}\hlkwd{length}\hldef{(x)]))} \hlopt{<} \hlnum{0.001}\hldef{)} \hlkwa{break}

    \hldef{i} \hlkwb{<-} \hldef{i} \hlopt{+} \hlnum{1}
  \hldef{\}}

  \hlkwd{return}\hldef{(x[}\hlkwd{length}\hldef{(x)])}
\hldef{\}}
\end{alltt}
\end{kframe}
\end{knitrout}

\newpage

$\bullet$ \textbf{Example 1} : $f(x) = \dfrac{1}{3} x^3 - 5x + 1$

\begin{knitrout}
\definecolor{shadecolor}{rgb}{0.969, 0.969, 0.969}\color{fgcolor}\begin{kframe}
\begin{alltt}
\hldef{f1} \hlkwb{<-} \hlkwa{function}\hldef{(}\hlkwc{x}\hldef{) (}\hlnum{1}\hlopt{/}\hlnum{3}\hldef{)} \hlopt{*} \hldef{x}\hlopt{^}\hlnum{3} \hlopt{-} \hlnum{5}\hlopt{*}\hldef{x} \hlopt{+} \hlnum{1}
\hldef{sol1} \hlkwb{<-} \hlkwd{newton_raphson_1}\hldef{(}\hlkwc{func} \hldef{= f1,} \hlkwc{x_0} \hldef{=} \hlnum{1}\hldef{,} \hlkwc{iterations} \hldef{=} \hlnum{100}\hldef{)}
\hldef{sol1}
\end{alltt}
\begin{verbatim}
## [1] 0.2005376
\end{verbatim}
\end{kframe}
\end{knitrout}

\begin{knitrout}
\definecolor{shadecolor}{rgb}{0.969, 0.969, 0.969}\color{fgcolor}\begin{kframe}
\begin{alltt}
\hlkwd{f1}\hldef{(sol1)}
\end{alltt}
\begin{verbatim}
## [1] 2.271497e-08
\end{verbatim}
\end{kframe}
\end{knitrout}

$\bullet$ \textbf{Example 2} : $f(x) = x^4 + 2x^3 + 2x - 2$

\begin{knitrout}
\definecolor{shadecolor}{rgb}{0.969, 0.969, 0.969}\color{fgcolor}\begin{kframe}
\begin{alltt}
\hldef{f2} \hlkwb{<-} \hlkwa{function}\hldef{(}\hlkwc{x}\hldef{) x}\hlopt{^}\hlnum{4} \hlopt{-} \hlnum{2} \hlopt{*} \hldef{x}\hlopt{^}\hlnum{3} \hlopt{+} \hlnum{2}\hlopt{*}\hldef{x} \hlopt{-} \hlnum{2}
\hldef{sol2} \hlkwb{<-} \hlkwd{newton_raphson_1}\hldef{(}\hlkwc{func} \hldef{= f2,} \hlkwc{x_0} \hldef{=} \hlnum{2}\hldef{,} \hlkwc{iterations} \hldef{=} \hlnum{100}\hldef{)}
\hldef{sol2}
\end{alltt}
\begin{verbatim}
## [1] 1.716822
\end{verbatim}
\end{kframe}
\end{knitrout}

\begin{knitrout}
\definecolor{shadecolor}{rgb}{0.969, 0.969, 0.969}\color{fgcolor}\begin{kframe}
\begin{alltt}
\hlkwd{f2}\hldef{(sol2)}
\end{alltt}
\begin{verbatim}
## [1] 0.0006800216
\end{verbatim}
\end{kframe}
\end{knitrout}

$\bullet$ \textbf{Example 3} : $f(x) = 2e^x - 2x - 3$; $x_0 = 0.5$

\begin{knitrout}
\definecolor{shadecolor}{rgb}{0.969, 0.969, 0.969}\color{fgcolor}\begin{kframe}
\begin{alltt}
\hldef{f3} \hlkwb{<-} \hlkwa{function}\hldef{(}\hlkwc{x}\hldef{)} \hlnum{2} \hlopt{*} \hlkwd{exp}\hldef{(x)} \hlopt{-} \hlnum{2}\hlopt{*}\hldef{x} \hlopt{-} \hlnum{3}
\hldef{sol3} \hlkwb{<-} \hlkwd{newton_raphson_1}\hldef{(}\hlkwc{func} \hldef{= f3,} \hlkwc{x_0} \hldef{=} \hlnum{0.5}\hldef{,} \hlkwc{iterations} \hldef{=} \hlnum{100}\hldef{)}
\hldef{sol3}
\end{alltt}
\begin{verbatim}
## [1] 0.8576769
\end{verbatim}
\end{kframe}
\end{knitrout}

\begin{knitrout}
\definecolor{shadecolor}{rgb}{0.969, 0.969, 0.969}\color{fgcolor}\begin{kframe}
\begin{alltt}
\hlkwd{f3}\hldef{(sol3)}
\end{alltt}
\begin{verbatim}
## [1] 6.157436e-07
\end{verbatim}
\end{kframe}
\end{knitrout}

$\bullet$ \textbf{Example 4} : $f(x) = - 4x + cos(x) + 2$; $x_0 = 0.5$

\begin{knitrout}
\definecolor{shadecolor}{rgb}{0.969, 0.969, 0.969}\color{fgcolor}\begin{kframe}
\begin{alltt}
\hldef{f4} \hlkwb{<-} \hlkwa{function}\hldef{(}\hlkwc{x}\hldef{)} \hlopt{-}\hlnum{4} \hlopt{*} \hldef{x} \hlopt{+} \hlkwd{cos}\hldef{(x)} \hlopt{+} \hlnum{2}
\hldef{sol4} \hlkwb{<-} \hlkwd{newton_raphson_1}\hldef{(}\hlkwc{func} \hldef{= f4,} \hlkwc{x_0} \hldef{=} \hlnum{0.5}\hldef{,} \hlkwc{iterations} \hldef{=} \hlnum{100}\hldef{)}
\hldef{sol4}
\end{alltt}
\begin{verbatim}
## [1] 0.692426
\end{verbatim}
\end{kframe}
\end{knitrout}

\begin{knitrout}
\definecolor{shadecolor}{rgb}{0.969, 0.969, 0.969}\color{fgcolor}\begin{kframe}
\begin{alltt}
\hlkwd{f4}\hldef{(sol4)}
\end{alltt}
\begin{verbatim}
## [1] -4.67322e-06
\end{verbatim}
\end{kframe}
\end{knitrout}

\end{document}
