\documentclass[11pt, a4paper]{article}\usepackage[]{graphicx}\usepackage[]{xcolor}
% maxwidth is the original width if it is less than linewidth
% otherwise use linewidth (to make sure the graphics do not exceed the margin)
\makeatletter
\def\maxwidth{ %
  \ifdim\Gin@nat@width>\linewidth
    \linewidth
  \else
    \Gin@nat@width
  \fi
}
\makeatother

\definecolor{fgcolor}{rgb}{0.345, 0.345, 0.345}
\newcommand{\hlnum}[1]{\textcolor[rgb]{0.686,0.059,0.569}{#1}}%
\newcommand{\hlsng}[1]{\textcolor[rgb]{0.192,0.494,0.8}{#1}}%
\newcommand{\hlcom}[1]{\textcolor[rgb]{0.678,0.584,0.686}{\textit{#1}}}%
\newcommand{\hlopt}[1]{\textcolor[rgb]{0,0,0}{#1}}%
\newcommand{\hldef}[1]{\textcolor[rgb]{0.345,0.345,0.345}{#1}}%
\newcommand{\hlkwa}[1]{\textcolor[rgb]{0.161,0.373,0.58}{\textbf{#1}}}%
\newcommand{\hlkwb}[1]{\textcolor[rgb]{0.69,0.353,0.396}{#1}}%
\newcommand{\hlkwc}[1]{\textcolor[rgb]{0.333,0.667,0.333}{#1}}%
\newcommand{\hlkwd}[1]{\textcolor[rgb]{0.737,0.353,0.396}{\textbf{#1}}}%
\let\hlipl\hlkwb

\usepackage{framed}
\makeatletter
\newenvironment{kframe}{%
 \def\at@end@of@kframe{}%
 \ifinner\ifhmode%
  \def\at@end@of@kframe{\end{minipage}}%
  \begin{minipage}{\columnwidth}%
 \fi\fi%
 \def\FrameCommand##1{\hskip\@totalleftmargin \hskip-\fboxsep
 \colorbox{shadecolor}{##1}\hskip-\fboxsep
     % There is no \\@totalrightmargin, so:
     \hskip-\linewidth \hskip-\@totalleftmargin \hskip\columnwidth}%
 \MakeFramed {\advance\hsize-\width
   \@totalleftmargin\z@ \linewidth\hsize
   \@setminipage}}%
 {\par\unskip\endMakeFramed%
 \at@end@of@kframe}
\makeatother

\definecolor{shadecolor}{rgb}{.97, .97, .97}
\definecolor{messagecolor}{rgb}{0, 0, 0}
\definecolor{warningcolor}{rgb}{1, 0, 1}
\definecolor{errorcolor}{rgb}{1, 0, 0}
\newenvironment{knitrout}{}{} % an empty environment to be redefined in TeX

\usepackage{alltt}

\usepackage[top = 1 in, bottom = 1 in, left = 1 in, right = 1 in ]{geometry}

\usepackage{amsmath, amssymb, amsfonts}
\usepackage{enumerate}
\usepackage{array}
\usepackage{multirow}
\usepackage{dingbat}
\usepackage{fontawesome5}
\usepackage{tasks}
\usepackage{bbding}
\usepackage{undertilde}
\usepackage{twemojis}
% how to use bull's eye ----- \scalebox{2.0}{\twemoji{bullseye}}
\usepackage{fontspec}
\usepackage{customdice}
% how to put dice face ------ \dice{2}

\title{MSMS 206 : Practical 03}
\author{Ananda Biswas}
\date{\today}

\newfontface\myfont{Myfont1-Regular.ttf}[LetterSpace=0.05em]
% how to use ---- {\setlength{\spaceskip}{1em plus 0.5em minus 0.5em} \fontsize{17}{20}\myfont --write text here-- \par}

\newfontface\cbfont{CaveatBrush-Regular.ttf}
% how to use --- \myfont --write text here--
\IfFileExists{upquote.sty}{\usepackage{upquote}}{}
\begin{document}

\maketitle


\scalebox{2.0}{\twemoji{bullseye}} \hspace{0.2cm} \textcolor{blue}{\textbf{Question}}

\vspace{0.3cm}

\begin{enumerate}[(1)]

\item Calculate $L$ and $U$ such that $A = LU$ where $L$ is a lower triangular matrix and $U$ is an upper triangular matrix for given

\begin{gather*}
A = 
\begin{bmatrix}
1 & 1 & -1 \\
1 & -2 & 3 \\
2 & 3 & 1 \\
\end{bmatrix}.
\end{gather*}


\item Solve the following system of linear equations using $LU$ decomposition method.

\begin{align*}
x_1 + x_2 - x_3 &= 4 \\
x_1 - 2x_2 + 3x_3 &= -6 \\
2x_1 + 3x_2 + x_3 &= 7 \\
\end{align*}

\end{enumerate}


\faArrowAltCircleRight[regular] \hspace{0.2cm} $LU$ decomposition is only possible for non-singular matrices. It is assumed that there is no need for row swapping in the Gaussian elimination.

\begin{knitrout}
\definecolor{shadecolor}{rgb}{0.969, 0.969, 0.969}\color{fgcolor}\begin{kframe}
\begin{alltt}
\hldef{LU_decomposer} \hlkwb{<-} \hlkwa{function}\hldef{(}\hlkwc{A}\hldef{)\{}

  \hlkwa{if}\hldef{(}\hlkwd{det}\hldef{(A)} \hlopt{==} \hlnum{0}\hldef{)} \hlkwd{stop}\hldef{(}\hlsng{"Matrix must be non-singular."}\hldef{)}

  \hldef{I} \hlkwb{<-} \hlkwd{matrix}\hldef{(}\hlkwc{data} \hldef{=} \hlnum{0}\hldef{,} \hlkwc{nrow} \hldef{=} \hlkwd{nrow}\hldef{(A),} \hlkwc{ncol} \hldef{=} \hlkwd{ncol}\hldef{(A))}

  \hlkwa{for} \hldef{(i} \hlkwa{in} \hlnum{1}\hlopt{:}\hlkwd{nrow}\hldef{(I)) \{}
    \hldef{I[i, i]} \hlkwb{<-} \hldef{I[i, i]} \hlopt{+} \hlnum{1}
  \hldef{\}}

  \hldef{r} \hlkwb{<-} \hlkwd{dim}\hldef{(A)[}\hlnum{1}\hldef{]}
  \hldef{c} \hlkwb{<-} \hlkwd{dim}\hldef{(A)[}\hlnum{2}\hldef{]}

  \hldef{i} \hlkwb{<-} \hlnum{1}\hldef{; j} \hlkwb{<-} \hlnum{1}

  \hlkwa{while}\hldef{(j} \hlopt{<=} \hldef{c) \{}

    \hlkwa{while}\hldef{(i} \hlopt{<=} \hldef{r) \{}

      \hlkwa{if}\hldef{(i} \hlopt{!=} \hldef{r)\{}
        \hldef{a1} \hlkwb{<-} \hlkwd{as.matrix}\hldef{(A[(i}\hlopt{+}\hlnum{1}\hldef{)}\hlopt{:}\hldef{r, j]} \hlopt{/} \hldef{A[i, j])}

        \hldef{a2} \hlkwb{<-} \hlkwd{t}\hldef{(}\hlkwd{as.matrix}\hldef{(A[i, ]))}

        \hldef{A[(i}\hlopt{+}\hlnum{1}\hldef{)}\hlopt{:}\hldef{r, ]} \hlkwb{<-} \hldef{A[(i}\hlopt{+}\hlnum{1}\hldef{)}\hlopt{:}\hldef{r, ]} \hlopt{-} \hldef{a1} \hlopt \hldef{a2}

        \hldef{I[(i}\hlopt{+}\hlnum{1}\hldef{)}\hlopt{:}\hldef{r, j]} \hlkwb{<-} \hlkwd{as.vector}\hldef{(a1)}

        \hlkwa{break}
      \hldef{\}}
      \hldef{i} \hlkwb{<-} \hldef{i} \hlopt{+} \hlnum{1}
    \hldef{\}}
    \hldef{j} \hlkwb{<-} \hldef{j} \hlopt{+} \hlnum{1}

    \hldef{i} \hlkwb{<-} \hldef{j}
  \hldef{\}}

  \hlkwd{return}\hldef{(}\hlkwd{list}\hldef{(I, A))}
\hldef{\}}
\end{alltt}
\end{kframe}
\end{knitrout}

\textbf{\textit{LU\_decomposer()}} returns a list containing $L$ and $U$ respectively.

\begin{knitrout}
\definecolor{shadecolor}{rgb}{0.969, 0.969, 0.969}\color{fgcolor}\begin{kframe}
\begin{alltt}
\hldef{A} \hlkwb{<-} \hlkwd{matrix}\hldef{(}\hlkwc{data} \hldef{=} \hlkwd{c}\hldef{(}\hlnum{1}\hldef{,} \hlnum{1}\hldef{,} \hlopt{-}\hlnum{1}\hldef{,}
                     \hlnum{1}\hldef{,} \hlopt{-}\hlnum{2}\hldef{,} \hlnum{3}\hldef{,}
                     \hlnum{2}\hldef{,} \hlnum{3}\hldef{,} \hlnum{1}\hldef{),}
            \hlkwc{nrow} \hldef{=} \hlnum{3}\hldef{,} \hlkwc{ncol} \hldef{=} \hlnum{3}\hldef{,} \hlkwc{byrow} \hldef{=} \hlnum{TRUE}\hldef{)}
\end{alltt}
\end{kframe}
\end{knitrout}

\begin{knitrout}
\definecolor{shadecolor}{rgb}{0.969, 0.969, 0.969}\color{fgcolor}\begin{kframe}
\begin{alltt}
\hlkwd{LU_decomposer}\hldef{(A)}
\end{alltt}
\begin{verbatim}
## [[1]]
##      [,1]       [,2] [,3]
## [1,]    1  0.0000000    0
## [2,]    1  1.0000000    0
## [3,]    2 -0.3333333    1
## 
## [[2]]
##      [,1] [,2]      [,3]
## [1,]    1    1 -1.000000
## [2,]    0   -3  4.000000
## [3,]    0    0  4.333333
\end{verbatim}
\end{kframe}
\end{knitrout}

\begin{knitrout}
\definecolor{shadecolor}{rgb}{0.969, 0.969, 0.969}\color{fgcolor}\begin{kframe}
\begin{alltt}
\hldef{L} \hlkwb{<-} \hlkwd{LU_decomposer}\hldef{(A)[[}\hlnum{1}\hldef{]]}
\hldef{U} \hlkwb{<-} \hlkwd{LU_decomposer}\hldef{(A)[[}\hlnum{2}\hldef{]]}
\end{alltt}
\end{kframe}
\end{knitrout}

\newpage

\faArrowAltCircleRight[regular] \hspace{0.2cm}

\begin{knitrout}
\definecolor{shadecolor}{rgb}{0.969, 0.969, 0.969}\color{fgcolor}\begin{kframe}
\begin{alltt}
\hldef{b} \hlkwb{<-} \hlkwd{matrix}\hldef{(}\hlkwc{data} \hldef{=} \hlkwd{c}\hldef{(}\hlnum{4}\hldef{,} \hlopt{-}\hlnum{6}\hldef{,} \hlnum{7}\hldef{),} \hlkwc{nrow} \hldef{=} \hlnum{3}\hldef{,} \hlkwc{ncol} \hldef{=} \hlnum{1}\hldef{,} \hlkwc{byrow} \hldef{=} \hlnum{TRUE}\hldef{)}
\end{alltt}
\end{kframe}
\end{knitrout}

$\bullet$ 
\begin{align*}
&\hspace{20pt}A\utilde{x} = \utilde{b} \\
&\Rightarrow LU\utilde{x} = \utilde{b} \\
&\Rightarrow \utilde{x} = U^{-1}L^{-1}\utilde{b} \\
\end{align*}

\begin{knitrout}
\definecolor{shadecolor}{rgb}{0.969, 0.969, 0.969}\color{fgcolor}\begin{kframe}
\begin{alltt}
\hlkwd{solve}\hldef{(U)} \hlopt \hlkwd{solve}\hldef{(L)} \hlopt \hldef{b}
\end{alltt}
\begin{verbatim}
##      [,1]
## [1,]    1
## [2,]    2
## [3,]   -1
\end{verbatim}
\end{kframe}
\end{knitrout}

\smallpencil {\setlength{\spaceskip}{1em plus 0.5em minus 0.5em} \fontsize{17}{20}\myfont The solution of the given system of equations is \par}

\begin{gather*}
\begin{bmatrix}
x_1 \\
x_2 \\
x_3 \\
\end{bmatrix}
=
\begin{bmatrix}
1 \\
2 \\
-1 \\
\end{bmatrix}.
\end{gather*}

\end{document}
