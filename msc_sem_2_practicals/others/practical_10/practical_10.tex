\documentclass[11pt, a4paper]{article}\usepackage[]{graphicx}\usepackage[]{xcolor}
% maxwidth is the original width if it is less than linewidth
% otherwise use linewidth (to make sure the graphics do not exceed the margin)
\makeatletter
\def\maxwidth{ %
  \ifdim\Gin@nat@width>\linewidth
    \linewidth
  \else
    \Gin@nat@width
  \fi
}
\makeatother

\definecolor{fgcolor}{rgb}{0.345, 0.345, 0.345}
\newcommand{\hlnum}[1]{\textcolor[rgb]{0.686,0.059,0.569}{#1}}%
\newcommand{\hlsng}[1]{\textcolor[rgb]{0.192,0.494,0.8}{#1}}%
\newcommand{\hlcom}[1]{\textcolor[rgb]{0.678,0.584,0.686}{\textit{#1}}}%
\newcommand{\hlopt}[1]{\textcolor[rgb]{0,0,0}{#1}}%
\newcommand{\hldef}[1]{\textcolor[rgb]{0.345,0.345,0.345}{#1}}%
\newcommand{\hlkwa}[1]{\textcolor[rgb]{0.161,0.373,0.58}{\textbf{#1}}}%
\newcommand{\hlkwb}[1]{\textcolor[rgb]{0.69,0.353,0.396}{#1}}%
\newcommand{\hlkwc}[1]{\textcolor[rgb]{0.333,0.667,0.333}{#1}}%
\newcommand{\hlkwd}[1]{\textcolor[rgb]{0.737,0.353,0.396}{\textbf{#1}}}%
\let\hlipl\hlkwb

\usepackage{framed}
\makeatletter
\newenvironment{kframe}{%
 \def\at@end@of@kframe{}%
 \ifinner\ifhmode%
  \def\at@end@of@kframe{\end{minipage}}%
  \begin{minipage}{\columnwidth}%
 \fi\fi%
 \def\FrameCommand##1{\hskip\@totalleftmargin \hskip-\fboxsep
 \colorbox{shadecolor}{##1}\hskip-\fboxsep
     % There is no \\@totalrightmargin, so:
     \hskip-\linewidth \hskip-\@totalleftmargin \hskip\columnwidth}%
 \MakeFramed {\advance\hsize-\width
   \@totalleftmargin\z@ \linewidth\hsize
   \@setminipage}}%
 {\par\unskip\endMakeFramed%
 \at@end@of@kframe}
\makeatother

\definecolor{shadecolor}{rgb}{.97, .97, .97}
\definecolor{messagecolor}{rgb}{0, 0, 0}
\definecolor{warningcolor}{rgb}{1, 0, 1}
\definecolor{errorcolor}{rgb}{1, 0, 0}
\newenvironment{knitrout}{}{} % an empty environment to be redefined in TeX

\usepackage{alltt}

\usepackage[top = 0.8 in, bottom = 0.8 in, left = 1 in, right = 1 in]{geometry}

\usepackage{amsmath, amssymb, amsfonts}
\usepackage{enumerate}
\usepackage{array}
\usepackage{multirow}
\usepackage{dingbat}
\usepackage{fontawesome5}
\usepackage{tasks}
\usepackage{bbding}
\usepackage{undertilde}
\usepackage{twemojis}
\usepackage{simpsons}
% how to use bull's eye ----- \scalebox{2.0}{\twemoji{bullseye}}
\usepackage{fontspec}
\usepackage{customdice}
% how to put dice face ------ \dice{2}

\title{MSMS 206 : Practical 10}
\author{Ananda Biswas}
\date{\today}

\newfontface\myfont{Myfont1-Regular.ttf}[LetterSpace=0.05em]
% how to use ---- {\setlength{\spaceskip}{1em plus 0.5em minus 0.5em} \fontsize{17}{20}\myfont --write text here-- \par}

\newfontface\cbfont{CaveatBrush-Regular.ttf}
% how to use --- \myfont --write text here--
\IfFileExists{upquote.sty}{\usepackage{upquote}}{}
\begin{document}

\maketitle


\scalebox{2.0}{\twemoji{bullseye}} \hspace{0.2cm} \textcolor{blue}{\textbf{Question : }} In a certain social mobility study, the population under consideration was divided into three income groups: upper, middle and lower. It has been found that 70\% of sons of upper income group parents themselves become upper income, 20\% middle income and 10\% lower income. Of the sons of middle-income parents, 30\% move to upper income group, 50\% remain in middle income group and 20\% become lower. Of the sons of lower income parents, 10\% move to upper income group, 20\% to middle income group and 70\% remain lower. Draw up a matrix to represent these movements.

At a certain point of time, the population is found to have 10\% men in upper income group, 50\% in middle income group and 40\% in lower. Assuming that each man has one son and one grandson, what will be the group composition of grandsons? \\[1.5em]


\faArrowAltCircleRight[regular] \hspace{0.2cm} \underline{Transition Probability Matrix}

\begin{knitrout}
\definecolor{shadecolor}{rgb}{0.969, 0.969, 0.969}\color{fgcolor}\begin{kframe}
\begin{alltt}
\hldef{TPM} \hlkwb{<-} \hlkwd{matrix}\hldef{(}\hlkwd{c}\hldef{(}\hlnum{0.7}\hldef{,} \hlnum{0.2}\hldef{,} \hlnum{0.1}\hldef{,}
                \hlnum{0.3}\hldef{,} \hlnum{0.5}\hldef{,} \hlnum{0.2}\hldef{,}
                \hlnum{0.1}\hldef{,} \hlnum{0.2}\hldef{,} \hlnum{0.7}\hldef{),} \hlkwc{nrow} \hldef{=} \hlnum{3}\hldef{,} \hlkwc{ncol} \hldef{=} \hlnum{3}\hldef{,} \hlkwc{byrow} \hldef{=} \hlnum{TRUE}\hldef{)}

\hldef{states} \hlkwb{<-} \hlkwd{c}\hldef{(}\hlsng{"upper"}\hldef{,} \hlsng{"middle"}\hldef{,} \hlsng{"lower"}\hldef{)}

\hlkwd{rownames}\hldef{(TPM)} \hlkwb{<-} \hldef{states}
\hlkwd{colnames}\hldef{(TPM)} \hlkwb{<-} \hldef{states}
\end{alltt}
\end{kframe}
\end{knitrout}

\smallpencil {\setlength{\spaceskip}{1em plus 0.5em minus 0.5em} \fontsize{17}{20}\myfont The transition probability matrix is \par}
\begin{knitrout}
\definecolor{shadecolor}{rgb}{0.969, 0.969, 0.969}\color{fgcolor}\begin{kframe}
\begin{alltt}
\hldef{TPM}
\end{alltt}
\begin{verbatim}
##        upper middle lower
## upper    0.7    0.2   0.1
## middle   0.3    0.5   0.2
## lower    0.1    0.2   0.7
\end{verbatim}
\end{kframe}
\end{knitrout}

\faArrowAltCircleRight[regular] \hspace{0.2cm} \underline{Future Distribution from Initial Distribution}

\begin{knitrout}
\definecolor{shadecolor}{rgb}{0.969, 0.969, 0.969}\color{fgcolor}\begin{kframe}
\begin{alltt}
\hldef{X_0} \hlkwb{<-} \hlkwd{matrix}\hldef{(}\hlkwd{c}\hldef{(}\hlnum{0.1}\hldef{,} \hlnum{0.5}\hldef{,} \hlnum{0.4}\hldef{),} \hlkwc{nrow} \hldef{=} \hlnum{1}\hldef{)}
\hldef{X_1} \hlkwb{<-} \hldef{X_0} \hlopt \hldef{TPM}
\hldef{X_2} \hlkwb{<-} \hldef{X_1} \hlopt \hldef{TPM}
\end{alltt}
\end{kframe}
\end{knitrout}

\smallpencil {\setlength{\spaceskip}{1em plus 0.5em minus 0.5em} \fontsize{17}{20}\myfont The group composition of grandsons is \par}

\begin{knitrout}
\definecolor{shadecolor}{rgb}{0.969, 0.969, 0.969}\color{fgcolor}\begin{kframe}
\begin{alltt}
\hldef{X_2}
\end{alltt}
\begin{verbatim}
##      upper middle lower
## [1,] 0.326  0.305 0.369
\end{verbatim}
\end{kframe}
\end{knitrout}


\end{document}
