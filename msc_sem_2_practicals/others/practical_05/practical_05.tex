\documentclass[11pt, a4paper]{article}\usepackage[]{graphicx}\usepackage[]{xcolor}
% maxwidth is the original width if it is less than linewidth
% otherwise use linewidth (to make sure the graphics do not exceed the margin)
\makeatletter
\def\maxwidth{ %
  \ifdim\Gin@nat@width>\linewidth
    \linewidth
  \else
    \Gin@nat@width
  \fi
}
\makeatother

\definecolor{fgcolor}{rgb}{0.345, 0.345, 0.345}
\newcommand{\hlnum}[1]{\textcolor[rgb]{0.686,0.059,0.569}{#1}}%
\newcommand{\hlsng}[1]{\textcolor[rgb]{0.192,0.494,0.8}{#1}}%
\newcommand{\hlcom}[1]{\textcolor[rgb]{0.678,0.584,0.686}{\textit{#1}}}%
\newcommand{\hlopt}[1]{\textcolor[rgb]{0,0,0}{#1}}%
\newcommand{\hldef}[1]{\textcolor[rgb]{0.345,0.345,0.345}{#1}}%
\newcommand{\hlkwa}[1]{\textcolor[rgb]{0.161,0.373,0.58}{\textbf{#1}}}%
\newcommand{\hlkwb}[1]{\textcolor[rgb]{0.69,0.353,0.396}{#1}}%
\newcommand{\hlkwc}[1]{\textcolor[rgb]{0.333,0.667,0.333}{#1}}%
\newcommand{\hlkwd}[1]{\textcolor[rgb]{0.737,0.353,0.396}{\textbf{#1}}}%
\let\hlipl\hlkwb

\usepackage{framed}
\makeatletter
\newenvironment{kframe}{%
 \def\at@end@of@kframe{}%
 \ifinner\ifhmode%
  \def\at@end@of@kframe{\end{minipage}}%
  \begin{minipage}{\columnwidth}%
 \fi\fi%
 \def\FrameCommand##1{\hskip\@totalleftmargin \hskip-\fboxsep
 \colorbox{shadecolor}{##1}\hskip-\fboxsep
     % There is no \\@totalrightmargin, so:
     \hskip-\linewidth \hskip-\@totalleftmargin \hskip\columnwidth}%
 \MakeFramed {\advance\hsize-\width
   \@totalleftmargin\z@ \linewidth\hsize
   \@setminipage}}%
 {\par\unskip\endMakeFramed%
 \at@end@of@kframe}
\makeatother

\definecolor{shadecolor}{rgb}{.97, .97, .97}
\definecolor{messagecolor}{rgb}{0, 0, 0}
\definecolor{warningcolor}{rgb}{1, 0, 1}
\definecolor{errorcolor}{rgb}{1, 0, 0}
\newenvironment{knitrout}{}{} % an empty environment to be redefined in TeX

\usepackage{alltt}

\usepackage[top = 1 in, bottom = 1 in, left = 1 in, right = 1 in ]{geometry}

\usepackage{amsmath, amssymb, amsfonts}
\usepackage{enumerate}
\usepackage{array}
\usepackage{multirow}
\usepackage{dingbat}
\usepackage{fontawesome5}
\usepackage{tasks}
\usepackage{bbding}
\usepackage{twemojis}
% how to use bull's eye ----- \scalebox{2.0}{\twemoji{bullseye}}
\usepackage{fontspec}
\usepackage{customdice}
% how to put dice face ------ \dice{2}

\title{MSMS 206 : Practical 05}
\author{Ananda Biswas}
\date{\today}

\newfontface\myfont{Myfont1-Regular.ttf}[LetterSpace=0.05em]
% how to use ---- {\setlength{\spaceskip}{1em plus 0.5em minus 0.5em} \fontsize{17}{20}\myfont --write text here-- \par}

\newfontface\cbfont{CaveatBrush-Regular.ttf}
% how to use --- \myfont --write text here--
\IfFileExists{upquote.sty}{\usepackage{upquote}}{}
\begin{document}

\maketitle


\scalebox{2.0}{\twemoji{bullseye}} \hspace{0.2cm} \textcolor{blue}{\textbf{Question :}} Perform $k-$means clustering for $\{2, 4, 10, 12, 3, 20, 30, 11, 25\}$ for $k = 2$. Assume $2$ and $4$ as initial cluster centroids.

\vspace{1cm}

\faArrowAltCircleRight[regular] \hspace{0.2cm} After a choice of initial centroids, the $k-$means clustering algorithm is as follows :

\begin{enumerate}[(1)]
\item calculate the distance of each data-point from each of the centroids
\item assign each of the data-points to its closest centroid
\item relocate the centroids to the average location of the data-points of similar group
\end{enumerate}

And we repeat this procedure until the assignments don't change after the centroid locations were recomputed.

\begin{knitrout}
\definecolor{shadecolor}{rgb}{0.969, 0.969, 0.969}\color{fgcolor}\begin{kframe}
\begin{alltt}
\hldef{df} \hlkwb{<-} \hlkwd{data.frame}\hldef{(}\hlkwc{x} \hldef{=} \hlkwd{c}\hldef{(}\hlnum{2}\hldef{,} \hlnum{4}\hldef{,} \hlnum{10}\hldef{,} \hlnum{12}\hldef{,} \hlnum{3}\hldef{,} \hlnum{20}\hldef{,} \hlnum{30}\hldef{,} \hlnum{11}\hldef{,} \hlnum{25}\hldef{))}
\end{alltt}
\end{kframe}
\end{knitrout}







Let us have a look at the data-points.

\begin{figure}[!htbp]
\centering
\includegraphics[scale = 0.4]{initial.png}
\end{figure}

Now We put the initial centroids.






\begin{figure}[!htbp]
\centering
\includegraphics[scale = 0.4]{initial_with_centroid.png}
\end{figure}

\begin{knitrout}
\definecolor{shadecolor}{rgb}{0.969, 0.969, 0.969}\color{fgcolor}\begin{kframe}
\begin{alltt}
\hldef{m} \hlkwb{<-} \hlkwd{dim}\hldef{(df)[}\hlnum{1}\hldef{]} \hlcom{# number of data-points}
\hldef{n} \hlkwb{<-} \hlkwd{dim}\hldef{(df)[}\hlnum{2}\hldef{]} \hlcom{# dimension of data-points}

\hldef{k} \hlkwb{<-} \hlnum{2} \hlcom{# number of clusters}
\end{alltt}
\end{kframe}
\end{knitrout}

\begin{knitrout}
\definecolor{shadecolor}{rgb}{0.969, 0.969, 0.969}\color{fgcolor}\begin{kframe}
\begin{alltt}
\hldef{X} \hlkwb{<-} \hlkwd{as.matrix}\hldef{(df)}
\end{alltt}
\end{kframe}
\end{knitrout}

Now we initialize the centroids as 2 and 4.

\begin{knitrout}
\definecolor{shadecolor}{rgb}{0.969, 0.969, 0.969}\color{fgcolor}\begin{kframe}
\begin{alltt}
\hldef{centroid} \hlkwb{<-} \hlkwd{matrix}\hldef{(}\hlkwc{data} \hldef{=} \hlkwd{c}\hldef{(}\hlnum{2}\hldef{,}
                            \hlnum{4}\hldef{),}
                   \hlkwc{nrow} \hldef{= k,} \hlkwc{ncol} \hldef{=} \hlnum{1}\hldef{,} \hlkwc{byrow} \hldef{=} \hlnum{TRUE}\hldef{)}
\end{alltt}
\end{kframe}
\end{knitrout}

We now deploy our $k-$means clustering algorithm. We created a list named $\textit{iteration\_record()}$ for visualization of the process that will come later.

\begin{knitrout}
\definecolor{shadecolor}{rgb}{0.969, 0.969, 0.969}\color{fgcolor}\begin{kframe}
\begin{alltt}
\hldef{cluster} \hlkwb{<-} \hlkwd{c}\hldef{()}

\hldef{iteration_record} \hlkwb{<-} \hlkwd{list}\hldef{()}

\hlkwa{repeat}\hldef{\{}
  \hldef{dist_mat} \hlkwb{<-} \hlkwd{matrix}\hldef{(}\hlnum{0}\hldef{,} \hlkwc{nrow} \hldef{= m,} \hlkwc{ncol} \hldef{= k)}

  \hlkwa{for} \hldef{(i} \hlkwa{in} \hlnum{1}\hlopt{:}\hldef{k) \{}
    \hldef{d} \hlkwb{<-} \hlkwd{apply}\hldef{(X,} \hlnum{1}\hldef{,} \hlkwc{FUN} \hldef{=} \hlkwa{function}\hldef{(}\hlkwc{x}\hldef{)} \hlkwd{return}\hldef{(x} \hlopt{-} \hldef{centroid[i, ]))}

    \hldef{d} \hlkwb{<-} \hlkwd{matrix}\hldef{(d,} \hlkwc{nrow} \hldef{= m,} \hlkwc{ncol} \hldef{= n,} \hlkwc{byrow} \hldef{=} \hlnum{TRUE}\hldef{)}

    \hldef{dist_mat[,i]} \hlkwb{<-} \hlkwd{sqrt}\hldef{(}\hlkwd{diag}\hldef{( d} \hlopt \hlkwd{t}\hldef{(d) ) )}
  \hldef{\}}

  \hldef{cluster} \hlkwb{<-} \hlkwd{apply}\hldef{(dist_mat,} \hlnum{1}\hldef{,} \hlkwc{FUN} \hldef{=} \hlkwa{function}\hldef{(}\hlkwc{x}\hldef{)} \hlkwd{return}\hldef{(}\hlkwd{which}\hldef{(x} \hlopt{==} \hlkwd{min}\hldef{(x))[}\hlnum{1}\hldef{]))}

  \hldef{new_centroid} \hlkwb{<-} \hlkwd{matrix}\hldef{(}\hlkwc{data} \hldef{=} \hlnum{0}\hldef{,} \hlkwc{nrow} \hldef{= k,} \hlkwc{ncol} \hldef{= n)}

  \hlkwa{for} \hldef{(i} \hlkwa{in} \hlnum{1}\hlopt{:}\hldef{k) \{}
    \hldef{new_centroid[i, ]} \hlkwb{<-} \hlkwd{mean}\hldef{(X[}\hlkwd{which}\hldef{(cluster} \hlopt{==} \hldef{i), ])}
  \hldef{\}}

  \hldef{iteration_record} \hlkwb{<-} \hlkwd{append}\hldef{(iteration_record,}
                             \hlkwd{list}\hldef{(}\hlkwd{list}\hldef{(}\hlkwc{mat} \hldef{=} \hlkwd{cbind}\hldef{(X, dist_mat, cluster),}
                                       \hlkwc{new_centroid} \hldef{= new_centroid)))}

  \hlkwa{if}\hldef{(}\hlkwd{any}\hldef{(centroid} \hlopt{-} \hldef{new_centroid} \hlopt{!=} \hlnum{0}\hldef{))\{}
    \hldef{centroid} \hlkwb{<-} \hldef{new_centroid}
  \hldef{\}} \hlkwa{else}\hldef{\{}
    \hlkwa{break}
  \hldef{\}}
\hldef{\}}
\end{alltt}
\end{kframe}
\end{knitrout}

\smallpencil {\setlength{\spaceskip}{1em plus 0.5em minus 0.5em} \fontsize{17}{20}\myfont The final clustering of the data-points is as follows : \par}

\begin{knitrout}
\definecolor{shadecolor}{rgb}{0.969, 0.969, 0.969}\color{fgcolor}\begin{kframe}
\begin{alltt}
\hldef{cluster}
\end{alltt}
\begin{verbatim}
## [1] 1 1 1 1 1 2 2 1 2
\end{verbatim}
\end{kframe}
\end{knitrout}

\newpage

\begin{knitrout}
\definecolor{shadecolor}{rgb}{0.969, 0.969, 0.969}\color{fgcolor}\begin{kframe}
\begin{alltt}
\hlkwd{length}\hldef{(iteration_record)}
\end{alltt}
\begin{verbatim}
## [1] 5
\end{verbatim}
\end{kframe}
\end{knitrout}

There were 5 iterations, we shall take a look at them one by one.

\vspace{1cm}




\leftpointright $\textbf{Iteration 1}$

\begin{knitrout}
\definecolor{shadecolor}{rgb}{0.969, 0.969, 0.969}\color{fgcolor}\begin{kframe}
\begin{alltt}
\hldef{iteration_record[[}\hlnum{1}\hldef{]]}\hlopt{$}\hldef{mat}
\end{alltt}
\begin{verbatim}
##        x distance_from_centroid_1 distance_from_centroid_2 cluster
##  [1,]  2                        0                        2       1
##  [2,]  4                        2                        0       2
##  [3,] 10                        8                        6       2
##  [4,] 12                       10                        8       2
##  [5,]  3                        1                        1       1
##  [6,] 20                       18                       16       2
##  [7,] 30                       28                       26       2
##  [8,] 11                        9                        7       2
##  [9,] 25                       23                       21       2
\end{verbatim}
\end{kframe}
\end{knitrout}





The data-points along with relocated centroids are as follows :

\begin{figure}[!htbp]
\centering
\includegraphics[scale = 0.4]{iter_1.png}
\end{figure}


\leftpointright $\textbf{Iteration 2}$

\begin{knitrout}
\definecolor{shadecolor}{rgb}{0.969, 0.969, 0.969}\color{fgcolor}\begin{kframe}
\begin{alltt}
\hldef{iteration_record[[}\hlnum{2}\hldef{]]}\hlopt{$}\hldef{mat}
\end{alltt}
\begin{verbatim}
##        x distance_from_centroid_1 distance_from_centroid_2 cluster
##  [1,]  2                      0.5                       14       1
##  [2,]  4                      1.5                       12       1
##  [3,] 10                      7.5                        6       2
##  [4,] 12                      9.5                        4       2
##  [5,]  3                      0.5                       13       1
##  [6,] 20                     17.5                        4       2
##  [7,] 30                     27.5                       14       2
##  [8,] 11                      8.5                        5       2
##  [9,] 25                     22.5                        9       2
\end{verbatim}
\end{kframe}
\end{knitrout}





The data-points along with relocated centroids are as follows :

\begin{figure}[!htbp]
\centering
\includegraphics[scale = 0.4]{iter_2.png}
\end{figure}




\newpage

\leftpointright $\textbf{Iteration 3}$

\begin{knitrout}
\definecolor{shadecolor}{rgb}{0.969, 0.969, 0.969}\color{fgcolor}\begin{kframe}
\begin{alltt}
\hldef{iteration_record[[}\hlnum{3}\hldef{]]}\hlopt{$}\hldef{mat}
\end{alltt}
\begin{verbatim}
##        x distance_from_centroid_1 distance_from_centroid_2 cluster
##  [1,]  2                        1                       16       1
##  [2,]  4                        1                       14       1
##  [3,] 10                        7                        8       1
##  [4,] 12                        9                        6       2
##  [5,]  3                        0                       15       1
##  [6,] 20                       17                        2       2
##  [7,] 30                       27                       12       2
##  [8,] 11                        8                        7       2
##  [9,] 25                       22                        7       2
\end{verbatim}
\end{kframe}
\end{knitrout}





The data-points along with relocated centroids are as follows :

\begin{figure}[!htbp]
\centering
\includegraphics[scale = 0.4]{iter_3.png}
\end{figure}





\vspace{1cm}

\leftpointright $\textbf{Iteration 4}$

\begin{knitrout}
\definecolor{shadecolor}{rgb}{0.969, 0.969, 0.969}\color{fgcolor}\begin{kframe}
\begin{alltt}
\hldef{iteration_record[[}\hlnum{4}\hldef{]]}\hlopt{$}\hldef{mat}
\end{alltt}
\begin{verbatim}
##        x distance_from_centroid_1 distance_from_centroid_2 cluster
##  [1,]  2                     2.75                     17.6       1
##  [2,]  4                     0.75                     15.6       1
##  [3,] 10                     5.25                      9.6       1
##  [4,] 12                     7.25                      7.6       1
##  [5,]  3                     1.75                     16.6       1
##  [6,] 20                    15.25                      0.4       2
##  [7,] 30                    25.25                     10.4       2
##  [8,] 11                     6.25                      8.6       1
##  [9,] 25                    20.25                      5.4       2
\end{verbatim}
\end{kframe}
\end{knitrout}





The data-points along with relocated centroids are as follows :

\begin{figure}[!htbp]
\centering
\includegraphics[scale = 0.4]{iter_4.png}
\end{figure}





\newpage

\leftpointright $\textbf{Iteration 5}$

\begin{knitrout}
\definecolor{shadecolor}{rgb}{0.969, 0.969, 0.969}\color{fgcolor}\begin{kframe}
\begin{alltt}
\hldef{iteration_record[[}\hlnum{5}\hldef{]]}\hlopt{$}\hldef{mat}
\end{alltt}
\begin{verbatim}
##        x distance_from_centroid_1 distance_from_centroid_2 cluster
##  [1,]  2                        5                       23       1
##  [2,]  4                        3                       21       1
##  [3,] 10                        3                       15       1
##  [4,] 12                        5                       13       1
##  [5,]  3                        4                       22       1
##  [6,] 20                       13                        5       2
##  [7,] 30                       23                        5       2
##  [8,] 11                        4                       14       1
##  [9,] 25                       18                        0       2
\end{verbatim}
\end{kframe}
\end{knitrout}





The data-points along with relocated centroids are as follows :

\begin{figure}[!htbp]
\centering
\includegraphics[scale = 0.4]{iter_5.png}
\end{figure}

\smallpencil {\setlength{\spaceskip}{1em plus 0.5em minus 0.5em} \fontsize{17}{20}\myfont We notice that there is no change in location centroids from Iteration 4 to Iteration 5. So the process stops and we get our final set of clusters.\par}

\end{document}
