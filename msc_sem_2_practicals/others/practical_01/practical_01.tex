\documentclass[11pt, a4paper]{article}\usepackage[]{graphicx}\usepackage[]{xcolor}
% maxwidth is the original width if it is less than linewidth
% otherwise use linewidth (to make sure the graphics do not exceed the margin)
\makeatletter
\def\maxwidth{ %
  \ifdim\Gin@nat@width>\linewidth
    \linewidth
  \else
    \Gin@nat@width
  \fi
}
\makeatother

\definecolor{fgcolor}{rgb}{0.345, 0.345, 0.345}
\newcommand{\hlnum}[1]{\textcolor[rgb]{0.686,0.059,0.569}{#1}}%
\newcommand{\hlsng}[1]{\textcolor[rgb]{0.192,0.494,0.8}{#1}}%
\newcommand{\hlcom}[1]{\textcolor[rgb]{0.678,0.584,0.686}{\textit{#1}}}%
\newcommand{\hlopt}[1]{\textcolor[rgb]{0,0,0}{#1}}%
\newcommand{\hldef}[1]{\textcolor[rgb]{0.345,0.345,0.345}{#1}}%
\newcommand{\hlkwa}[1]{\textcolor[rgb]{0.161,0.373,0.58}{\textbf{#1}}}%
\newcommand{\hlkwb}[1]{\textcolor[rgb]{0.69,0.353,0.396}{#1}}%
\newcommand{\hlkwc}[1]{\textcolor[rgb]{0.333,0.667,0.333}{#1}}%
\newcommand{\hlkwd}[1]{\textcolor[rgb]{0.737,0.353,0.396}{\textbf{#1}}}%
\let\hlipl\hlkwb

\usepackage{framed}
\makeatletter
\newenvironment{kframe}{%
 \def\at@end@of@kframe{}%
 \ifinner\ifhmode%
  \def\at@end@of@kframe{\end{minipage}}%
  \begin{minipage}{\columnwidth}%
 \fi\fi%
 \def\FrameCommand##1{\hskip\@totalleftmargin \hskip-\fboxsep
 \colorbox{shadecolor}{##1}\hskip-\fboxsep
     % There is no \\@totalrightmargin, so:
     \hskip-\linewidth \hskip-\@totalleftmargin \hskip\columnwidth}%
 \MakeFramed {\advance\hsize-\width
   \@totalleftmargin\z@ \linewidth\hsize
   \@setminipage}}%
 {\par\unskip\endMakeFramed%
 \at@end@of@kframe}
\makeatother

\definecolor{shadecolor}{rgb}{.97, .97, .97}
\definecolor{messagecolor}{rgb}{0, 0, 0}
\definecolor{warningcolor}{rgb}{1, 0, 1}
\definecolor{errorcolor}{rgb}{1, 0, 0}
\newenvironment{knitrout}{}{} % an empty environment to be redefined in TeX

\usepackage{alltt}

\usepackage[top = 1 in, bottom = 1 in, left = 1 in, right = 1 in ]{geometry}

\usepackage{amsmath, amssymb, amsfonts}
\usepackage{enumerate}
\usepackage{array}
\usepackage{multirow}
\usepackage{dingbat}
\usepackage{fontawesome5}
\usepackage{tasks}
\usepackage{bbding}
\usepackage{twemojis}
% how to use bull's eye ----- \scalebox{2.0}{\twemoji{bullseye}}
\usepackage{fontspec}
\usepackage{customdice}
% how to put dice face ------ \dice{2}

\title{MSMS 206 : Practical 01}
\author{Ananda Biswas}
\date{\today}

\newfontface\myfont{Myfont1-Regular.ttf}[LetterSpace=0.05em]
% how to use ---- {\setlength{\spaceskip}{1em plus 0.5em minus 0.5em} \fontsize{17}{20}\myfont --write text here-- \par}

\newfontface\cbfont{CaveatBrush-Regular.ttf}
% how to use --- \myfont --write text here--
\IfFileExists{upquote.sty}{\usepackage{upquote}}{}
\begin{document}

\maketitle


\scalebox{2.0}{\twemoji{bullseye}} \hspace{0.2cm} \textcolor{blue}{\textbf{Question}}

\vspace{0.3cm}

Fit a multiple linear regression model for the following data-set and obtain the following results.

\begin{enumerate}[(a)]

\item estimate of the regression coefficients and $\sigma$,
\item confidence interval of the regression coefficients,
\item coefficient of determination,
\item adjusted coefficient of determination.
\end{enumerate}

\begin{table}[!htbp]
\def\arraystretch{1}

\begin{center}
\begin{tabular}{|>{\centering}m{2.5cm}|>{\centering}m{2.5cm}|>{\centering\arraybackslash}m{2.5cm}|}

\hline

$y$ & $x_1$ & $x_2$ \\

\hline

16.68 &	7 & 	560 \\
11.5 &	3 & 	220 \\
12.03 &	3 & 	340 \\
14.88 &	4 & 	80 \\
13.75 &	6 & 	150 \\
18.11 &	7 & 	330 \\
8 &	2 & 	110 \\
17.83 &	7 & 	210 \\
79.2 &	30 & 	1460 \\
21 &	10 & 	215 \\
13.5 &	4 & 	255 \\
19.75 &	6 & 	462 \\
24 &	9 & 	448 \\
29 &	10 & 	776 \\
15.35 &	6 & 	220 \\
19 &	7 & 	132 \\
9.5 &	 3 &	36 \\
35.1 &	17 & 	770 \\
17.9 &	10 & 	140 \\
52.32 &	26 & 	810 \\
18.75 &	9 & 	450 \\
19.83 &	8 & 	635 \\
10.75 &	4 & 	150 \\
21.5 &	5 & 	605 \\
40.33 &	16 & 	688 \\

\hline

\end{tabular}
\end{center}
\end{table}

\newpage

\faArrowAltCircleRight[regular] \hspace{0.2cm} We fit a multiple linear regression model $Y = \beta_0 + \beta_1 X_1 + \beta_2 X_2$.

\begin{knitrout}\tiny
\definecolor{shadecolor}{rgb}{0.969, 0.969, 0.969}\color{fgcolor}\begin{kframe}
\begin{alltt}
\hldef{raw_data} \hlkwb{<-} \hlkwd{read.csv}\hldef{(}\hlsng{'https://raw.githubusercontent.com/sakunisgithub/data_sets/refs/heads/master/msc_semester_2/tripti_madam_practical_01_data.csv'}\hldef{)}
\end{alltt}
\end{kframe}
\end{knitrout}

\begin{knitrout}
\definecolor{shadecolor}{rgb}{0.969, 0.969, 0.969}\color{fgcolor}\begin{kframe}
\begin{alltt}
\hlkwd{dim}\hldef{(raw_data)}
\end{alltt}
\begin{verbatim}
## [1] 25  3
\end{verbatim}
\end{kframe}
\end{knitrout}

\begin{knitrout}
\definecolor{shadecolor}{rgb}{0.969, 0.969, 0.969}\color{fgcolor}\begin{kframe}
\begin{alltt}
\hlkwd{names}\hldef{(raw_data)}
\end{alltt}
\begin{verbatim}
## [1] "Y"   "X_1" "X_2"
\end{verbatim}
\end{kframe}
\end{knitrout}

\begin{knitrout}
\definecolor{shadecolor}{rgb}{0.969, 0.969, 0.969}\color{fgcolor}\begin{kframe}
\begin{alltt}
\hldef{model1} \hlkwb{<-} \hlkwd{lm}\hldef{(Y} \hlopt{~} \hldef{X_1} \hlopt{+} \hldef{X_2,} \hlkwc{data} \hldef{= raw_data)}
\hldef{model_summary} \hlkwb{<-} \hlkwd{summary}\hldef{(model1)}
\end{alltt}
\end{kframe}
\end{knitrout}

\begin{knitrout}
\definecolor{shadecolor}{rgb}{0.969, 0.969, 0.969}\color{fgcolor}\begin{kframe}
\begin{alltt}
\hldef{model_summary}
\end{alltt}
\begin{verbatim}
## 
## Call:
## lm(formula = Y ~ X_1 + X_2, data = raw_data)
## 
## Residuals:
##    Min     1Q Median     3Q    Max 
## -5.776 -0.659  0.164  1.173  7.387 
## 
## Coefficients:
##             Estimate Std. Error t value Pr(>|t|)    
## (Intercept) 2.326534   1.096063   2.123 0.045279 *  
## X_1         1.614726   0.170574   9.466 3.24e-09 ***
## X_2         0.014414   0.003615   3.987 0.000623 ***
## ---
## Signif. codes:  0 '***' 0.001 '**' 0.01 '*' 0.05 '.' 0.1 ' ' 1
## 
## Residual standard error: 3.254 on 22 degrees of freedom
## Multiple R-squared:  0.9597,	Adjusted R-squared:  0.956 
## F-statistic: 261.9 on 2 and 22 DF,  p-value: 4.561e-16
\end{verbatim}
\end{kframe}
\end{knitrout}

\smallpencil {\setlength{\spaceskip}{1em plus 0.5em minus 0.5em} \fontsize{17}{20}\myfont Estimates of regression coefficients are \par}

\begin{knitrout}
\definecolor{shadecolor}{rgb}{0.969, 0.969, 0.969}\color{fgcolor}\begin{kframe}
\begin{alltt}
\hldef{estimates} \hlkwb{<-} \hldef{model_summary}\hlopt{$}\hldef{coefficients[,} \hlsng{'Estimate'}\hldef{]}
\hldef{estimates}
\end{alltt}
\begin{verbatim}
## (Intercept)         X_1         X_2 
##  2.32653410  1.61472628  0.01441393
\end{verbatim}
\end{kframe}
\end{knitrout}

\newpage

\smallpencil {\setlength{\spaceskip}{1em plus 0.5em minus 0.5em} \fontsize{17}{20}\myfont Estimate of error standard deviation $\sigma$ is \par}

\begin{knitrout}
\definecolor{shadecolor}{rgb}{0.969, 0.969, 0.969}\color{fgcolor}\begin{kframe}
\begin{alltt}
\hldef{sigma_hat} \hlkwb{<-} \hldef{model_summary}\hlopt{$}\hldef{sigma}
\hldef{sigma_hat}
\end{alltt}
\begin{verbatim}
## [1] 3.254133
\end{verbatim}
\end{kframe}
\end{knitrout}

Now we shall calculate the confidence intervals of the regression coefficients and the intercept. $100(1 - \alpha)\%$ confidence interval for $\beta_j$ $\forall j = 0(1)2$ is given by 

$$ \left( \hat{\beta_j} - t_{\frac{\alpha}{2}, n - p} \cdot \text{se}(\hat{\beta_j}), \hat{\beta_j} + t_{\frac{\alpha}{2}, n - p} \cdot \text{se}(\hat{\beta_j}) \right).$$

where $n$ is total number of observations and $p$ is total number of parameters in the model.

\begin{knitrout}
\definecolor{shadecolor}{rgb}{0.969, 0.969, 0.969}\color{fgcolor}\begin{kframe}
\begin{alltt}
\hldef{std_errors} \hlkwb{<-} \hldef{model_summary}\hlopt{$}\hldef{coefficients[,} \hlsng{'Std. Error'}\hldef{]}

\hldef{t_tabulated} \hlkwb{<-} \hlkwd{qt}\hldef{(}\hlnum{0.025}\hldef{,} \hlnum{22}\hldef{,} \hlkwc{lower.tail} \hldef{=} \hlnum{FALSE}\hldef{)}

\hldef{CI_lower} \hlkwb{<-} \hldef{estimates} \hlopt{-} \hldef{t_tabulated} \hlopt{*} \hldef{std_errors}
\hldef{CI_upper} \hlkwb{<-} \hldef{estimates} \hlopt{+} \hldef{t_tabulated} \hlopt{*} \hldef{std_errors}
\end{alltt}
\end{kframe}
\end{knitrout}

\smallpencil {\setlength{\spaceskip}{1em plus 0.5em minus 0.5em} \fontsize{17}{20}\myfont 95\% confidence interval for $\beta_0$ is \par}

\begin{knitrout}
\definecolor{shadecolor}{rgb}{0.969, 0.969, 0.969}\color{fgcolor}\begin{kframe}
\begin{alltt}
\hldef{CI_lower[}\hlnum{1}\hldef{]; CI_upper[}\hlnum{1}\hldef{]}
\end{alltt}
\begin{verbatim}
## (Intercept) 
##   0.0534379
## (Intercept) 
##     4.59963
\end{verbatim}
\end{kframe}
\end{knitrout}

\smallpencil {\setlength{\spaceskip}{1em plus 0.5em minus 0.5em} \fontsize{17}{20}\myfont 95\% confidence interval for $\beta_1$ is \par}

\begin{knitrout}
\definecolor{shadecolor}{rgb}{0.969, 0.969, 0.969}\color{fgcolor}\begin{kframe}
\begin{alltt}
\hldef{CI_lower[}\hlnum{2}\hldef{]; CI_upper[}\hlnum{2}\hldef{]}
\end{alltt}
\begin{verbatim}
##      X_1 
## 1.260978
##      X_1 
## 1.968475
\end{verbatim}
\end{kframe}
\end{knitrout}

\smallpencil {\setlength{\spaceskip}{1em plus 0.5em minus 0.5em} \fontsize{17}{20}\myfont 95\% confidence interval for $\beta_2$ is \par}

\begin{knitrout}
\definecolor{shadecolor}{rgb}{0.969, 0.969, 0.969}\color{fgcolor}\begin{kframe}
\begin{alltt}
\hldef{CI_lower[}\hlnum{3}\hldef{]; CI_upper[}\hlnum{3}\hldef{]}
\end{alltt}
\begin{verbatim}
##         X_2 
## 0.006916106
##        X_2 
## 0.02191175
\end{verbatim}
\end{kframe}
\end{knitrout}

\newpage

Now we shall calculate the confidence interval of the error variance $\sigma^2$. $100(1 - \alpha)\%$ confidence interval for $\sigma^2$ is given by 

$$\left( \dfrac{\text{RSS}}{\chi^2_{\frac{\alpha}{2}, n - p}}, \dfrac{\text{RSS}}{\chi^2_{1 - \frac{\alpha}{2}, n - p}} \right).$$

\begin{knitrout}
\definecolor{shadecolor}{rgb}{0.969, 0.969, 0.969}\color{fgcolor}\begin{kframe}
\begin{alltt}
\hldef{RSS} \hlkwb{<-} \hlkwd{sum}\hldef{(model_summary}\hlopt{$}\hldef{residuals}\hlopt{^}\hlnum{2}\hldef{)} \hlcom{# Residual Sum of Squares}
\end{alltt}
\end{kframe}
\end{knitrout}

\smallpencil {\setlength{\spaceskip}{1em plus 0.5em minus 0.5em} \fontsize{17}{20}\myfont 95\% confidence interval for $\sigma^2$ is \par}

\begin{knitrout}
\definecolor{shadecolor}{rgb}{0.969, 0.969, 0.969}\color{fgcolor}\begin{kframe}
\begin{alltt}
\hldef{RSS} \hlopt{/} \hlkwd{qchisq}\hldef{(}\hlnum{0.025}\hldef{,} \hlnum{22}\hldef{,} \hlkwc{lower.tail} \hldef{=} \hlnum{FALSE}\hldef{)} \hlcom{# lower bound}
\end{alltt}
\begin{verbatim}
## [1] 6.333929
\end{verbatim}
\begin{alltt}
\hldef{RSS} \hlopt{/} \hlkwd{qchisq}\hldef{(}\hlnum{1}\hlopt{-}\hlnum{0.025}\hldef{,} \hlnum{22}\hldef{,} \hlkwc{lower.tail} \hldef{=} \hlnum{FALSE}\hldef{)} \hlcom{# upper bound}
\end{alltt}
\begin{verbatim}
## [1] 21.21286
\end{verbatim}
\end{kframe}
\end{knitrout}

\smallpencil {\setlength{\spaceskip}{1em plus 0.5em minus 0.5em} \fontsize{17}{20}\myfont 95\% confidence interval for $\sigma$ is \par}

\begin{knitrout}
\definecolor{shadecolor}{rgb}{0.969, 0.969, 0.969}\color{fgcolor}\begin{kframe}
\begin{alltt}
\hlkwd{sqrt}\hldef{(RSS} \hlopt{/} \hlkwd{qchisq}\hldef{(}\hlnum{0.025}\hldef{,} \hlnum{22}\hldef{,} \hlkwc{lower.tail} \hldef{=} \hlnum{FALSE}\hldef{))} \hlcom{# lower bound}
\end{alltt}
\begin{verbatim}
## [1] 2.51673
\end{verbatim}
\begin{alltt}
\hlkwd{sqrt}\hldef{(RSS} \hlopt{/} \hlkwd{qchisq}\hldef{(}\hlnum{1}\hlopt{-}\hlnum{0.025}\hldef{,} \hlnum{22}\hldef{,} \hlkwc{lower.tail} \hldef{=} \hlnum{FALSE}\hldef{))} \hlcom{# upper bound}
\end{alltt}
\begin{verbatim}
## [1] 4.605742
\end{verbatim}
\end{kframe}
\end{knitrout}

\smallpencil {\setlength{\spaceskip}{1em plus 0.5em minus 0.5em} \fontsize{17}{20}\myfont $R^2$ for the model is \par}

\begin{knitrout}
\definecolor{shadecolor}{rgb}{0.969, 0.969, 0.969}\color{fgcolor}\begin{kframe}
\begin{alltt}
\hldef{model_summary}\hlopt{$}\hldef{r.squared}
\end{alltt}
\begin{verbatim}
## [1] 0.9596944
\end{verbatim}
\end{kframe}
\end{knitrout}

\smallpencil {\setlength{\spaceskip}{1em plus 0.5em minus 0.5em} \fontsize{17}{20}\myfont Adjusted $R^2$ for the model is \par}

\begin{knitrout}
\definecolor{shadecolor}{rgb}{0.969, 0.969, 0.969}\color{fgcolor}\begin{kframe}
\begin{alltt}
\hldef{model_summary}\hlopt{$}\hldef{adj.r.squared}
\end{alltt}
\begin{verbatim}
## [1] 0.9560302
\end{verbatim}
\end{kframe}
\end{knitrout}



\end{document}
