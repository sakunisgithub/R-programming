\documentclass[11pt, a4paper]{article}\usepackage[]{graphicx}\usepackage[]{xcolor}
% maxwidth is the original width if it is less than linewidth
% otherwise use linewidth (to make sure the graphics do not exceed the margin)
\makeatletter
\def\maxwidth{ %
  \ifdim\Gin@nat@width>\linewidth
    \linewidth
  \else
    \Gin@nat@width
  \fi
}
\makeatother

\definecolor{fgcolor}{rgb}{0.345, 0.345, 0.345}
\newcommand{\hlnum}[1]{\textcolor[rgb]{0.686,0.059,0.569}{#1}}%
\newcommand{\hlsng}[1]{\textcolor[rgb]{0.192,0.494,0.8}{#1}}%
\newcommand{\hlcom}[1]{\textcolor[rgb]{0.678,0.584,0.686}{\textit{#1}}}%
\newcommand{\hlopt}[1]{\textcolor[rgb]{0,0,0}{#1}}%
\newcommand{\hldef}[1]{\textcolor[rgb]{0.345,0.345,0.345}{#1}}%
\newcommand{\hlkwa}[1]{\textcolor[rgb]{0.161,0.373,0.58}{\textbf{#1}}}%
\newcommand{\hlkwb}[1]{\textcolor[rgb]{0.69,0.353,0.396}{#1}}%
\newcommand{\hlkwc}[1]{\textcolor[rgb]{0.333,0.667,0.333}{#1}}%
\newcommand{\hlkwd}[1]{\textcolor[rgb]{0.737,0.353,0.396}{\textbf{#1}}}%
\let\hlipl\hlkwb

\usepackage{framed}
\makeatletter
\newenvironment{kframe}{%
 \def\at@end@of@kframe{}%
 \ifinner\ifhmode%
  \def\at@end@of@kframe{\end{minipage}}%
  \begin{minipage}{\columnwidth}%
 \fi\fi%
 \def\FrameCommand##1{\hskip\@totalleftmargin \hskip-\fboxsep
 \colorbox{shadecolor}{##1}\hskip-\fboxsep
     % There is no \\@totalrightmargin, so:
     \hskip-\linewidth \hskip-\@totalleftmargin \hskip\columnwidth}%
 \MakeFramed {\advance\hsize-\width
   \@totalleftmargin\z@ \linewidth\hsize
   \@setminipage}}%
 {\par\unskip\endMakeFramed%
 \at@end@of@kframe}
\makeatother

\definecolor{shadecolor}{rgb}{.97, .97, .97}
\definecolor{messagecolor}{rgb}{0, 0, 0}
\definecolor{warningcolor}{rgb}{1, 0, 1}
\definecolor{errorcolor}{rgb}{1, 0, 0}
\newenvironment{knitrout}{}{} % an empty environment to be redefined in TeX

\usepackage{alltt}

\usepackage[top = 0.6 in, bottom = 0.6 in, left = 0.8 in, right = 0.8 in]{geometry}
\usepackage{amsmath, amssymb, amsfonts}

\allowdisplaybreaks[4]

\usepackage{enumerate}
\usepackage{array}
\usepackage{multirow}
\usepackage{dingbat}
\usepackage{fontawesome5}
\usepackage{tasks}
\usepackage{bbding}
\usepackage{twemojis}
% how to use bull's eye ----- \scalebox{2.0}{\twemoji{bullseye}}
\usepackage{fontspec}


\title{MSMS 408 : Practical 01}
\author{Ananda Biswas \\[1em] Exam Roll No. : 24419STC053}
\date{\today}

\newfontface\myfont{Myfont1-Regular.ttf}[LetterSpace=0.05em]

\newfontface\cbfont{CaveatBrush-Regular.ttf}
% how to use --- \myfont --write text here--
\IfFileExists{upquote.sty}{\usepackage{upquote}}{}
\begin{document}

\maketitle


\section*{\faArrowAltCircleRight[regular] \textcolor{blue}{Question}}

\settasks{label = (\roman*),
          label-width = 4ex,
          item-indent = 8em}

\hspace{1cm} Use Monte Carlo Integration to approximate the following integrals
\begin{tasks}(2)
\task $\displaystyle{\int \limits_0^1 e^{e^x} dx}$,
\task $\displaystyle{\int \limits_{-2}^{2} e^{x + x^2} dx}$
\end{tasks}

and check how the estimates converge to the true value as the sample size increases.

\section*{\faArrowAltCircleRight[regular] \textcolor{blue}{Theory}}

Suppose we have to compute

\begin{equation}
I = \int \limits_a^b f(x) dx.
\end{equation}

We rewrite

\begin{equation}
I = (b-a) E[f(X)]
\end{equation}

where $X \sim U(a, b)$. \\[0.2em]

To approximate $I$, we generate random numbers $X_1, X_2, \ldots, X_n$ from $U(a, b)$ and calculate

\begin{align}
I \approx (b-a) \cdot \dfrac{1}{n} \sum \limits_{i=1}^{n} f(X_i).
\end{align}

By law of large numbers, $\dfrac{1}{n} \sum \limits_{i=1}^{n} f(X_i) \rightarrow E[f(X)]$ as $n \rightarrow \infty$.

\section*{\faArrowAltCircleRight[regular] \textcolor{blue}{R Program}}

\begin{knitrout}
\definecolor{shadecolor}{rgb}{0.969, 0.969, 0.969}\color{fgcolor}\begin{kframe}
\begin{alltt}
\hlkwd{set.seed}\hldef{(}\hlnum{22}\hldef{)}
\end{alltt}
\end{kframe}
\end{knitrout}

\begin{knitrout}
\definecolor{shadecolor}{rgb}{0.969, 0.969, 0.969}\color{fgcolor}\begin{kframe}
\begin{alltt}
\hldef{Monte.Carlo.Integration} \hlkwb{<-} \hlkwa{function}\hldef{(}\hlkwc{integrand}\hldef{,} \hlkwc{lower.limit}\hldef{,} \hlkwc{upper.limit}\hldef{,} \hlkwc{sample.size}\hldef{)\{}

  \hldef{x} \hlkwb{<-} \hlkwd{runif}\hldef{(}\hlkwc{n} \hldef{= sample.size,} \hlkwc{min} \hldef{= lower.limit,} \hlkwc{max} \hldef{= upper.limit)}

  \hldef{I} \hlkwb{<-} \hldef{(upper.limit} \hlopt{-} \hldef{lower.limit)} \hlopt{*} \hlkwd{mean}\hldef{(}\hlkwd{integrand}\hldef{(x))}

  \hlkwd{return}\hldef{(I)}
\hldef{\}}
\end{alltt}
\end{kframe}
\end{knitrout}

\newpage

\scalebox{2.0}{\twemoji{keycap: 1}} \hspace{0.1cm} $\displaystyle{\int \limits_0^1 e^{e^x} dx}$

\begin{knitrout}
\definecolor{shadecolor}{rgb}{0.969, 0.969, 0.969}\color{fgcolor}\begin{kframe}
\begin{alltt}
\hldef{func_1} \hlkwb{<-} \hlkwa{function}\hldef{(}\hlkwc{x}\hldef{)} \hlkwd{exp}\hldef{(}\hlkwd{exp}\hldef{(x))}
\end{alltt}
\end{kframe}
\end{knitrout}

\begin{knitrout}
\definecolor{shadecolor}{rgb}{0.969, 0.969, 0.969}\color{fgcolor}\begin{kframe}
\begin{alltt}
\hldef{sample_sizes} \hlkwb{<-} \hlkwd{c}\hldef{(}\hlnum{100}\hldef{,} \hlnum{500}\hldef{,} \hlnum{700}\hldef{,} \hlnum{1000}\hldef{,} \hlnum{3000}\hldef{,} \hlnum{5000}\hldef{)}
\end{alltt}
\end{kframe}
\end{knitrout}

\begin{knitrout}
\definecolor{shadecolor}{rgb}{0.969, 0.969, 0.969}\color{fgcolor}\begin{kframe}
\begin{alltt}
\hldef{results.1} \hlkwb{<-} \hlkwd{c}\hldef{()}

\hlkwa{for} \hldef{(size} \hlkwa{in} \hldef{sample_sizes) \{}
  \hldef{results.1} \hlkwb{<-} \hlkwd{append}\hldef{(results.1,}
                      \hlkwd{Monte.Carlo.Integration}\hldef{(}\hlkwc{integrand} \hldef{= func_1,} \hlkwc{sample.size} \hldef{= size,}
                                              \hlkwc{lower.limit} \hldef{=} \hlnum{0}\hldef{,} \hlkwc{upper.limit} \hldef{=} \hlnum{1}\hldef{))}
\hldef{\}}
\end{alltt}
\end{kframe}
\end{knitrout}

\begin{knitrout}
\definecolor{shadecolor}{rgb}{0.969, 0.969, 0.969}\color{fgcolor}\begin{kframe}
\begin{alltt}
\hldef{df1} \hlkwb{<-} \hlkwd{data.frame}\hldef{(}\hlkwc{sample.size} \hldef{= sample_sizes,}
                  \hlkwc{approximate.integral.value} \hldef{= results.1)}
\end{alltt}
\end{kframe}
\end{knitrout}



\begin{kframe}
\begin{alltt}
\hlkwd{stargazer}\hldef{(df1,} \hlkwc{summary} \hldef{=} \hlnum{FALSE}\hldef{,} \hlkwc{rownames} \hldef{=} \hlnum{FALSE}\hldef{,} \hlkwc{label} \hldef{=} \hlsng{"Table 1"}\hldef{)}
\end{alltt}
\end{kframe}
% Table created by stargazer v.5.2.3 by Marek Hlavac, Social Policy Institute. E-mail: marek.hlavac at gmail.com
% Date and time: Mon, Feb 09, 2026 - 21:32:13
\begin{table}[!htbp] \centering 
  \caption{} 
  \label{Table 1} 
\begin{tabular}{@{\extracolsep{5pt}} cc} 
\\[-1.8ex]\hline 
\hline \\[-1.8ex] 
sample.size & approximate.integral.value \\ 
\hline \\[-1.8ex] 
$100$ & $6.006$ \\ 
$500$ & $6.015$ \\ 
$700$ & $6.322$ \\ 
$1,000$ & $6.410$ \\ 
$3,000$ & $6.246$ \\ 
$5,000$ & $6.316$ \\ 
\hline \\[-1.8ex] 
\end{tabular} 
\end{table} 


\newpage

\scalebox{2.0}{\twemoji{keycap: 2}} \hspace{0.1cm} $\displaystyle{\int \limits_{-2}^{2} e^{x + x^2} dx}$

\begin{knitrout}
\definecolor{shadecolor}{rgb}{0.969, 0.969, 0.969}\color{fgcolor}\begin{kframe}
\begin{alltt}
\hldef{func_2} \hlkwb{<-} \hlkwa{function}\hldef{(}\hlkwc{x}\hldef{)} \hlkwd{exp}\hldef{(x} \hlopt{+} \hldef{x}\hlopt{^}\hlnum{2}\hldef{)}
\end{alltt}
\end{kframe}
\end{knitrout}

\begin{knitrout}
\definecolor{shadecolor}{rgb}{0.969, 0.969, 0.969}\color{fgcolor}\begin{kframe}
\begin{alltt}
\hldef{results.2} \hlkwb{<-} \hlkwd{c}\hldef{()}

\hlkwa{for} \hldef{(size} \hlkwa{in} \hldef{sample_sizes) \{}
  \hldef{results.2} \hlkwb{<-} \hlkwd{append}\hldef{(results.2,}
                      \hlkwd{Monte.Carlo.Integration}\hldef{(}\hlkwc{integrand} \hldef{= func_2,} \hlkwc{sample.size} \hldef{= size,}
                                              \hlkwc{lower.limit} \hldef{=} \hlopt{-}\hlnum{2}\hldef{,} \hlkwc{upper.limit} \hldef{=} \hlnum{2}\hldef{))}
\hldef{\}}
\end{alltt}
\end{kframe}
\end{knitrout}

\begin{knitrout}
\definecolor{shadecolor}{rgb}{0.969, 0.969, 0.969}\color{fgcolor}\begin{kframe}
\begin{alltt}
\hldef{df2} \hlkwb{<-} \hlkwd{data.frame}\hldef{(}\hlkwc{sample.size} \hldef{= sample_sizes,}
                  \hlkwc{approximate.integral.value} \hldef{= results.2)}
\end{alltt}
\end{kframe}
\end{knitrout}

\begin{kframe}
\begin{alltt}
\hlkwd{stargazer}\hldef{(df2,} \hlkwc{summary} \hldef{=} \hlnum{FALSE}\hldef{,} \hlkwc{rownames} \hldef{=} \hlnum{FALSE}\hldef{,} \hlkwc{label} \hldef{=} \hlsng{"Table 2"}\hldef{)}
\end{alltt}
\end{kframe}
% Table created by stargazer v.5.2.3 by Marek Hlavac, Social Policy Institute. E-mail: marek.hlavac at gmail.com
% Date and time: Mon, Feb 09, 2026 - 21:32:13
\begin{table}[!htbp] \centering 
  \caption{} 
  \label{Table 2} 
\begin{tabular}{@{\extracolsep{5pt}} cc} 
\\[-1.8ex]\hline 
\hline \\[-1.8ex] 
sample.size & approximate.integral.value \\ 
\hline \\[-1.8ex] 
$100$ & $107.312$ \\ 
$500$ & $103.273$ \\ 
$700$ & $93.557$ \\ 
$1,000$ & $98.880$ \\ 
$3,000$ & $95.832$ \\ 
$5,000$ & $95.987$ \\ 
\hline \\[-1.8ex] 
\end{tabular} 
\end{table} 


\section*{\faArrowAltCircleRight[regular] \textcolor{blue}{Conclusion}}

\smallpencil \hspace{0.1cm} {\setlength{\spaceskip}{1em plus 0.5em minus 0.5em} \fontsize{17}{20}\myfont As sample size increases, the approximates gradually converge to the true integral value. \par}

\end{document}
