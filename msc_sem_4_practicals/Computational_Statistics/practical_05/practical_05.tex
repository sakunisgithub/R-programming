\documentclass[11pt, a4paper]{article}\usepackage[]{graphicx}\usepackage[]{xcolor}
% maxwidth is the original width if it is less than linewidth
% otherwise use linewidth (to make sure the graphics do not exceed the margin)
\makeatletter
\def\maxwidth{ %
  \ifdim\Gin@nat@width>\linewidth
    \linewidth
  \else
    \Gin@nat@width
  \fi
}
\makeatother

\definecolor{fgcolor}{rgb}{0.345, 0.345, 0.345}
\newcommand{\hlnum}[1]{\textcolor[rgb]{0.686,0.059,0.569}{#1}}%
\newcommand{\hlsng}[1]{\textcolor[rgb]{0.192,0.494,0.8}{#1}}%
\newcommand{\hlcom}[1]{\textcolor[rgb]{0.678,0.584,0.686}{\textit{#1}}}%
\newcommand{\hlopt}[1]{\textcolor[rgb]{0,0,0}{#1}}%
\newcommand{\hldef}[1]{\textcolor[rgb]{0.345,0.345,0.345}{#1}}%
\newcommand{\hlkwa}[1]{\textcolor[rgb]{0.161,0.373,0.58}{\textbf{#1}}}%
\newcommand{\hlkwb}[1]{\textcolor[rgb]{0.69,0.353,0.396}{#1}}%
\newcommand{\hlkwc}[1]{\textcolor[rgb]{0.333,0.667,0.333}{#1}}%
\newcommand{\hlkwd}[1]{\textcolor[rgb]{0.737,0.353,0.396}{\textbf{#1}}}%
\let\hlipl\hlkwb

\usepackage{framed}
\makeatletter
\newenvironment{kframe}{%
 \def\at@end@of@kframe{}%
 \ifinner\ifhmode%
  \def\at@end@of@kframe{\end{minipage}}%
  \begin{minipage}{\columnwidth}%
 \fi\fi%
 \def\FrameCommand##1{\hskip\@totalleftmargin \hskip-\fboxsep
 \colorbox{shadecolor}{##1}\hskip-\fboxsep
     % There is no \\@totalrightmargin, so:
     \hskip-\linewidth \hskip-\@totalleftmargin \hskip\columnwidth}%
 \MakeFramed {\advance\hsize-\width
   \@totalleftmargin\z@ \linewidth\hsize
   \@setminipage}}%
 {\par\unskip\endMakeFramed%
 \at@end@of@kframe}
\makeatother

\definecolor{shadecolor}{rgb}{.97, .97, .97}
\definecolor{messagecolor}{rgb}{0, 0, 0}
\definecolor{warningcolor}{rgb}{1, 0, 1}
\definecolor{errorcolor}{rgb}{1, 0, 0}
\newenvironment{knitrout}{}{} % an empty environment to be redefined in TeX

\usepackage{alltt}

\usepackage[top = 0.6 in, bottom = 0.6 in, left = 0.8 in, right = 0.8 in]{geometry}
\usepackage{amsmath, amssymb, amsfonts}

\allowdisplaybreaks[4]

\usepackage{enumerate}
\usepackage{array}
\usepackage{multirow}
\usepackage{dingbat}
\usepackage{fontawesome5}
\usepackage{tasks}
\usepackage{bbding}
\usepackage{twemojis}
% how to use bull's eye ----- \scalebox{2.0}{\twemoji{bullseye}}
\usepackage{fontspec}


\title{MSMS 408 : Practical 05}
\author{Ananda Biswas \\[1em] Exam Roll No. : 24419STC053}
\date{\today}

\newfontface\myfont{Myfont1-Regular.ttf}[LetterSpace=0.05em]

\newfontface\cbfont{CaveatBrush-Regular.ttf}
% how to use --- \myfont --write text here--
\IfFileExists{upquote.sty}{\usepackage{upquote}}{}
\begin{document}

\maketitle


\section*{\faArrowAltCircleRight[regular] \textcolor{blue}{Question}}

\hspace{1cm} Use rejection sampling technique to generate random sample from a distribution with PDF
\begin{equation*}
f(x) = 20 \cdot x (1-x)^3 \cdot I_{(0, 1)}(x)
\end{equation*}

\section*{\faArrowAltCircleRight[regular] \textcolor{blue}{Algorithm}}

Suppose we have to generate random sample from $p(x)$, we call it our \textbf{target distribution}. We choose a \textbf{proposal distribution} $q(x)$, say.

\begin{enumerate}[I.]
\item Generate $x_i \sim q(x)$ $\forall$ $i = 1(1)n$.
\item Generate $u_i \sim U(0, 1)$ $\forall$ $i = 1(1)n$.
\item Accept $x_i$ if $$u_i \leq \dfrac{p(x_i)}{M \cdot q(x_i)}$$ where $M > 0$ is a suitable constant so that $p(x) \leq M \cdot q(x)$ is satisfied.
\end{enumerate}

Accepted samples follow the target distribution $p(x)$. \\[0.2em]

Here, we are given that $p(x) = 20 \cdot x (1-x)^3$. We take $q(x) = 1 \cdot I_{(0, 1)}(x)$ \textit{i.e.} $U(0, 1)$ distribution. \\[0.2em]

To find constant $M$ recall that we need $p(x) \leq M \cdot q(x) \Rightarrow p(x) \leq M$. Notice, $p(x)$ is the PDF of a $\beta_1(2, 4)$ distribution which has mode $$\dfrac{2-1}{2+4-2} = \dfrac{1}{4}.$$ So $p(0.25) \approx 2.11$ is a suitable choice for $M$.

\newpage

\section*{\faArrowAltCircleRight[regular] \textcolor{blue}{R Program}}

\begin{knitrout}
\definecolor{shadecolor}{rgb}{0.969, 0.969, 0.969}\color{fgcolor}\begin{kframe}
\begin{alltt}
\hlkwd{set.seed}\hldef{(}\hlnum{22}\hldef{)}
\end{alltt}
\end{kframe}
\end{knitrout}

\begin{knitrout}
\definecolor{shadecolor}{rgb}{0.969, 0.969, 0.969}\color{fgcolor}\begin{kframe}
\begin{alltt}
\hldef{n} \hlkwb{<-} \hlnum{2000}
\end{alltt}
\end{kframe}
\end{knitrout}

\begin{knitrout}
\definecolor{shadecolor}{rgb}{0.969, 0.969, 0.969}\color{fgcolor}\begin{kframe}
\begin{alltt}
\hldef{u} \hlkwb{<-} \hlkwd{runif}\hldef{(n)}
\hldef{x} \hlkwb{<-} \hlkwd{runif}\hldef{(n)}
\end{alltt}
\end{kframe}
\end{knitrout}

\begin{knitrout}
\definecolor{shadecolor}{rgb}{0.969, 0.969, 0.969}\color{fgcolor}\begin{kframe}
\begin{alltt}
\hldef{p} \hlkwb{<-} \hlkwa{function}\hldef{(}\hlkwc{x}\hldef{)} \hlnum{20} \hlopt{*} \hldef{x} \hlopt{*} \hldef{(}\hlnum{1}\hlopt{-}\hldef{x)}\hlopt{^}\hlnum{3}
\end{alltt}
\end{kframe}
\end{knitrout}

\begin{knitrout}
\definecolor{shadecolor}{rgb}{0.969, 0.969, 0.969}\color{fgcolor}\begin{kframe}
\begin{alltt}
\hldef{M} \hlkwb{<-} \hlnum{2.11}
\end{alltt}
\end{kframe}
\end{knitrout}

\begin{knitrout}
\definecolor{shadecolor}{rgb}{0.969, 0.969, 0.969}\color{fgcolor}\begin{kframe}
\begin{alltt}
\hldef{good} \hlkwb{<-} \hldef{u} \hlopt{<=} \hlkwd{p}\hldef{(x)} \hlopt{/} \hldef{M}
\hldef{accepted} \hlkwb{<-} \hldef{x[good]}
\end{alltt}
\end{kframe}
\end{knitrout}

\begin{knitrout}
\definecolor{shadecolor}{rgb}{0.969, 0.969, 0.969}\color{fgcolor}\begin{kframe}
\begin{alltt}
\hldef{count_accepted} \hlkwb{<-} \hlkwd{length}\hldef{(accepted)} \hlcom{# or sum(good)}
\hldef{count_accepted}
\end{alltt}
\begin{verbatim}
## [1] 929
\end{verbatim}
\end{kframe}
\end{knitrout}

Thus we have a sample of size $929$ from $p(x)$. \\[0.3em]

The \textbf{acceptance rate} was $0.4645$ whereas the \textbf{theoretical acceptance rate} is $0.4739336$.

\newpage

\begin{knitrout}
\definecolor{shadecolor}{rgb}{0.969, 0.969, 0.969}\color{fgcolor}\begin{kframe}
\begin{alltt}
\hldef{df} \hlkwb{<-} \hlkwd{data.frame}\hldef{(}\hlkwc{x} \hldef{= x,} \hlkwc{u} \hldef{= u,} \hlkwc{result} \hldef{=} \hlkwd{ifelse}\hldef{(good,} \hlsng{"accepted"}\hldef{,} \hlsng{"rejected"}\hldef{))}
\end{alltt}
\end{kframe}
\end{knitrout}

Let us have a histogram of the generated sample.



\begin{knitrout}
\definecolor{shadecolor}{rgb}{0.969, 0.969, 0.969}\color{fgcolor}\begin{kframe}
\begin{alltt}
\hldef{df} \hlopt
  \hlkwd{filter}\hldef{(result} \hlopt{==} \hlsng{"accepted"}\hldef{)} \hlopt
    \hlkwd{ggplot}\hldef{(}\hlkwd{aes}\hldef{(}\hlkwc{x} \hldef{= x))} \hlopt{+}
    \hlkwd{geom_histogram}\hldef{(}\hlkwd{aes}\hldef{(}\hlkwc{y} \hldef{=} \hlkwd{after_stat}\hldef{(density)),} \hlkwc{bins} \hldef{=} \hlnum{10}\hldef{,}
                   \hlkwc{linewidth} \hldef{=} \hlnum{1}\hldef{,} \hlkwc{color} \hldef{=} \hlsng{"black"}\hldef{,} \hlkwc{fill} \hldef{=} \hlsng{"#27EEF5"}\hldef{)} \hlopt{+}
  \hlkwd{stat_function}\hldef{(}\hlkwc{fun} \hldef{= p,} \hlkwc{linewidth} \hldef{=} \hlnum{1}\hldef{)} \hlopt{+}
  \hlkwd{labs}\hldef{(}\hlkwc{x} \hldef{=} \hlsng{"X"}\hldef{,} \hlkwc{y} \hldef{=} \hlsng{"Density"}\hldef{,}
       \hlkwc{title} \hldef{=} \hlsng{"Historgram of Generated Random Sample"}\hldef{)}
\end{alltt}
\end{kframe}
\includegraphics[width=\maxwidth]{figure/unnamed-chunk-10-1} 
\end{knitrout}

\section*{\faArrowAltCircleRight[regular] \textcolor{blue}{Conclusion}}

\smallpencil \hspace{0.1cm} {\setlength{\spaceskip}{1em plus 0.5em minus 0.5em} \fontsize{17}{20}\myfont The actual density curve fits the histogram well. \par}

\newpage

Let us have a simple visualization of the rejection sampling technique.

\begin{knitrout}
\definecolor{shadecolor}{rgb}{0.969, 0.969, 0.969}\color{fgcolor}\begin{kframe}
\begin{alltt}
\hldef{p_scaled} \hlkwb{<-} \hlkwa{function}\hldef{(}\hlkwc{x}\hldef{)} \hlkwd{p}\hldef{(x)} \hlopt{/} \hldef{M}
\end{alltt}
\end{kframe}
\end{knitrout}

\begin{knitrout}
\definecolor{shadecolor}{rgb}{0.969, 0.969, 0.969}\color{fgcolor}\begin{kframe}
\begin{alltt}
\hldef{df} \hlopt
  \hlkwd{ggplot}\hldef{(}\hlkwd{aes}\hldef{(}\hlkwc{x} \hldef{= x,} \hlkwc{y} \hldef{= u,} \hlkwc{shape} \hldef{= result))} \hlopt{+}
  \hlkwd{geom_point}\hldef{()} \hlopt{+}
  \hlkwd{scale_shape_manual}\hldef{(}\hlkwc{values} \hldef{=} \hlkwd{c}\hldef{(}\hlsng{"accepted"} \hldef{=} \hlnum{16}\hldef{,}
                                \hlsng{"rejected"} \hldef{=} \hlnum{1}\hldef{))} \hlopt{+}
  \hlkwd{stat_function}\hldef{(}\hlkwc{fun} \hldef{= p_scaled,} \hlkwc{linewidth} \hldef{=} \hlnum{1.25}\hldef{,} \hlkwc{inherit.aes} \hldef{=} \hlnum{FALSE}\hldef{)} \hlopt{+}
  \hlkwd{theme}\hldef{(}\hlkwc{legend.title} \hldef{=} \hlkwd{element_blank}\hldef{())}
\end{alltt}
\end{kframe}
\includegraphics[width=\maxwidth]{figure/unnamed-chunk-12-1} 
\end{knitrout}

\smallpencil \hspace{0.1cm} The $(x, u)$ pairs that fall under the scaled target density are accepted and rest are rejected.

\end{document}
