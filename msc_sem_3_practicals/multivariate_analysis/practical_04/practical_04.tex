\documentclass[11pt, a4paper]{article}\usepackage[]{graphicx}\usepackage[]{xcolor}
% maxwidth is the original width if it is less than linewidth
% otherwise use linewidth (to make sure the graphics do not exceed the margin)
\makeatletter
\def\maxwidth{ %
  \ifdim\Gin@nat@width>\linewidth
    \linewidth
  \else
    \Gin@nat@width
  \fi
}
\makeatother

\definecolor{fgcolor}{rgb}{0.345, 0.345, 0.345}
\newcommand{\hlnum}[1]{\textcolor[rgb]{0.686,0.059,0.569}{#1}}%
\newcommand{\hlsng}[1]{\textcolor[rgb]{0.192,0.494,0.8}{#1}}%
\newcommand{\hlcom}[1]{\textcolor[rgb]{0.678,0.584,0.686}{\textit{#1}}}%
\newcommand{\hlopt}[1]{\textcolor[rgb]{0,0,0}{#1}}%
\newcommand{\hldef}[1]{\textcolor[rgb]{0.345,0.345,0.345}{#1}}%
\newcommand{\hlkwa}[1]{\textcolor[rgb]{0.161,0.373,0.58}{\textbf{#1}}}%
\newcommand{\hlkwb}[1]{\textcolor[rgb]{0.69,0.353,0.396}{#1}}%
\newcommand{\hlkwc}[1]{\textcolor[rgb]{0.333,0.667,0.333}{#1}}%
\newcommand{\hlkwd}[1]{\textcolor[rgb]{0.737,0.353,0.396}{\textbf{#1}}}%
\let\hlipl\hlkwb

\usepackage{framed}
\makeatletter
\newenvironment{kframe}{%
 \def\at@end@of@kframe{}%
 \ifinner\ifhmode%
  \def\at@end@of@kframe{\end{minipage}}%
  \begin{minipage}{\columnwidth}%
 \fi\fi%
 \def\FrameCommand##1{\hskip\@totalleftmargin \hskip-\fboxsep
 \colorbox{shadecolor}{##1}\hskip-\fboxsep
     % There is no \\@totalrightmargin, so:
     \hskip-\linewidth \hskip-\@totalleftmargin \hskip\columnwidth}%
 \MakeFramed {\advance\hsize-\width
   \@totalleftmargin\z@ \linewidth\hsize
   \@setminipage}}%
 {\par\unskip\endMakeFramed%
 \at@end@of@kframe}
\makeatother

\definecolor{shadecolor}{rgb}{.97, .97, .97}
\definecolor{messagecolor}{rgb}{0, 0, 0}
\definecolor{warningcolor}{rgb}{1, 0, 1}
\definecolor{errorcolor}{rgb}{1, 0, 0}
\newenvironment{knitrout}{}{} % an empty environment to be redefined in TeX

\usepackage{alltt}

\usepackage[top = 1 in, bottom = 0.9 in, left = 1 in, right = 1 in]{geometry}

\usepackage{amsmath, amssymb, amsfonts}
\usepackage{enumerate}
\usepackage{array}
\usepackage{multirow}
\usepackage{dingbat}
\usepackage{fontawesome5}
\usepackage{tasks}
\usepackage{undertilde}
\usepackage{bbding}
\usepackage{twemojis}
% how to use bull's eye ----- \scalebox{2.0}{\twemoji{bullseye}}
\usepackage{fontspec}
\usepackage{customdice}
% how to put dice face ------ \dice{2}

\title{MSMS 308 : Practical 04}
\author{Ananda Biswas}
\date{\today}

\newfontface\myfont{Myfont1-Regular.ttf}[LetterSpace=0.05em]
% how to use ---- {\setlength{\spaceskip}{1em plus 0.5em minus 0.5em} \fontsize{17}{20}\myfont --write text here-- \par}

\newfontface\cbfont{CaveatBrush-Regular.ttf}
% how to use -- \cbfont --write text here--
\IfFileExists{upquote.sty}{\usepackage{upquote}}{}
\begin{document}

\maketitle


\section*{\faArrowAltCircleRight[regular] \textcolor{blue}{Question}}

\hspace{1cm} Let the components of $\utilde{X}$ correspond to the scores on tests in Arithmetic speed $(X_1)$, Arithmetic power $(X_2)$, Memory for words $(X_3)$, Memory for meaningful symbols $(X_4)$ and Memory for meaningless symbols $(X_5)$. The observed correlations in a sample of $140$ observations are given below:

\begin{gather*}
R = 
\begin{bmatrix}
1.0000 & 0.4248 & 0.0420 & 0.0215 & 0.0573 \\
0.4248 & 1.0000 & 0.1487 & 0.2489 & 0.2843 \\
0.0420 & 0.1487 & 1.0000 & 0.6693 & 0.4662 \\
0.0215 & 0.2489 & 0.6693 & 1.0000 & 0.6915 \\
0.0573 & 0.2843 & 0.4662 & 0.6915 & 1.0000
\end{bmatrix}
\end{gather*}

\begin{enumerate}[(a)]
\item Find the partial correlation coefficient between $X_1$ and $X_2$ holding other variables fixed.

\item Find the multiple correlation coefficient $R_{1.2345}$.
\end{enumerate}

\section*{\faArrowAltCircleRight[regular] \textcolor{blue}{R Program}}

\begin{knitrout}
\definecolor{shadecolor}{rgb}{0.969, 0.969, 0.969}\color{fgcolor}\begin{kframe}
\begin{alltt}
\hldef{R} \hlkwb{<-} \hlkwd{matrix}\hldef{(}\hlkwd{c}\hldef{(}\hlnum{1.0000}\hldef{,} \hlnum{0.4248}\hldef{,} \hlnum{0.0420}\hldef{,} \hlnum{0.0215}\hldef{,} \hlnum{0.0573}\hldef{,}
              \hlnum{0.4248}\hldef{,} \hlnum{1.0000}\hldef{,} \hlnum{0.1487}\hldef{,} \hlnum{0.2489}\hldef{,} \hlnum{0.2843}\hldef{,}
              \hlnum{0.0420}\hldef{,} \hlnum{0.1487}\hldef{,} \hlnum{1.0000}\hldef{,} \hlnum{0.6693}\hldef{,} \hlnum{0.4662}\hldef{,}
              \hlnum{0.0215}\hldef{,} \hlnum{0.2489}\hldef{,} \hlnum{0.6693}\hldef{,} \hlnum{1.0000}\hldef{,} \hlnum{0.6915}\hldef{,}
              \hlnum{0.0573}\hldef{,} \hlnum{0.2843}\hldef{,} \hlnum{0.4662}\hldef{,} \hlnum{0.6915}\hldef{,} \hlnum{1.0000}\hldef{),} \hlkwc{nrow} \hldef{=} \hlnum{5}\hldef{,} \hlkwc{ncol} \hldef{=} \hlnum{5}\hldef{,} \hlkwc{byrow} \hldef{=} \hlnum{TRUE}\hldef{)}
\end{alltt}
\end{kframe}
\end{knitrout}

\dice{1} \hspace{0.1cm} The partial correlation coefficient between $X_1$ and $X_2$ is $r_{12.345} = - \dfrac{R_{12}}{\sqrt{R_{11} R_{22}}}$.

\begin{knitrout}
\definecolor{shadecolor}{rgb}{0.969, 0.969, 0.969}\color{fgcolor}\begin{kframe}
\begin{alltt}
\hldef{cofactor} \hlkwb{<-} \hlkwa{function}\hldef{(}\hlkwc{mat}\hldef{,} \hlkwc{i}\hldef{,} \hlkwc{j}\hldef{) \{}
  \hldef{minor} \hlkwb{<-} \hldef{mat[}\hlopt{-}\hldef{i,} \hlopt{-}\hldef{j]}
  \hlkwd{return} \hldef{((}\hlopt{-}\hlnum{1}\hldef{)}\hlopt{^}\hldef{(i} \hlopt{+} \hldef{j)} \hlopt{*} \hlkwd{det}\hldef{(minor))}
\hldef{\}}
\end{alltt}
\end{kframe}
\end{knitrout}

\begin{knitrout}
\definecolor{shadecolor}{rgb}{0.969, 0.969, 0.969}\color{fgcolor}\begin{kframe}
\begin{alltt}
\hldef{r_12.345} \hlkwb{<-} \hlopt{-} \hlkwd{cofactor}\hldef{(R,} \hlnum{1}\hldef{,} \hlnum{2}\hldef{)} \hlopt{/} \hlkwd{sqrt}\hldef{(}\hlkwd{cofactor}\hldef{(R,} \hlnum{1}\hldef{,} \hlnum{1}\hldef{)} \hlopt{*} \hlkwd{cofactor}\hldef{(R,} \hlnum{2}\hldef{,} \hlnum{2}\hldef{))}
\hldef{r_12.345}
\end{alltt}
\begin{verbatim}
## [1] 0.4314625
\end{verbatim}
\end{kframe}
\end{knitrout}

$\therefore$ $r_{12.345} = 0.4314625$.\\[1em]


\dice{2} \hspace{0.1cm} The multiple correlation coefficient between $X_1$ and $X_2, X_3, X_4, X_5$ in terms of correlation matrix $R$ is given by $$R_{1.2345} = \sqrt{1 - \dfrac{|R|}{|R_2|}}$$.

\begin{knitrout}
\definecolor{shadecolor}{rgb}{0.969, 0.969, 0.969}\color{fgcolor}\begin{kframe}
\begin{alltt}
\hldef{R_1.2345} \hlkwb{<-} \hlkwd{sqrt}\hldef{(}\hlnum{1} \hlopt{-} \hlkwd{det}\hldef{(R)} \hlopt{/} \hlkwd{det}\hldef{(R[}\hlopt{-}\hlnum{1}\hldef{,} \hlopt{-}\hlnum{1}\hldef{]))}
\hldef{R_1.2345}
\end{alltt}
\begin{verbatim}
## [1] 0.436401
\end{verbatim}
\end{kframe}
\end{knitrout}

$\therefore$ $R_{1.2345} = 0.436401.$



\end{document}
