\documentclass[11pt, a4paper]{article}\usepackage[]{graphicx}\usepackage[]{xcolor}
% maxwidth is the original width if it is less than linewidth
% otherwise use linewidth (to make sure the graphics do not exceed the margin)
\makeatletter
\def\maxwidth{ %
  \ifdim\Gin@nat@width>\linewidth
    \linewidth
  \else
    \Gin@nat@width
  \fi
}
\makeatother

\definecolor{fgcolor}{rgb}{0.345, 0.345, 0.345}
\newcommand{\hlnum}[1]{\textcolor[rgb]{0.686,0.059,0.569}{#1}}%
\newcommand{\hlsng}[1]{\textcolor[rgb]{0.192,0.494,0.8}{#1}}%
\newcommand{\hlcom}[1]{\textcolor[rgb]{0.678,0.584,0.686}{\textit{#1}}}%
\newcommand{\hlopt}[1]{\textcolor[rgb]{0,0,0}{#1}}%
\newcommand{\hldef}[1]{\textcolor[rgb]{0.345,0.345,0.345}{#1}}%
\newcommand{\hlkwa}[1]{\textcolor[rgb]{0.161,0.373,0.58}{\textbf{#1}}}%
\newcommand{\hlkwb}[1]{\textcolor[rgb]{0.69,0.353,0.396}{#1}}%
\newcommand{\hlkwc}[1]{\textcolor[rgb]{0.333,0.667,0.333}{#1}}%
\newcommand{\hlkwd}[1]{\textcolor[rgb]{0.737,0.353,0.396}{\textbf{#1}}}%
\let\hlipl\hlkwb

\usepackage{framed}
\makeatletter
\newenvironment{kframe}{%
 \def\at@end@of@kframe{}%
 \ifinner\ifhmode%
  \def\at@end@of@kframe{\end{minipage}}%
  \begin{minipage}{\columnwidth}%
 \fi\fi%
 \def\FrameCommand##1{\hskip\@totalleftmargin \hskip-\fboxsep
 \colorbox{shadecolor}{##1}\hskip-\fboxsep
     % There is no \\@totalrightmargin, so:
     \hskip-\linewidth \hskip-\@totalleftmargin \hskip\columnwidth}%
 \MakeFramed {\advance\hsize-\width
   \@totalleftmargin\z@ \linewidth\hsize
   \@setminipage}}%
 {\par\unskip\endMakeFramed%
 \at@end@of@kframe}
\makeatother

\definecolor{shadecolor}{rgb}{.97, .97, .97}
\definecolor{messagecolor}{rgb}{0, 0, 0}
\definecolor{warningcolor}{rgb}{1, 0, 1}
\definecolor{errorcolor}{rgb}{1, 0, 0}
\newenvironment{knitrout}{}{} % an empty environment to be redefined in TeX

\usepackage{alltt}

\usepackage[top = 0.8 in, bottom = 0.8 in, left = 1 in, right = 1 in]{geometry}
\usepackage{amsmath, amssymb, amsfonts}
\usepackage{enumerate}
\usepackage{array}
\usepackage{multirow}
\usepackage{dingbat}
\usepackage{fontawesome5}
\usepackage{tasks}
\usepackage{bbding}
\usepackage{twemojis}
% how to use bull's eye ----- \scalebox{2.0}{\twemoji{bullseye}}
\usepackage{fontspec}
\usepackage{customdice}
% how to put dice face ------ \dice{2}

\title{MSMS 308 : Practical 06}
\author{Ananda Biswas \\[1em] Exam Roll No. : 24419STC053}
\date{\today}

\newfontface\myfont{Myfont1-Regular.ttf}[LetterSpace=0.05em]
% how to use ---- {\setlength{\spaceskip}{1em plus 0.5em minus 0.5em} \fontsize{17}{20}\myfont --write text here-- \par}

\newfontface\cbfont{CaveatBrush-Regular.ttf}
% how to use --- \myfont --write text here--
\IfFileExists{upquote.sty}{\usepackage{upquote}}{}
\begin{document}

\maketitle


\section*{\faArrowAltCircleRight[regular] \textcolor{blue}{Question}}

\hspace{1cm} Obtain a sufficiently large sample from an exponential distribution under a Type II censoring scheme. After generating the sample, calculate the maximum likelihood (ML)
estimate of the distribution parameter. Also, evaluate the performance of the estimate by
computing the bias, variance, and mean squared error (MSE) for different sample sizes.

\section*{\faArrowAltCircleRight[regular] \textcolor{blue}{Theory}}

Let \(T_1,\dots,T_n \stackrel{iid}{\sim} \mathrm{Exp}(\lambda)\) with density \(f(t)\) and survival \(S(t)\). Under Type II censoring we observe the first \(r\) failure times \(0 < t_{(1)} \leq \cdots \leq t_{(r)}\); the remaining \(n-r\) units are right-censored at the common time \(t_{(r)}\).

\medskip
The likelihood function is
\begin{align*}
L(\lambda) &= \dfrac{n!}{(n-r)!} \cdot \left[\prod_{i=1}^{r} f\!\big(t_{(i)}\big)\right] \cdot \Big[S\!\big(t_{(r)}\big)\Big]^{\,n-r}. \\[4pt]
&= \dfrac{n!}{(n-r)!} \; \prod_{i=1}^{r} \Big(\lambda e^{-\lambda t_{(i)}}\Big) \; \cdot \; \Big(e^{-\lambda t_{(r)}}\Big)^{n-r} \\[4pt]
&= \dfrac{n!}{(n-r)!} \; \lambda^{r} \; \exp\!\left(-\lambda \sum_{i=1}^{r} t_{(i)}\right) \; \exp\!\left(-\lambda (n-r)\,t_{(r)}\right) \\[4pt]
&= \dfrac{n!}{(n-r)!} \; \lambda^{r} \; \exp\!\left\{-\lambda \Bigg[\sum_{i=1}^{r} t_{(i)} + (n-r)\,t_{(r)}\Bigg]\right\}
\end{align*}


\medskip
The log-likelihood function is
\[
\ell(\lambda)
= \log \left( \dfrac{n!}{(n-r)!} \right) + r\log\lambda - \lambda \Bigg[\sum_{i=1}^{r} t_{(i)} + (n-r)\,t_{(r)}\Bigg].
\]

\medskip
Now we differentiate $\ell(\lambda)$ w.r.t. $\lambda$ and set the derivative to zero.
\begin{align*}
\dfrac{\partial\ell}{\partial\lambda} &= \dfrac{r}{\lambda} - \Bigg[\sum_{i=1}^{r} t_{(i)} + (n-r)\,t_{(r)}\Bigg] = 0 \\[4pt]
\Rightarrow \dfrac{r}{\lambda} &= \sum_{i=1}^{r} t_{(i)} + (n-r)\,t_{(r)} \\[4pt]
\end{align*}

$$ \therefore \widehat{\lambda}_{MLE} = \dfrac{r}{\,\sum \limits_{i=1}^{r} t_{(i)} + (n-r)\,t_{(r)}\,} $$


\section*{\faArrowAltCircleRight[regular] \textcolor{blue}{R Program}}

Let $n = 100$ and $r = 75$.

\begin{knitrout}
\definecolor{shadecolor}{rgb}{0.969, 0.969, 0.969}\color{fgcolor}\begin{kframe}
\begin{alltt}
\hldef{n} \hlkwb{<-} \hlnum{100}\hldef{; r} \hlkwb{<-} \hlnum{75}
\end{alltt}
\end{kframe}
\end{knitrout}

\begin{knitrout}
\definecolor{shadecolor}{rgb}{0.969, 0.969, 0.969}\color{fgcolor}\begin{kframe}
\begin{alltt}
\hldef{x} \hlkwb{<-} \hlkwd{sort}\hldef{(}\hlkwd{rexp}\hldef{(n,} \hlkwc{rate} \hldef{=} \hlnum{2}\hldef{))}
\hldef{t_values} \hlkwb{<-} \hldef{x[}\hlnum{1}\hlopt{:}\hldef{r]}
\end{alltt}
\end{kframe}
\end{knitrout}

\begin{knitrout}
\definecolor{shadecolor}{rgb}{0.969, 0.969, 0.969}\color{fgcolor}\begin{kframe}
\begin{alltt}
\hldef{lambda_hat} \hlkwb{<-} \hldef{r} \hlopt{/} \hldef{(}\hlkwd{sum}\hldef{(t_values)} \hlopt{+} \hldef{(n} \hlopt{-} \hldef{r)} \hlopt{*} \hlkwd{max}\hldef{(t_values)); lambda_hat}
\end{alltt}
\begin{verbatim}
## [1] 2.247462
\end{verbatim}
\end{kframe}
\end{knitrout}

$\therefore$ $\widehat{\lambda}_{MLE} = 2.2474623$. \\[1em]

Now we shall compare the estimates for different sample sizes by evaluating their bias, variance and MSE.

\begin{knitrout}
\definecolor{shadecolor}{rgb}{0.969, 0.969, 0.969}\color{fgcolor}\begin{kframe}
\begin{alltt}
\hldef{bias} \hlkwb{=} \hldef{variance} \hlkwb{=} \hldef{MSE} \hlkwb{=} \hlkwd{c}\hldef{()}

\hldef{n} \hlkwb{<-} \hlkwd{c}\hldef{(}\hlnum{50}\hldef{,} \hlnum{100}\hldef{,} \hlnum{150}\hldef{,} \hlnum{200}\hldef{,} \hlnum{250}\hldef{,} \hlnum{300}\hldef{)}

\hlcom{# suppose we censor after having first 75% observations}
\hldef{r} \hlkwb{<-} \hlkwd{round}\hldef{(n} \hlopt{*} \hlnum{0.75}\hldef{)}
\end{alltt}
\end{kframe}
\end{knitrout}

\begin{knitrout}
\definecolor{shadecolor}{rgb}{0.969, 0.969, 0.969}\color{fgcolor}\begin{kframe}
\begin{alltt}
\hldef{true_lambda} \hlkwb{<-} \hlnum{1}
\end{alltt}
\end{kframe}
\end{knitrout}

\begin{knitrout}
\definecolor{shadecolor}{rgb}{0.969, 0.969, 0.969}\color{fgcolor}\begin{kframe}
\begin{alltt}
\hlkwa{for} \hldef{(i} \hlkwa{in} \hlnum{1}\hlopt{:}\hlkwd{length}\hldef{(r)) \{}

  \hldef{lambda_hat} \hlkwb{<-} \hlkwd{c}\hldef{()}

  \hlkwa{for} \hldef{(j} \hlkwa{in} \hlnum{1}\hlopt{:}\hlnum{1000}\hldef{) \{}
    \hldef{x} \hlkwb{<-} \hlkwd{sort}\hldef{(}\hlkwd{rexp}\hldef{(n[i],} \hlkwc{rate} \hldef{= true_lambda))}

    \hldef{s} \hlkwb{<-} \hldef{x[}\hlnum{1}\hlopt{:}\hldef{r[i]]}

    \hldef{lambda_hat[j]} \hlkwb{<-} \hldef{r[i]} \hlopt{/} \hldef{(}\hlkwd{sum}\hldef{(s)} \hlopt{+} \hldef{(n[i]} \hlopt{-} \hldef{r[i])} \hlopt{*} \hlkwd{max}\hldef{(s))}
  \hldef{\}}

  \hldef{bias[i]} \hlkwb{<-} \hlkwd{mean}\hldef{(lambda_hat)} \hlopt{-} \hldef{true_lambda}

  \hldef{variance[i]} \hlkwb{<-} \hlkwd{mean}\hldef{( (lambda_hat} \hlopt{-} \hlkwd{mean}\hldef{(lambda_hat))}\hlopt{^}\hlnum{2} \hldef{)}
\hldef{\}}

\hldef{MSE} \hlkwb{<-} \hldef{bias}\hlopt{^}\hlnum{2} \hlopt{+} \hldef{variance}
\end{alltt}
\end{kframe}
\end{knitrout}

\begin{knitrout}
\definecolor{shadecolor}{rgb}{0.969, 0.969, 0.969}\color{fgcolor}\begin{kframe}
\begin{alltt}
\hldef{df} \hlkwb{<-} \hlkwd{data.frame}\hldef{(}\hlkwc{Sample.Size} \hldef{= n,}
                 \hlkwc{Observed} \hldef{= r,} \hlkwc{Censored} \hldef{= n} \hlopt{-} \hldef{r,}
                 \hlkwc{Bias} \hldef{= bias,} \hlkwc{Variance} \hldef{= variance,} \hlkwc{MSE} \hldef{= MSE)}
\end{alltt}
\end{kframe}
\end{knitrout}

\begin{knitrout}
\definecolor{shadecolor}{rgb}{0.969, 0.969, 0.969}\color{fgcolor}\begin{kframe}
\begin{alltt}
\hlkwd{library}\hldef{(stargazer)}
\end{alltt}
\end{kframe}
\end{knitrout}

Bias, variance and MSE of the estimates for different sample sizes are as follows :

\begin{kframe}
\begin{alltt}
\hlkwd{stargazer}\hldef{(df,} \hlkwc{summary} \hldef{=} \hlnum{FALSE}\hldef{,} \hlkwc{rownames} \hldef{=} \hlnum{FALSE}\hldef{,} \hlkwc{digits} \hldef{=} \hlnum{5}\hldef{)}
\end{alltt}
\end{kframe}
% Table created by stargazer v.5.2.3 by Marek Hlavac, Social Policy Institute. E-mail: marek.hlavac at gmail.com
% Date and time: Mon, Oct 27, 2025 - 22:53:31
\begin{table}[!htbp] \centering 
  \caption{} 
  \label{} 
\begin{tabular}{@{\extracolsep{5pt}} cccccc} 
\\[-1.8ex]\hline 
\hline \\[-1.8ex] 
Sample.Size & Observed & Censored & Bias & Variance & MSE \\ 
\hline \\[-1.8ex] 
$50$ & $38$ & $12$ & $0.03321$ & $0.02758$ & $0.02869$ \\ 
$100$ & $75$ & $25$ & $0.01595$ & $0.01363$ & $0.01389$ \\ 
$150$ & $112$ & $38$ & $0.00654$ & $0.00936$ & $0.00940$ \\ 
$200$ & $150$ & $50$ & $0.00144$ & $0.00658$ & $0.00659$ \\ 
$250$ & $188$ & $62$ & $0.00470$ & $0.00527$ & $0.00530$ \\ 
$300$ & $225$ & $75$ & $0.00800$ & $0.00508$ & $0.00514$ \\ 
\hline \\[-1.8ex] 
\end{tabular} 
\end{table} 



\smallpencil \hspace{0.1cm} {\setlength{\spaceskip}{1em plus 0.5em minus 0.5em} \fontsize{17}{20}\myfont With increasing sample size, there is a steady decrease in bias, variance and mean squared error of the estimates. \par}

\end{document}
